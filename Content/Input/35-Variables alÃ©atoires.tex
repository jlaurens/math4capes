% !TEX encoding = UTF-8
% !TEX program = xelatex
% !TEX root =../Main/main.tex

\section{Variables aléatoires}
Variable aléatoires sur un univers fini : lois usuelles (lois uniformes, lois binomiales), variables aléatoires
indépendantes, espérance, variance et écart-type. Variables aléatoires discrètes : espérance et variance, lois de
Poisson, lois géométriques. Lois exponentielles, loi faible des grands nombres.

On rappelle que \(𝟙_{𝐸}\) est l'indicatrice de l'ensemble \(𝐸\) qui vaut 1 sur \(𝐸\) et 0 sur son complémentaire.
\subsection{Variables aléatoires discrètes}
\subsubsection{Définitions}
Un espace probabilisé \((Ω,𝒯,ℙ_Ω)\) et un espace mesurable \((Ω',ℰ)\) étant donnés.
\paragraph{Cas général}
\begin{definition}
[Variable aléatoire]
Une \textbf{variable aléatoire} est une application \(𝑋\) de \(Ω\) dans \(Ω'\) telle que la préimage par \(𝑋\) de toute
partie mesurable est un événement.
\end{definition}
\begin{remark}
De manière équivalente : la préimage par \(𝑋\) de tout élément de \(ℰ\) est un élément de
\(𝒯\).
\end{remark}
\begin{theorem}
[Loi de la variable aléatoire]
L'application qui à tout élément \(𝐸\) de \(ℰ\) associe \(ℙ_Ω\bigl(𝑋^{-1}(𝐸)\bigr)\) est une mesure de
probabilités sur \(\left(Ω',ℰ\right)\), elle est notée \(ℙ_{𝑋}\).

C'est la \mykeyword{loi de la variable aléatoire} \(𝑋\)
\end{theorem}
\begin{proof}
On a :
\begin{enumerate}
\item
\vskip\abovedisplayskip
\(
0≤ℙ_{𝑋}(𝐸)=ℙ_Ω\bigl(𝑋^{-1}(𝐸)\bigr)≤1
\)
\item
\vskip\abovedisplayskip
\(
ℙ_{𝑋}(Ω')=ℙ_Ω\bigl(𝑋-1(Ω')\bigr)=ℙ_Ω(Ω)=1
\)
\item
\vskip\belowdisplayskip
Si \((𝐸_{𝑖})_{𝑖∈𝐼}\) est une famille dénombrable d'éléments de \(ℰ\) deux à
deux disjoints, alors \(\bigl(𝑋^{-1}(𝐸_{𝑖})\bigr)_{𝑖∈𝐼}\) est une famille dénombrable d'événements de \(Ω\) deux à deux
disjoints. En effet, par contraposition :
\end{enumerate}
\begin{align*}
𝑋^{-1}(𝐸_{𝑖})∩𝑋^{-1}(𝐸_{𝑗})≠ ∅
&{}⟹∃𝜔,\;𝜔∈𝑋^{-1}(𝐸_{𝑖})\text{ et
}𝜔∈𝑋^{-1}(𝐸_{𝑗})
\\&{}⟹
∃𝜔,\;𝑋(𝜔)∈𝐸_{𝑖}\myand𝑋(𝜔)∈𝐸_{𝑗}
\\&{}⟹
∃𝑥,\;𝑥∈𝐸_{𝑖}\myand𝑥∈𝐸_{𝑗}
\\&{}⟹
𝐸_{𝑖}∩𝐸_{𝑗}≠ ∅
\end{align*}
ainsi,
\begin{align*}
ℙ_{𝑋}\Bigl(\bigcup_{𝑖∈𝐼}𝐸_{𝑖}\Bigr)
&{}=
ℙ_Ω\Bigl(𝑋^{-1}\Bigl(\bigcup_{𝑖∈𝐼}𝐸_{𝑖}\Bigr)\Bigr)
=
ℙ_Ω\Bigl(\bigcup_{𝑖∈𝐼}𝑋^{-1}(𝐸_{𝑖})\Bigr)
\\&{}=∑_{𝑖∈𝐼}ℙ_Ω\bigl(𝑋^{-1}(𝐸_{𝑖})\bigr)
=∑_{𝑖∈𝐼}ℙ_{𝑋}(𝐸_{𝑖})
\qedhere
\end{align*}
\end{proof}
%
\begin{terminology}
Sont synonymes
\begin{itemize}
\item
\(ℙ(𝑋∈𝐸)\), \(ℙ_{𝑋}(𝐸)\) et \(ℙ_Ω(𝑋^{-1}(𝐸))\).
\item
\(ℙ(𝑋=𝑥)\) et  \(ℙ_{𝑋}\bigl(\{𝑥\}\bigr)\) et \(ℙ_Ω\bigl(𝑋^{-1}\bigl(\{𝑥\}\bigr)\bigr)\).
\end{itemize}
\end{terminology}
%
\begin{remark}
Ce n'est plus une notation fonctionnelle mais c'est pratique d'un certain point de vue car cela permet de cacher \(Ω\) et de se consacrer à l'essentiel.
\end{remark}
\begin{terminology}
Si \(𝑋\) prend des valeurs réelles, sont synonymes
\begin{itemize}
\item
\(ℙ(𝑋<𝑎)\) et \(ℙ(𝑋∈\left]-∞,𝑎\right[)\),
\item
\(ℙ(𝑋⩽𝑎)\) et \(ℙ(𝑋∈\left]-∞,𝑎\right])\),
\item
\(ℙ(𝑋>𝑎)\) et \(ℙ(𝑋∈\left]𝑎,-∞\right[)\),
\item
\(ℙ(𝑋⩾𝑎)\) et \(ℙ(𝑋∈\left[𝑎,-∞\right[)\),
\item
\(ℙ(𝑎<𝑋<𝑏)\) et \(ℙ(𝑋∈\left]𝑎,𝑏\right[)\),
\item
\(ℙ(𝑎<𝑋⩽𝑏)\) et \(ℙ(𝑋∈\left]𝑎,𝑏\right])\),
\item
\(ℙ(𝑎⩽𝑋<𝑏)\) et \(ℙ(𝑋∈\left[𝑎,𝑏\right[)\),
\item
\(ℙ(𝑎⩽𝑋⩽𝑏)\) et \(ℙ(𝑋∈\left[𝑎,𝑏\right])\).
\end{itemize}
\end{terminology}
\paragraph{Variables aléatoires sur des espaces dénombrables}
\begin{definition}
Si \(Ω\) et \(Ω'\) sont \mykeyword{dénombrables}, avec événements ou ensembles mesurables par défaut, une
\mykeyword{variable aléatoire} est une application de \(Ω\) dans \(Ω'\).
\end{definition}
\begin{theorem}
L'application qui à tout événement élémentaire \(\{𝑥\}\) de \(Ω'\) associe \(ℙ_Ω\bigl(𝑋^{-1}(𝑥)\bigr)\) définit bien une loi de
probabilités sur  \(Ω'\), elle est notée \(ℙ_{𝑋}\).
\end{theorem}
\begin{proof}
Les préimages par \(𝑋\) des événements élémentaires de  \(Ω'\) forment un système complet
d'événements. À compléter...
\end{proof}
Dans la suite, \(∆\) désigne une \mykeyword{partie dénombrable} de \(ℝ\).
\begin{definition}
[Variable aléatoire discrète]
Une \mykeyword{variable aléatoire discrète} est une application de \(Ω\) dans \(∆\) telle que la préimage
de tout nombre est un événement.
\end{definition}
\begin{remark}
 une variable aléatoire discrète prend ses valeurs dans une partie dénombrable de \(ℝ\),
pas une partie discrète de \(ℝ\).
\end{remark}
Dans les deux contextes précédent ci-dessus, on a le
\begin{theorem}
[Loi d'une variable aléatoire discrète]
L'application qui à tout événement élémentaire \(\{𝑥\}\) de \(∆\) associe \(ℙ_Ω\bigl(𝑋^{-1}(𝑥)\bigr)\) définit bien une loi de
probabilités sur  \(∆\), elle est notée \(ℙ_{𝑋}\).
\end{theorem}
\begin{proof}
On a :
\begin{enumerate}
\item
\vskip\abovedisplayskip\(
0⩽ℙ_{𝑋}(𝑥)=ℙ_Ω\bigl(𝑋^{-1}(𝑥)\b)⩽1
\)
\item
\vskip\belowdisplayskip
\(\bigl(𝑋^{-1}(𝑥)\bigr)_{𝑥∈∆}\) est une famille dénombrable d'événements de \(Ω\) deux à deux disjoints.
En effet, par contraposition :
\end{enumerate}
\begin{align*}
𝑋^{-1}(𝑥)∩𝑋^{-1}(𝑥')≠∅
&⟹
∃𝜔,\;𝜔∈𝑋^{-1}(𝑥)\myand𝜔∈𝑋^{-1}(𝑥')
\\&{}⟹
∃𝜔,\;𝑋(𝜔)=𝑥\myand𝑋(𝜔)=𝑥'
\\&{}⟹
𝑥=𝑥'
\end{align*}
ainsi,
\begin{gather*}
ℙ_Ω\Bigl(\bigcup_{𝑥∈∆}𝑋^{-1}(𝑥)\Bigr)=∑_{𝑥∈∆}ℙ_Ω\bigl(𝑋^{-1}(𝑥)\bigr)=∑_{𝑥∈∆}ℙ_{𝑋}(𝑥)=1
\qedhere
\end{gather*}
\end{proof}
%
Dans la suite, les variables aléatoires discrètes ont leurs valeurs dans \(∆\).
\subsubsection{Opérations algébriques}
\begin{theorem}
Si \(𝑋\) et \(𝑌\) sont deux variables aléatoires discrètes définies sur \(Ω\), \(𝜆\) et \(𝜇\) deux réels.
\begin{itemize}
\item
\(𝜆𝑋+𝜇\),
\(𝑋+𝑌\) et \(𝑋×𝑌\) sont aussi des variables aléatoires discrètes définies sur \(Ω\).
\item
Si \(𝑌\) ne s'annule pas,
\(𝑋/𝑌\) est aussi une variable aléatoire discrète définie sur  \(Ω\).
\end{itemize}
\end{theorem}
\begin{proof}
À préciser...
\end{proof}
\subsubsection[Espérance]{Espérance}
\begin{definition}
[Espérance]
L'\mykeyword{espérance} d'une variable aléatoire discrète \(𝑋\) est
\begin{itemize}
\item
le nombre  \(∑_{𝑥∈𝑋(Ω)}𝑥ℙ(𝑋=𝑥)\) si \(𝑋(Ω)\) est fini, 
\item
la somme de la série \(∑_{𝑖∈ℕ}𝑥_{𝑖}ℙ(𝑋=𝑥_{𝑖})\) si \(𝑋(Ω)=\left\{𝑥_{𝑖},𝑖∈ℕ\right\}\) et
s'il y a convergence absolue
\item
non définie sinon.
\end{itemize}
Elle est notée \(𝔼(𝑋)\) ou \(\overline{𝑋}\).
\mykeyword{Moyenne (arithmétique)} est synonyme d'espérance.
\end{definition}
\begin{remark}
On ne peut pas définir une notion d'espérance si la série est semi-convergente car la
somme dépend alors de l'ordre des termes.
\end{remark}
\begin{theorem}
[Linéarité de l'espérance]
Si \(𝑋\) et \(𝑌\) sont deux variables aléatoires discrètes définies sur \(Ω\), \(𝜆\) et \(𝜇\) deux réels, alors
\(𝔼(𝜆𝑋+𝜇𝑌)=𝜆𝔼(𝑋)+𝜇𝔼(𝑌)\) dès que deux des trois espérances sont bien définies.
\end{theorem}
\begin{proof}
À préciser...
\end{proof}
\begin{theorem}
[Croissance de l'espérance]
Si elle existe, l'espérance d'une variable aléatoire discrète à valeurs positives est positive.
\end{theorem}
\begin{proof}
À préciser...
\end{proof}
\subsubsection{Variance}
\begin{definition}
[Variance, écart type]
La \mykeyword{variance} d'une variable aléatoire discrète \(𝑋\) est \(𝔼\bigl((𝑋-𝔼(𝑋)\bigr)^2)\) si elle existe.
Elle est notée \(𝕍(𝑋)\). L'\mykeyword{écart-type} est la racine carrée de la variance, il est noté
\(σ(𝑋)\).
\end{definition}
\begin{remark}
L'écart-type est traduit en anglais par ``standard deviation'' dont a première lettre correspond à \(𝕍\) en grec.
\end{remark}
\begin{theorem}
[de König]
\begin{equation*}
𝔼\bigl((𝑋-𝔼(𝑋)\bigr)^2=𝔼(𝑋^2)-𝔼(𝑋)^2
\end{equation*}
\end{theorem}
\begin{proof}
À compléter...
\end{proof}
\begin{theorem}
[Homogénéité]
Pour une variable aléatoire discrète \(𝑋\), \(𝜆\) et \(𝜇\) deux nombres réels, on a
\(𝕍(𝜆𝑋+𝜇)=𝜆^2𝕍(𝑋)\) et \(σ(𝜆𝑋+𝜇)=|𝜆|σ(𝑋)\), quand c'est bien défini.
\end{theorem}
\begin{remark}
 on dit que la variance est homogène de degré 2, l'écart-type de degré 1.
\end{remark}
\begin{proof}
À préciser...
\end{proof}
\begin{theorem}
[Normalisation]
Pour une variable aléatoire discrète \(𝑋\) d'espérance \(𝜇\) et d'écart-type \(𝜎\), \(\frac{𝑋-𝜇} 𝜎\) a pour
espérance \(0\) et pour écart-type \(1\).
\end{theorem}
\begin{proof}
À préciser...
\end{proof}
\begin{theorem}
[Sous additivité]
Si \(𝑋\) et \(𝑌\) sont deux variables aléatoires discrètes définies sur \(Ω\), on a
\(σ(𝑋+𝑌)⩽σ(𝑋)+σ(𝑌)\).
\end{theorem}
\begin{proof}
À préciser...
\end{proof}
\subsubsection{Indépendance}
\begin{theorem}
[Loi jointe, lois marginales]
Si \((𝑋,𝑌)\) est un couple de variables aléatoires discrètes définies sur \(Ω\), à valeurs respectivement dans
\(∆_{𝑋}\) et \(∆_{𝑌}\). L'application qui à tout événement élémentaire \(\{(𝑥,𝑦)\}\) de \(∆_{𝑋}×∆_{𝑌}\) associe
\(ℙ_Ω\bigl(𝑋^{-1}(𝑥)∩𝑌^{-1}(𝑦)\bigr)\) définit bien une loi de probabilités sur \(∆_{𝑋}×∆_{𝑌}\). C'est la
\mykeyword{loi jointe} du couple, les lois de \(𝑋\) et de \(𝑌\) en sont les
\mykeyword{lois marginales}.
On note
\begin{gather*}
ℙ_{𝑋,𝑌}(𝐸×𝐹)\overset{\text{déf.}}{=}ℙ(𝑋∈𝐸,𝑌∈𝐹)\overset{\text{déf.}}{=}ℙ_Ω(𝑋^{-1}(𝐸)∩𝑌^{-1}(𝐹)).
\end{gather*}
\end{theorem}
\begin{lemma}
Dans le contexte ci-dessus,
\(\bigl\{𝑋^{-1}(𝑥)∩𝑌^{-1}(𝑦)\mathbin{|}(𝑥,𝑦)∈∆_{𝑋}×∆_{𝑌}\bigr\}\)
forme un système complet d'événements de \(Ω\).
\end{lemma}
\begin{proof}
Pour le lemme, on a
\begin{enumerate}
\item
pour tout issue \(ω\) de \(Ω\),
on a \(ω∈𝑋^{-1}\bigl(X(ω)\bigr)∩𝑌^{-1}\bigl(𝑌(ω)\bigr)\),
\item
si \((𝑥,𝑦)≠(𝑥',𝑦')\), on a \(𝑥≠𝑥'\) ou \(𝑦≠𝑦'\), par conséquent \(𝑋^{-1}(𝑥)∩𝑋^{-1}(𝑥')=∅\)
\ ou \(𝑌^{-1}(𝑦)∩𝑌^{-1}(𝑦')=∅\). En tout cas,
\(𝑋^{-1}(𝑥)∩𝑌^{-1}(𝑦)∩𝑋^{-1}(𝑥')∩𝑌^{-1}(𝑦')= ∅\).
\end{enumerate}
Pour le théorème, il suffit de vérifier que la somme éventuellement infinie des probabilités des événements élémentaires
vaut \(1\). D'après le lemme précédent, on a un système complet d'événement dénombrable, cela suffit.
\end{proof}
\begin{definition}
Deux variables aléatoires discrètes \(𝑋\) et \(𝑌\) définies sur \(Ω\), à valeurs respectivement dans \(∆_{𝑋}\) et
\(∆_{𝑌}\) sont \mykeyword{indépendantes} signifie que pour tout événement élémentaire \(\{(𝑥,𝑦)\}\) de
\(∆_{𝑋}×∆_{𝑌}\) on a
\begin{gather*}
ℙ_{𝑋,𝑌}(\{(𝑥,𝑦)\})=ℙ_{𝑋}(\{𝑥\})×ℙ_{𝑌}(\{𝑦\})
\intertext{ou de manière synonyme}
ℙ(𝑋=𝑥,𝑌=𝑦)=ℙ(𝑋=𝑥)×ℙ(𝑌=𝑦).
\end{gather*}
\end{definition}
%
\begin{lemma}
Si \(𝑋\) et \(𝑌\) sont deux variables aléatoires discrètes indépendantes définies sur \(Ω\), pour toutes paires de
fonctions réelles d'une variable réelle \(𝑓\) et \(𝑔\), on a \(𝔼(𝑓(𝑋)×𝑔(𝑌))=𝔼(𝑓(𝑋))×𝔼(𝑔(𝑌))\), si ces
quantités sont bien définies.
\end{lemma}
\begin{proof}
À compléter...
\end{proof}
\begin{proposition}
Si \(𝑋\) et \(𝑌\) sont deux variables aléatoires discrètes indépendantes définies sur \(Ω\), alors
\(𝔼(𝑋×𝑌)=𝔼(𝑋)×𝔼(𝑌)\) et \(𝕍(𝑋+𝑌)=𝕍(𝑋)+𝕍(𝑌)\), si toutes ces quantités sont bien définies.
\end{proposition}
\begin{proof}
À compléter...
\end{proof}
\begin{terminology}
Indépendance et \mykeyword{corrélation} sont contraires, tout comme indépendant et
\mykeyword{corrélé}.
\end{terminology}
\subsection[Exemples de lois discrètes]{Exemples de lois discrètes}
\subsubsection{Lois uniformes discrètes}
C'est la transposition de l'équiprobabilité.
\subsubsection{Épreuve de Bernoulli}
\begin{definition}
[Épreuve de Bernoulli]
«succès» et «échec» désignant deux objets différents,
une \mykeyword{épreuve de Bernoulli} est une expérience aléatoire modélisée par l'univers \(Ω=\{\text{succès},\text{échec}\}\), ainsi
que \(ℙ\bigl(\{\text{succès}\}\bigr)=𝑝\) et \(ℙ\bigl(\{\text{échec}\bigr\})=1-𝑝\), où \(𝑝\) est un paramètre réel de \([0,1]\).

L'application qui à \(\text{succès}\) associe 1 et à \(\text{échec}\) associe 0 définit une variable aléatoire discrète
particulière.
\end{definition}
\begin{definition}
[Loi de Bernoulli]
Une variable aléatoire \(𝑋\) discrète suit la loi de Bernoulli de paramètre \(𝑝\), réel de \([0,1]\), si \(ℙ(𝑋=1)=𝑝\)
et \(ℙ(𝑋=0)=1-𝑝\).

On note \(𝑋∼𝓑(𝑝)\), qui est lu «\(𝑋\) suit la loi
\(𝓑(𝑝)\)» ou «\(𝑋\) suit la loi de Bernoulli de paramètre \(𝑝\)».
On peut aussi trouver les notations \(𝑋⤳𝓑(𝑝)\) et \(𝑋↪𝓑(𝑝)\).
\end{definition}

\begin{theorem}
[Espérance, variance, écart-type]
Si \(𝑋∼𝓑(𝑝)\), \(𝔼(𝑋)=𝑝\), \(𝕍(𝑋)=𝑝(1-𝑝)\) et \(σ(𝑋)=\sqrt{𝑝(1-𝑝)}\).
\end{theorem}
\subsubsection{Lois géométriques}
\paragraph{Modèle}
Une suite d'épreuves de Bernoulli mutuellement indépendantes : occurrence du premier succès.
\paragraph{Formule}
\begin{definition}
[Loi géométrique]
Une variable aléatoire \(𝑋\) discrète suit la loi de géométrique de paramètre \(𝑝\), réel de \([0,1]\), si
\(ℙ(𝑋=𝑘)=𝑝(1-𝑝)^{𝑘-1}𝟙_ℕ(𝑘)\).
\end{definition}
\paragraph{Propiétés}
\begin{theorem}
[Espérance, variance, écart-type]
Si une variable aléatoire \(𝑋\) discrète suit la loi de géométrique de paramètre \(𝑝\), réel de \(\left]0,1\right]\),
\(𝔼(𝑋)=\frac 1{𝑝}\),  \(𝕍(𝑋)=\frac{1-𝑝}{𝑝^2}\).
\end{theorem}
\begin{proof}
À compléter...
\end{proof}
\subsubsection{Lois binomiales}
\paragraph{Modèle}
Le nombre de succès parmi \(𝑛\) épreuves de Bernoulli de même paramètre \(𝑝\) de \([0,1]\) et deux à deux
indépendantes.
\paragraph{Formule}
\begin{definition}
[Loi binomiale]
Une variable aléatoire \(𝑆\) discrète suit la loi de binomiale de paramètres \(𝑛\) entier naturel et \(𝑝\), réel de
\([0,1]\), si \(ℙ(𝑆=𝑘)=\left(\genfrac{}{}{0pt}{0}{𝑛}{𝑘}\right)𝑝^{𝑘}(1-𝑝)^{𝑛-𝑘}𝟙_ℕ(𝑘)\).

On note \(𝑆∼𝓑(𝑛,𝑝)\), qui est lu «\(𝑆\) suit la loi
\(𝓑(𝑛,𝑝)\)« ou «\(𝑆\) suit la loi binomiale de paramètres \(𝑛\) et \(𝑝\)
\(~\)».
\end{definition}
\begin{remark}
\(𝓑(𝑝)\) et \(𝓑(1,𝑝)\) sont les mêmes lois.
\end{remark}
\paragraph{Propriétés}
\begin{theorem}
Si \(𝑛\) variables aléatoires deux à deux indépendantes suivent la loi \(𝓑(𝑝)\) alors leur somme suit
la loi \(𝓑(𝑛,𝑝)\).
\end{theorem}
\begin{proof}
À compléter...
\end{proof}
\begin{theorem}
[Espérance, variance, écart-type]
\par\noindent
Si \(𝑆∼𝓑(𝑛,𝑝)\), \(𝔼(𝑆)=𝑛𝑝\), \(𝕍(𝑆)=𝑛𝑝(1-𝑝)\) et
\(σ(𝑆)=\sqrt{𝑛𝑝(1-𝑝)}\).
\end{theorem}
\begin{proof}
À compléter de deux manières : avec la linéarité et par le calcul direct
\end{proof}
\subsubsection{Lois de Poisson}
\paragraph{Modèle}
On modélise un nombre d'occurrences par unité lorsque la moyenne est connue.
\paragraph[Formule]{Formule}
\begin{definition}
[Loi de Poisson]
Une variable aléatoire \(𝑋\) discrète suit la loi de Poisson de paramètre \(𝜆\), notée \(𝒫(𝜆)\) avec
\(𝜆\) réel strictement positif, si \(ℙ(𝑋=𝑘)=\frac{𝜆^{𝑘}}{𝑘!}\operatorname{e}^{-𝜆}\;𝟙_ℕ(𝑘)\).

On note \(𝑋∼𝒫(𝜆)\), qui est lu «\(𝑋\) suit la loi
\(𝒫(𝜆)\)« ou «\(𝑋\) suit la loi de Poisson de paramètre \(𝜆\)«.
\end{definition}
\paragraph[Propriétés]{Propriétés}
\begin{theorem}
[Espérance, variance, écart-type]
Si \(𝑋∼𝒫(𝜆)\), \(𝔼(𝑋)=𝜆\), \(𝕍(𝑋)=𝜆\) et \(σ(𝑋)=\sqrt{𝜆}\).
\end{theorem}
\begin{theorem}
Si \(𝑋∼𝒫(𝜆)\) et \(𝑌∼𝒫(𝜇)\), \(𝑋\) et \(𝑌\) étant indépendantes, alors
\(𝑋+𝑌∼𝒫(𝜆+𝜇)\).
\end{theorem}
\begin{proof}
Pour tout entier naturel \(𝑘\), l'événement \(𝑋+𝑌=𝑘\) est la réunion disjointe de tous les événements \((𝑋=𝑖\text{
et }𝑌=𝑘–𝑖)\) lorsque \(𝑖\) varie entre \(0\) et \(𝑘\).
\begin{align*}
ℙ(𝑋+𝑌=𝑘)&{}=∑_{𝑖=0}^{𝑘}ℙ(𝑋=𝑖,𝑌=𝑘-𝑖)
\\&{}=
∑_{𝑖=0}^{𝑘}ℙ(𝑋=𝑖)\;ℙ(𝑌=𝑘-𝑖)
\\&{}=
∑_{𝑖=0}^{𝑘}\frac{λ^{𝑖}}{𝑖!}\operatorname{e}^{-λ}\frac{μ^{𝑘-𝑖}}{(𝑘-𝑖)!}\operatorname{e}^{-μ}
\\&{}=
\frac{(λ+μ)^{𝑘}}{𝑘!}\operatorname{e}^{-(λ+μ)}λ^{𝑖}μ^{𝑘-𝑖}
\end{align*}
On a utilisé l'indépendance pour obtenir la deuxième ligne.
\end{proof}
%
\subsection{Variables aléatoires à densité}
\subsubsection{Définitions}
\begin{definition}
[fonction de densité]
Une \mykeyword{(fonction de) densité} est une fonction réelle de variable réelle, positive, continue par morceaux, croissante et
d'intégrale \(1\).
\end{definition}
\begin{definition}
[variables aléatoires à densité]
Une variable aléatoire \(𝑋\) est à densité \(𝑓\) signifie que  pour tous réels \(𝑎\) et \(𝑏\), on a
\begin{gather*}
\displaystyle
ℙ(𝑎⩽𝑋)=\int _{𝑥=𝑎}^{+∞}𝑓(𝑥)\;𝖽𝑥
\\
ℙ(𝑋⩽𝑏)=\int _{𝑥=-∞}^{𝑏}𝑓(𝑥)\;𝖽𝑥
\end{gather*}
\end{definition}
%
\begin{corollary}
Dans le même contexte,
\begin{gather*}
ℙ(𝑎⩽𝑋⩽𝑏)=∫_{𝑥=𝑎}^{𝑏}𝑓(𝑥)\;𝖽𝑥
\end{gather*}
\end{corollary}
\begin{remark}
\vspace{-\baselineskip}
\begin{itemize}
\item
les intégrales impropres ne sont pas au programme de l'enseignement secondaire. Elles sont définies comme des limites
d'intégrales propres lorsque les bornes tendent vers l'infini de manière indépendante. Il se pose la question de la
convergence...
\item
En fait on a pour tout ensemble mesurable \(𝐸\),
\(ℙ(𝑋∈𝐸)=\int _{𝐸}𝑓(𝑥)\;𝖽𝑥\).
\end{itemize}
\end{remark}
%
\begin{proposition}
Si la variable aléatoire réelle \(𝑋\) est à densité \(𝑓\), pour tous réel \(𝑎\) on a
\(ℙ(𝑋=𝑎)=0\).
\end{proposition}
\begin{proof}
\(ℙ(𝑋=𝑎)=ℙ(𝑎⩽𝑋⩽𝑎)=\int _{𝑥=𝑎}^{𝑎}𝑓(𝑥)\;𝖽𝑥=0\)
\end{proof}
\subsubsection{Loi uniforme continue}
\begin{definition}
[Loi uniforme continue]
Une variable aléatoire réelle \(𝑋\) suit la \mykeyword{loi uniforme} sur l'intervalle
\([𝑎,𝑏]\), avec \(𝑎<𝑏\), signifie qu'elle a pour densité
\begin{equation*}
𝑥⟼\frac 1{𝑏-𝑎}\;𝟙_{[𝑎,𝑏]}(𝑥)
\end{equation*}
\end{definition}
\begin{theorem}
Soit \(𝑋\) une variable aléatoire réelle qui suit la loi uniforme sur l'intervalle \([𝑎,𝑏]\), avec
\(𝑎<𝑏\).
\(𝔼(X)=\frac{𝑎+𝑏} 2\), \(𝕍(X)=\frac{(𝑏-𝑎)^2}{12}\) et
\(σ(X)=\frac{𝑏-𝑎}{2\sqrt 3}\).
\end{theorem}
\begin{proof}
À compléter...
\end{proof}
\subsubsection[Loi exponentielle]{Loi exponentielle}
\begin{definition}
[Loi exponentielle]
Une variable aléatoire réelle \(𝑋\) suit la \mykeyword{loi exponentielle} de paramètre
\(𝜆\) , avec \(𝜆>0\), si elle a pour densité
\begin{equation*}
𝑥⟼𝜆\;\operatorname{e}^{-𝜆𝑥}\;𝟙_{\left[0,+∞\right[}(𝑥)
\end{equation*}
\end{definition}
\begin{theorem}
Soit \(𝑋\) une variable aléatoire réelle qui suit la loi exponentielle de paramètre \(𝜆\), avec
\(𝜆>0\).
\(𝔼(X)=\frac 1{𝜆}\), \(𝕍(X)=\frac 1{𝜆^2}\) et \(σ(X)=\frac 1{𝜆}\).
\end{theorem}
\begin{proof}
À compléter...
\end{proof}
\begin{theorem}
[Vieillissement]
Soit \(𝑋\) une variable aléatoire réelle qui suit une loi exponentielle, on a
\begin{gather*}
∀𝑡,𝑡'∈ℝ+,\;ℙ_{𝑋⩾𝑡}(𝑋⩾𝑡+𝑡')=ℙ(𝑋⩾𝑡').
\end{gather*}
\end{theorem}
\begin{remark}
Si \(𝑋\) modélise une durée de vie, alors la probabilité de vivre encore \(𝑡'\) ne dépend pas de l'âge déjà atteint \emph{id est} du viellissement.
\end{remark}
\begin{proof}
C'est une réécriture de la propriété fonctionnelle de l'exponentielle.

À compléter...
\end{proof}
\begin{remark}
La somme de variables aléatoires indépendantes qui suivent une loi exponentielle ne suit
pas une loi exponentielle mais une loi dite gamma. En revanche, c'est le cas pour leur minimum.
\end{remark}
\subsubsection{Loi normale}
\begin{definition}
[Loi normale]
Une variable aléatoire \(𝑋\) suit la \mykeyword{loi normale} de paramètres \(𝜇\) réel et \(𝜎\) réel
strictement positif signifie qu'elle a pour densité
\begin{equation*}
𝑥⟼\frac 1{\sqrt{2π𝜎^2}}\operatorname{e}^{\frac{-(𝑥-𝜇)^2}{2𝜎^2}}
\end{equation*}
On note \(𝑋∼𝒩(𝜇,𝜎)\), qui est lu «\(𝑋\) suit la loi
\(𝒩(𝜇,𝜎)\)» ou «\(𝑋\) suit la loi normale de paramètres \(𝜇\) et \(𝜎\)«. On peut
trouver la notation \(𝑋∼𝒩(𝜇,𝜎^2)\).
\end{definition}
\begin{theorem}
[Espérance et variance de la loi normale]
Si \(𝑋∼𝒩(𝜇,𝜎)\) alors \(𝔼(𝑋)=𝜇\) et \(𝕍(𝑋)=𝜎^2\).
\end{theorem}
\begin{proof}
À compléter...
\end{proof}
En particulier, pour \(𝜇=0\) et \(𝜎=1\) on a la
\begin{definition}
[Loi normale centrée réduite]
Une variable aléatoire \(𝑋\) suit la \mykeyword{loi normale centrée réduite}, signifie qu'elle a pour densité
\begin{equation*}
𝑥⟼\frac 1{\sqrt{2π}}\operatorname{e}^{\frac{-𝑥^2} 2}
\end{equation*}
On note \(𝑋∼𝒩(0,1)\), qui est lu comme ci-dessus ou «\(𝑋\) suit la loi
normale centrée réduite».
\end{definition}
\begin{theorem}
[Valeurs remarquables]
Pour tout \(𝛼\) de \(\left[0,1\right[\), il existe un unique réel \(𝑢_{𝛼}\) tel que
\begin{equation*}
\frac 1{\sqrt{2π}}\int _{𝑥=-𝑢_{𝛼}}^{𝑢_{𝛼}}\operatorname{e}^{\frac{-𝑥^2} 2}\;𝖽𝑥=𝛼
\end{equation*}
En particulier :
\tablefirsthead{\hline
\centering \(𝛼\) &
\centering\arraybslash \(𝑢_{𝛼}\)
\\
}
\tablehead{\hline
\centering \(𝛼\) &
\centering\arraybslash \(𝑢_{𝛼}\)
\\
}
\tabletail{}
\tablelasttail{}
\begin{supertabular}
{|m{2cm}|m{2cm}|}
\hline
\centering \(⩾99~\%\) &
\centering\arraybslash \(2,58\)
\\
\hline
\centering \(⩾95~\%\) &
\centering\arraybslash \(1,96\)
\\
\hline
\centering \(⩾90~\%\) &
\centering\arraybslash \(1,65\)
\\
\hline
\end{supertabular}
\end{theorem}
\begin{proof}
Dans une feuille de calcul, mettre en colonne A les nombres de \(1,64\) à
\(2,58\) avec un pas de \(0,01\). Mettre en B1 la formule
\(=\mathtt{1-2\ast \mathit{NORM.S.DIST}(\mathit{A1};1)}\) puis étendre vers le bas. À compléter...
\end{proof}
\begin{definition}
[Plages de normalité]
Si \(𝑋∼𝒩(𝜇,𝜎)\) alors
\begin{equation*}
ℙ(𝑋∈\left[𝜇-𝜎,𝜇+𝜎\right])≅68~\text{\%}
\end{equation*}
\begin{equation*}
ℙ(𝑋∈[𝜇-2𝜎,𝜇+2𝜎])≅95\text{\%}
\end{equation*}
\begin{equation*}
ℙ(𝑋∈[𝜇-3𝜎,𝜇+3𝜎])≅99~\text{\%}
\end{equation*}
Ces intervalles sont les \mykeyword{plages de normalité} à niveau de confiance de 68~\%,
95~\%, 99,7~\%.
\end{definition}
\subsection[Théorèmes limites]{Théorèmes limites}
\subsubsection{Loi faible des grands nombres}
\begin{theorem}
[Inégalité de Bienaymé-Tchebychev]
Soit \(𝑋\) une variable aléatoire réelle avec espérance et variance.
\begin{equation*}
∀𝜀∈\left]0,+∞\right[,\;ℙ\bigl(|𝑋-𝔼(𝑋)|⩾𝜀\bigr)⩽\frac{𝕍(𝑋)}{𝜀^2}
\end{equation*}
\end{theorem}
\begin{proof}
À compléter...
\end{proof}
\begin{theorem}
[Loi faible des grands nombres]
Soit \((𝑋_{𝑛})_{𝑛∈ℕ^{ *}}\) une suite de variables aléatoires réelles définies sur le même espace probabilisé,
deux à deux indépendantes, suivant la même loi d'espérance \(𝜇\) et d'écart-type \(𝜎\).
\begin{equation*}
∀𝜀∈\left]0,+∞\right[,\;ℙ\left(\left|\frac{𝑋_1+𝑋_2+...+𝑋_𝑛}{𝑛}-𝜇\right|⩾𝜀\right)\mathop{⟶}_{𝑛→∞}0
\end{equation*}
\end{theorem}
\begin{proof}
On applique l'inégalité de Bienaymé-Tchebychev à la moyenne
\(\frac{𝑋_1+𝑋_2+...+𝑋_𝑛}{𝑛}\).
Sachant que
\begin{gather*}
𝔼\left(\frac{𝑋_1+𝑋_2+...+𝑋_𝑛}{𝑛}\right)=𝜇\text{ et }𝕍\left(\frac{𝑋_1+𝑋_2+...+𝑋_𝑛}{𝑛}\right)=\frac{𝜎^2}
n
\intertext{on obtient :}
ℙ\left(\left|\frac{𝑋_1+𝑋_2+...+𝑋_𝑛}{𝑛}-𝜇\right|⩾𝜀\right)⩽\frac{𝜎^2}{𝑛𝜀^2}
\end{gather*}
d'où le résultat.
\end{proof}
\begin{corollary}
Dans le même contexte,
\begin{equation*}
∀𝜀∈\left]0,+∞\right[,\;ℙ\left(-𝜀<\frac{𝑋_1+𝑋_2+...+𝑋_𝑛}{𝑛}-𝜇<𝜀\right)\mathop{⟶}_{𝑛→∞}1
\end{equation*}
\end{corollary}
\begin{proof}
Il suffit de passer au complémentaire.
\begin{gather*}
\textstyle
ℙ\left(-𝜀<\frac{𝑋_1+𝑋_2+...+𝑋_𝑛}{𝑛}-𝜇<𝜀\right)=1-ℙ\left(\left|\frac{𝑋_1+𝑋_2+...+𝑋_𝑛}{𝑛}-𝜇\right|⩾𝜀\right)
\qedhere
\end{gather*}
\end{proof}
\begin{remark}
La loi faible des grands nombres est précisée par le théorème central limite ci-dessous.
\end{remark}

\subsubsection{Théorème central limite}
Hors programme mais pas de beaucoup, en tout cas absolument indispensable à la compréhension.
\begin{theorem}
[Central limite]
Soit \((𝑋_{𝑛})_{𝑛∈ℕ^{ *}}\) une suite de variables aléatoires réelles définies sur le même espace probabilisé,
deux à deux indépendantes, suivant la même loi d'espérance \(𝜇\) et d'écart-type \(𝜎\) non nul. Pour tout nombre
\(𝑡\),
\begin{equation*}
ℙ\left(\frac{𝑋_1+𝑋_2+...+𝑋_𝑛-𝑛𝜇}{𝜎\sqrt{𝑛}}⩽t\right)\mathop{⟶}_{𝑛→∞}\frac 1{\sqrt{2π}}\int
_{-∞}^t\operatorname{e}^{\frac{-x^2} 2}\mathit{dx}
\end{equation*}
\end{theorem}
\begin{remark}
C'est un théorème très puissant car on ne connaît de la loi commune à toutes les variables aléatoires que la moyenne et
l'écart type, et rien d'autre.
\end{remark}
\begin{proof}
Hors programme.
\end{proof}
\begin{theorem}
[Théorème de Moivre-Laplace]
Soit \((𝑆_{𝑛})_{𝑛∈ℕ^{ *}}\) une suite de variables aléatoires réelles telles que \(𝑆_{𝑛}∼ℬ(𝑛,𝑝)\),
avec \(𝑝\) dans \(\left]0,1\right[\).
Pour tous réels \(𝑎\) et \(𝑏\) tels que \(𝑎⩽𝑏\), on a
\begin{equation*}
ℙ\left(𝑎⩽\frac{𝑆_𝑛-𝑛𝑝}{\sqrt{𝑛𝑝(1-𝑝)}}⩽𝑏\right)\mathop{⟶}_{𝑛→∞}\frac 1{\sqrt{2π}}\int
_{𝑎}^{𝑏}\operatorname{e}^{\frac{-x^2} 2}\mathit{dx}
\end{equation*}
\end{theorem}
\begin{proof}
Hors programme.
\end{proof}
\begin{remark}
On retrouve le théorème central limite si
\(𝑆_{𝑛}=𝑋_1+𝑋_2+...+𝑋_{𝑛}\) et si les \(𝑋_{𝑛}\) suivent une loi de Bernoulli de paramètre \(𝑝\). Au passage :
peut-on toujours obtenir une telle décomposition de \(𝑆_{𝑛}\) ? À préciser...
\end{remark}
\subsubsection[Autre convergence]{Autre convergence}
\begin{theorem}
Si pour tout entier naturel \(𝑛\) on a \(𝑆_{𝑛}∼ℬ(𝑛,𝑝_{𝑛})\), avec
\(𝑛\;𝑝_{𝑛}\mathop{⟶}_{n→+∞}𝜆\) alors
\begin{equation*}
ℙ\left(𝑆_{𝑛}=𝑘\right)\mathop{⟶}_{𝑛→∞}\frac{𝜆^{𝑘}}{𝑘!}\;e^{-𝜆}
\end{equation*}
\end{theorem}
\begin{proof}
À compléter...
\end{proof}
\subsection{Fluctuation. Estimation.}
La fluctuation : on connaît la loi de l'échantillon, on connaît la fréquence, on mesure la cohérence des deux.

L'estimation : on connaît le type de loi de l'échantillon, on connaît la fréquence, on estime la moyenne.
\subsubsection[Intervalle de fluctuation en 2de]{Intervalle de fluctuation en 2\textsuperscript{de}}
\paragraph{Présentation}
Le programme de 2\textsuperscript{de} suggère de présenter aux élèves le résultat suivant : pour des échantillons de
taille \(𝑛\) supérieure ou égale à \(25\) et des proportions \(𝑝\) du caractère comprises entre
\(0,2\) et \(0,8\) : si \(𝑓\) désigne la fréquence du caractère dans l’échantillon,
\(𝑓\) appartient à l’intervalle \(\left[𝑝-1/\sqrt{𝑛},𝑝+1/\sqrt{𝑛}\right]\) avec une probabilité d’au moins
\(0,95\).
\paragraph{Justification asymptotique}
Pour \(0,2⩽𝑝⩽0,8\), on obtient \(0,4⩽\sqrt{𝑝(1-𝑝)}⩽0,5\), d'où
\begin{gather*}
-\frac{1,96\sqrt{𝑝(1-𝑝)}}{\sqrt{𝑛}}⩽\frac{𝑆_{𝑛}}{𝑛}-𝑝⩽\frac{1,96\sqrt{𝑝(1-𝑝)}}{\sqrt{𝑛}}
⟹
-\frac1{\sqrt{𝑛}}⩽\frac{𝑆_{𝑛}}{𝑛}-𝑝⩽\frac 1{\sqrt{𝑛}}.
\end{gather*}
Sous réserve que la limite existe, on retrouve bien l'estimation de la classe de seconde
\begin{equation*}
\lim _{n→+∞}ℙ\left(-\frac 1{\sqrt n}⩽\frac{𝑆_{𝑛}}{𝑛}-𝑝⩽\frac 1{\sqrt n}\right)⩾95\text{\%}
\end{equation*}
\begin{remark}
Dans ce programme, il est conseillé de s'assurer que \(𝑛⩾25\) et \(0,2⩽𝑝⩽0,8\). En
classe terminale, on prendra plutôt \(𝑛⩾30\), \(𝑛𝑝⩾5\) et \(𝑛(1-𝑝)⩾5\). Mais  à la limite !
\end{remark}
\paragraph[Justification]{Justification}
On sait que \(𝑛𝑓\) suit la loi \(
(𝑛,𝑝)\), ce qui nous permet d'avoir une idée beaucoup plus précise du
seuil de tolérance. Dans le tableau suivant, on a indiqué pour différentes valeurs de \(𝑛\) pour les lignes et \(𝑝\) pour les colonnes la probabilité
d'avoir \(𝑓\) dans l'intervalle indiqué ci-dessus, avec \(4\) chiffres après la virgule.
\begin{center}
\begin{tabular}{|*{7}{>{\(}c<{\)}|}}
\hline
𝑛∖𝑝 & 0,2 & 0,25 & 0,3 & 0,35 & 0,4 & 0,45
\\\hline
25 & 0,9944 & 0,9822 & 0,9736 & 0,9649 & 0,9774 & 0,9569
\\\hline
26 & 0,9891 & 0,9791 & 0,9678 & 0,9608 & 0,9569 & 0,9526
\\\hline
27 & 0,9866 & 0,9741 & 0,9807 & 0,9589 & 0,9508 & 0,9682
\\\hline
28 & 0,9832 & 0,9855 & 0,9636 & 0,9728 & 0,9673 & 0,9443
\\\hline
29 & 0,9915 & 0,9817 & 0,9763 & 0,9692 & 0,9438 & 0,9618
\\\hline
30 & 0,9893 & 0,9678 & 0,9737 & 0,9467 & 0,9616 & 0,9354
\\\hline
40 & 0,9906 & 0,9837 & 0,9766 & 0,9703 & 0,9655 & 0,9625
\\\hline
50 & 0,9925 & 0,9790 & 0,9805 & 0,9633 & 0,9707 & 0,9545
\\\hline
100 & 0,9916 & 0,9852 & 0,9786 & 0,9729 & 0,9685 & 0,9657
\\\hline
1000 & 0,9873 & 0,9786 & 0,9703 & 0,9633 & 0,9580 & 0,9548
\\\hline
10000 & 0,9880 & 0,9797 & 0,9717 & 0,9649 & 0,9598 & 0,9566
\\\hline
\end{tabular}
\end{center}
On a utilisé la
symétrie \(𝑝⟷1-𝑝\) pour restreindre \(𝑝\) entre \(0,2\) et \(0,5\), ainsi que la formule
suivante
\begin{align*}
=&\text{BINOM.DIST}(\text{FLOOR}(𝑛*𝑝+\text{SQRT}(𝑛));𝑛;𝑝;1)-...
\\
&...-\text{BINOM.DIST}(\text{CEILING}(𝑛*𝑝-\text{SQRT}(𝑛))-1;𝑛;𝑝;1)
\end{align*}
qui vient du
\begin{lemma}
\begin{equation*}
ℙ\left(-\frac 1{\sqrt{𝑛}}⩽𝑓-𝑝⩽\frac 1{\sqrt{𝑛}}\right)=ℙ(𝑛𝑓⩽⌊𝑛𝑝+\sqrt{𝑛}⌋)-ℙ(𝑛𝑓<⌈𝑛𝑝-\sqrt{𝑛}⌉)
\end{equation*}
\end{lemma}
\begin{proof}
On a
\begin{gather*}
-\frac 1{\sqrt{𝑛}}⩽𝑓-𝑝⩽\frac 1{\sqrt{𝑛}}⟺𝑛𝑝-\sqrt{𝑛}⩽𝑛𝑓⩽𝑛𝑝+\sqrt{𝑛}
\intertext{donc}
ℙ\left(-\frac 1{\sqrt{𝑛}}⩽𝑓-𝑝⩽\frac 1{\sqrt{𝑛}}\right)=ℙ(𝑛𝑓⩽𝑛𝑝+\sqrt{𝑛})-ℙ(𝑛𝑓<𝑛𝑝-\sqrt{𝑛}).
\intertext{Comme \(𝑛𝑓\) est un entier,}
𝑛𝑓⩽𝑛𝑝+\sqrt{𝑛}⟺𝑛𝑓⩽⌊𝑛𝑝+\sqrt{𝑛}⌋,\\
𝑛𝑓<𝑛𝑝-\sqrt{𝑛}⟺𝑛𝑓<⌈𝑛𝑝+\sqrt{𝑛}⌉.
\qedhere
\end{gather*}
\end{proof}
\begin{remark}
 si on fait une étude avec plus de données, on observe que pour avoir effectivement un
seuil plus grand que \(0,95\), il faut prendre \(𝑛⩾49\). Ce n'est pas du tout
\(𝑛⩾25\)\(~\)! C'est ce qui fait la différence entre un intervalle de fluctuation et un intervalle de
fluctuation asymptotique.
\end{remark}
\subsubsection{Intervalle de fluctuation en terminale.}
\begin{theorem}
Si la variable aléatoire \(𝑆_{𝑛}\) suit la loi \(ℬ(𝑛,𝑝)\), alors, pour tout \(𝛼\) dans
\(\left[0,1\right[\) on a,
\begin{gather*}
ℙ\left(-𝑢_{𝛼}⩽\frac{𝑆_{𝑛}-𝑛𝑝}{\sqrt{𝑛𝑝(1-𝑝)}}⩽𝑢_{𝛼}\right)\mathop{⟶}_{𝑛→∞}𝛼
\intertext{ou de manière équivalente}
ℙ\left(\frac{𝑆_{𝑛}}{𝑛}∈𝐼_{𝑛}^{𝛼}(𝑝)\right)
\mathop{⟶}_{𝑛→∞}
𝛼,\text{ avec }
𝐼_{𝑛}^{𝛼}(𝑝)\mybydef{=}𝑝+𝑢_{𝛼}\sqrt{\frac{𝑝(1-𝑝)}{𝑛}}[-1,1]
\end{gather*}
\(𝑢_{𝛼}\) étant la valeur remarquable associée à \(𝛼\).
\end{theorem}
\begin{proof}
Application du théorème de Moivre-Laplace. À compléter...
\end{proof}
Pour appliquer ce théorème, on a la
\begin{method}
Soit \((𝑋_{𝑛})_{𝑛∈ℕ^{ *}}\) une suite de variables aléatoires réelles deux à deux indépendantes qui suivent la
même loi de Bernoulli. Soit \(𝑓_{𝑛}\) une réalisation de \(\frac{𝑋_1+𝑋_2+...+𝑋_{𝑛}}{𝑛}\).
\begin{itemize}
\item
Si \(𝑓_{𝑛}∈𝐼_{𝑛}^{𝛼}(𝑝)\), on décidera que le paramètre de la loi de Bernoulli est \(𝑝\) au seuil asymptotique
\(1-𝛼\). Le risque asymptotique d'erreur dépend du paramètre réel de la loi.
\item
Si \(𝑓_{𝑛}∉𝐼_{𝑛}^{𝛼}(𝑝)\), on décidera que le paramètre de la loi de Bernoulli n'est pas \(𝑝\) au seuil
asymptotique \(1-𝛼\). Le risque asymptotique d'erreur est de \(𝛼\).
\end{itemize}
\end{method}
\subsubsection[Estimation]{Estimation}
Avec les notations précédentes, on observe que d'une certain manière,
\begin{equation*}
𝑝+𝑢_{𝛼}\sqrt{\frac{\frac{𝑆_{𝑛}}{𝑛}\left(1-\frac{𝑆_{𝑛}}{𝑛}\right)}{𝑛}}[-1,1]\mathop{⟶}_{𝑛→∞}𝑝+𝑢_{𝛼}\sqrt{\frac{𝑝(1-𝑝)}{𝑛}}[-1,1]
\end{equation*}
comme par ailleurs
\begin{equation*}
\frac{𝑆_{𝑛}}{𝑛}∈𝑝+𝑢_{𝛼}\sqrt{\frac{\frac{𝑆_{𝑛}}{𝑛}\left(1-\frac{𝑆_{𝑛}}{𝑛}\right)}{𝑛}}[-1,1]⟺𝑝∈\frac{𝑆_{𝑛}}{𝑛}+𝑢_{𝛼}\sqrt{\frac{\frac{𝑆_{𝑛}}{𝑛}\left(1-\frac{𝑆_{𝑛}}{𝑛}\right)}{𝑛}}[-1,1]
\end{equation*}
on en déduit d'une certaine manière le
\begin{theorem}
[Estimation de paramètre]
\begin{gather*}
ℙ\left(𝑝∈𝐽_{𝑛}^{𝛼}\right)\mathop{⟶}_{𝑛→∞}𝛼%
,\text{ avec }
𝐽_{𝑛}^{𝛼}\mybydef{=}\frac{𝑆_{𝑛}}{𝑛}+𝑢_{𝛼}\sqrt{\frac{\frac{𝑆_{𝑛}}{𝑛}\left(1-\frac{𝑆_{𝑛}}{𝑛}\right)}{𝑛}}[-1,1]
\end{gather*}
\(𝑢_{𝛼}\) étant la valeur remarquable associée à \(𝛼\).
\end{theorem}
\begin{proof}
Hors programme...
\end{proof}
