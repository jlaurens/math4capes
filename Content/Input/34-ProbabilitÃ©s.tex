% !TEX encoding = UTF-8
% !TEX program = xelatex
% !TEX root =../Main/main.tex

\section{Probabilités}
Espaces probabilisés finis. Probabilités conditionnelles, conditionnement et indépendance.
\subsection{Prérequis, rappels et précautions}
On suppose connus les notions élémentaires de la théorie des ensembles ainsi que le vocabulaire afférent. En
particulier, on utilisera les notions de réunion et d'intersection de familles d'ensembles rappelées au chapitre II.
\subsection{Généralités.}
Le cas des variables aléatoires qui suivent la loi de Poisson ne rentre pas dans le cadre de probabilités finies car
l'ensemble des événements est infini : il faut étendre le programme au cas dénombrable. De là au cas général, il n'y a
souvent qu'un pas. On donne les définitions essentielles en faisant apparaître, quand c'est nécessaire, les trois
situations :
\begin{enumerate}
\item cas fini,
\item cas dénombrable,
\item cas général.
\end{enumerate}
Ce sont les définitions mathématiques, pas les définitions pragmatiques utilisées dans l'enseignement secondaire.
\subsubsection{Univers}
\begin{terminology}
En théorie des probabilités, \mykeyword{univers} est synonyme de \mykeyword{ensemble},
\mykeyword{issue} est synonyme de \mykeyword{élément}. L'univers est noté en général \(Ω\).
Dans ce contexte, un autre synonyme moins fréquent de issue est éventualité.
\end{terminology}
Dans toute la suite \(Ω\) désigne un univers fini, dénombrable ou général selon le cas.
\subsubsection{Événements}
\begin{definition}
[Événements, Tribu, cas général]
Une \mykeyword{tribu sur }\(𝛀\) est un ensemble non vide de parties de \(Ω\) stable par
passage au complémentaire et par union dénombrable. Elle est notée en général \(𝒯\). Les éléments de
la tribu sont appelés \mykeyword{événements}. \((Ω,𝒯)\) est un \textbf{espace
probabilisable}.
\end{definition}
\begin{remark}
Si \ \(Ω\) est dénombrable, \(𝒫(Ω)\) est la tribu choisie par défaut, ce
qui permet une définition simplifiée des événements. Ce choix recouvre en réalité toutes les possibilités.
\end{remark}
\begin{definition}
[Événements, cas fini ou dénombrable]
Les parties de \(Ω\) sont appelés \mykeyword{événements}.
\end{definition}
\begin{remark}
\vspace*{-\baselineskip}
\begin{itemize}
\item
Dans les deux cas, un événement est un ensemble d'issues.
\item
Il ne faut pas confondre l'événement avec sa description. Par exemple, «~tirer un nombre multiple
de 3~» n'a pas du tout la même signification selon que le dé a quatre, six ou huit faces. D'ailleurs, la probabilité
n'est pas la même.
\end{itemize}
\end{remark}
\begin{definition}
Le \mykeyword{contraire} d'un événement \(𝐸\) est le complémentaire \(Ω∖ 𝐸\) de \(𝐸\) dans
\(Ω\). Il est noté \(\overline{𝐸}\), qui peut être lu « \(𝐸\) barre~». Deux événements sont
\mykeyword{contraires} si l'un est le contraire de l'autre.
\end{definition}
\begin{remark}
contraire et complémentaire sont deux termes synonymes en mathématique mais pas en
français.
\end{remark}
Dans le cas fini, \(∅\) et \(Ω\) sont des événements par définition, mais dans le cas
général, on a la
\begin{proposition}
\(∅\) et \(Ω\) sont des événements.
Cela équivaut à dire que toute tribu contient \(∅\) et \(Ω\).
\end{proposition}
\begin{proof}
Par définition d'une tribu. Il existe au moins un événement \(𝐸\). Le contraire \(\overline{𝐸}\),
\(𝐸∩\overline{𝐸}\) et \(𝐸∪\overline{𝐸}\) sont aussi des événements. Or
\(𝐸∩\overline{𝐸}=∅\) et \(𝐸∪\overline{𝐸}=Ω\).
\end{proof}
\begin{theorem}
Toute tribu est stable par intersection dénombrable.
\end{theorem}
\begin{proof}
C'est immédiat dans le cas fini ou dénombrable. On sait que l'intersection est le complémentaire de la réunion des
complémentaires : pour toute famille dénombrable d'événements \((𝐸_{𝑖})_{𝑖∈𝐼}\), \((\overline{𝐸}_{𝑖})_{𝑖∈𝐼}\) est
aussi une famille dénombrable d'événements, la réunion est un événement ainsi que son contraire.
\end{proof}
\begin{definition}
[Issues possibles]
Les éléments d'un événement sont les \mykeyword{issues possibles} de l'événement. Si l'événement n'est pas
spécifié, il s'agit de \(Ω\).
\end{definition}
\begin{definition}
[Événement incompatibles]
Deux événements sont \mykeyword{incompatibles} signifie que leur intersection est vide. Un événement est
\mykeyword{incompatible} avec un autre s'ils sont incompatibles.
\end{definition}
\begin{remark}
Tout événement est incompatible avec son contraire.
\end{remark}
\begin{terminology}
Incompatible est synonyme de \mykeyword{disjoint}, qui appartient plutôt au vocabulaire ensembliste.
\end{terminology}
\begin{theorem}
L'événement \(∅\) est incompatible avec tout événement. Il est le seul incompatible avec lui-même.
\end{theorem}
\begin{proof}
Pour tout événement \(𝐸\), on a \(\mathsf ∅∩𝐸= ∅\) et si \(𝐸∩𝐸=∅\) alors
\(𝐸=∅\). puisque \(𝐸∩𝐸=𝐸\).
\end{proof}
\begin{theorem}
Soit \(𝐸\), \(𝐹\) deux événements, on a
\begin{equation*}
𝐸⊂𝐹⟺𝐸∩\overline{𝐹}=∅
\end{equation*}
\emph{id est} \(𝐸\) est inclus dans \(𝐹\) si et seulement si \(𝐸\) et
\(\overline{𝐹}\) sont incompatibles.
\end{theorem}
%
\begin{theorem}
Soit \(𝐸\), \(𝐹\) deux événements, on a
\begin{equation*}
(𝐸∩𝐹)∩(𝐸∩\overline{𝐹})=∅\text{ et }𝐸=(𝐸∩𝐹)∪(𝐸∩\overline{𝐹}).
\end{equation*}
\(\bigl\{𝐸∩𝐹; 𝐸∩\overline{𝐹}\bigr\}\) est une partition de \(𝐸\).
\end{theorem}
\begin{proof}
Par associativité de l'intersection, on a
\((𝐸∩𝐹)∩(𝐸∩\overline{𝐹})=𝐸∩(𝐹∩\overline{𝐹})=𝐸∩∅=
∅\).
Par distributivité, on a
\((𝐸∩𝐹)∪(𝐸∩\overline{𝐹})=𝐸∩(𝐹∪\overline{𝐹})=𝐸∩Ω=𝐸\)
\end{proof}
\begin{definition}
[et, ou]
Soit \(𝐸\), \(𝐹\) deux événements, l'événement \(𝐸∩𝐹\) est noté \(𝑬\myand𝑭\), l'événement \(𝐸∪𝐹\) est noté \(𝑬\myor𝑭\).
\subsubsection[Probabilité]{Probabilité}
\end{definition}
\begin{definition}
[Probabilité]
Une \mykeyword{loi de probabilité} est une application \(ℙ\)
\begin{enumerate}
\item qui à tout événement associe un nombre dans \([0;1]\).
\item telle que \(ℙ(Ω)=1\)
\item et
\begin{enumerate}
\item cas fini : si deux événements \(𝐸\) et \(𝐹\) sont incompatibles, on a \(ℙ(𝐸∪𝐹)=ℙ(𝐸)+ℙ(𝐹)\),
\item autres cas : pour toute famille dénombrable \((𝐸_{𝑖})_{𝑖∈𝐼}\) d'événements deux à deux incompatibles
\(ℙ\left(\bigcup_{𝑖∈𝐼}𝐸_{𝑖}\right)=∑_{𝑖∈𝐼}ℙ\left(𝐸_{𝑖}\right)\)
\end{enumerate}
\end{enumerate}
Selon le cas, \((Ω,ℙ)\) ou \((Ω,𝒯,ℙ)\) est appelé \mykeyword{espace probabilisé}.
\\
Loi de probabilité, \mykeyword{mesure de probabilité}, et \mykeyword{probabilité} sont
synonymes l'un de l'autre.
\end{definition}
\begin{remark}
Pour information, la propriété 3-b ci-dessus est la σ-additivité de ℙ.
\end{remark}
Dans la suite de la section, \((Ω,ℙ)\) ou \((Ω,𝒯,ℙ)\) est un espace probabilisé, selon le cas.
%
\begin{theorem}
[Probabilité du contraire]
Pour tout événement \(𝐸\) \textit{, } \(ℙ\left(\overline{𝐸}\right)=1-ℙ(𝐸)\).
\end{theorem}
\begin{proof}
Pour tout événement \(𝐸\), on a \(𝐸∩\overline{𝐸}=∅\) et \(𝐸∪\overline{𝐸}=Ω\) donc
\(1=ℙ(Ω)=ℙ\bigl(𝐸∪\overline{𝐸}\bigr)=ℙ(𝐸)+ℙ\bigl(\overline{𝐸}\bigr)\).
\end{proof}
\begin{theorem}
On a \(ℙ(∅)=0\).
\end{theorem}
\begin{proof}
On applique le théorème précédent au cas particulier \(𝐸=Ω\). D'une autre façon, on a \(∅∩∅=∅\) et \(∅∪∅=∅\) donc \(ℙ(∅)=ℙ(∅∪∅)=ℙ(∅)+ℙ(∅)\) et par régularité
\(0=ℙ(∅)\).
\end{proof}
\begin{theorem}
[de l'intersection]
Pour des événements \(𝐸\) et \(𝐹\) on a
\begin{equation*}
ℙ(𝐸∩𝐹)+ℙ(𝐸∪𝐹)=ℙ(𝐸)+ℙ(𝐹)
\end{equation*}
\end{theorem}
\begin{proof}
On a
\((𝐸∩𝐹)∪(\overline{𝐸}∩𝐹)=(𝐸∪\overline{𝐸})∩𝐹=Ω∩𝐹=𝐹\) et
\begin{equation*}
(𝐸∩𝐹)∩(\overline{𝐸}∩𝐹)=(𝐸∩\overline{𝐸})∩𝐹=∅∩𝐹=∅
\end{equation*}
donc
\(ℙ(\overline E∩𝐹)=ℙ(𝐹)-ℙ(𝐸∩𝐹)\).
De plus
\begin{gather*}
𝐸∪(\overline{𝐸}∩𝐹)=(𝐸∪\overline{𝐸})∩(𝐸∪𝐹)=Ω∩(𝐸∪𝐹)=𝐸∪𝐹
\intertext{et}
𝐸∩(\overline{𝐸}∩𝐹)=(𝐸∩\overline{𝐸})∩𝐹=∅∩𝐹=∅
\end{gather*}
donc
\(ℙ(𝐸∪𝐹)=ℙ(𝐸)+ℙ(\overline{𝐸}∩𝐹)=ℙ(𝐸)+ℙ(𝐹)-ℙ(E∩𝐹)\).
\end{proof}
\begin{remark}
\( \)on pouvait traiter \(𝐸\) et \(𝐹\) de manière symétrique mais c'est un
peu plus long.
\end{remark}
\begin{definition}
[Événements certains, possible, impossible]
Un événement est \mykeyword{certain} quand sa probabilité vaut \(1\). Un événement est
\mykeyword{impossible} quand sa probabilité vaut \(0\), il est
\mykeyword{possible} sinon.
\end{definition}
\begin{attention}
\vspace{-\baselineskip}
\begin{itemize}
\item
L'univers des possibles est un événement certain, mais selon cette terminologie, ce n'est pas le seul.
\item
Dans le langage courant, improbable peut correspondre à une probabilité faible, mais pas nulle : voir «hautement
improbable».
\end{itemize}
\end{attention}
%
\begin{theorem}
[Croissance]
Soit \(𝐸\), \(𝐹\) deux événements, on a
\(𝐸⊂𝐹⟹ℙ(𝐸)≤ℙ(𝐹)\).
\end{theorem}
\begin{proof}
On a \(𝐹=𝐸∪(𝐹∖𝐸)\) et \(𝐸∩(𝐹∖ 𝐸)=∅\) donc \(ℙ(𝐹)=ℙ(𝐸)+ℙ(𝐹∖
𝐸)≥ℙ(𝐸)\).
\end{proof}
%
\begin{theorem}
[Continuité]
Si \((𝐸_{𝑛})_{𝑛∈ℕ}\) est une suite d'événements mutuellement incompatibles,
\(ℙ\left(\bigcup_{𝑖=0}^{𝑛}𝐸_{𝑖}\right)\bigcup_{𝑛→∞}ℙ\left(\operatornamewithlimits{?}_{𝑖=0}^∞𝐸_{𝑖}\right)
\).
\end{theorem}
%
\begin{remark}
Ce théorème n'a d'intérêt que pour un univers infini.
\end{remark}
\begin{proof}
C'est une conséquence immédiate de
\begin{gather*}
ℙ\Bigl(\bigcup_{𝑖=0}^{𝑛}𝐸_{𝑖}\Bigr)=\sum_{𝑖=0}^{𝑛}ℙ(𝐸_{𝑖}),\
ℙ\Bigl(\bigcup_{𝑖=0}^∞𝐸_{𝑖}\Bigr)=\sum _{𝑖=0}^∞ℙ(𝐸_{𝑖})
\intertext{et}
\sum
_{𝑖=0}^{𝑛}ℙ(𝐸_{𝑖})\mathop{⟶}_{𝑛→∞}\sum_{𝑖=0}^∞ℙ(𝐸_{𝑖}).\qedhere
\end{gather*}
\end{proof}
%
%
\begin{theorem}
[Continuité monotone]
\par\noindent
\begin{enumerate}
\item
Si \((𝐸_{𝑛})_{𝑛∈ℕ}\) est une suite croissante d'événements,
\begin{equation*}
ℙ(𝐸_{𝑛})\mathop{⟶}_{𝑛→∞}ℙ\Bigl(\bigcup_{𝑖∈ℕ}𝐸_{𝑖}\Bigr).
\end{equation*}
\item Si \((𝐸_{𝑛})_{𝑛∈ℕ}\) est une suite décroissante d'événements,
\begin{gather*}
ℙ(𝐸_{𝑛})\mathop{⟶}_{𝑛→∞}ℙ\Bigl(\bigcap_{𝑖∈ℕ}𝐸_{𝑖}\Bigr).
\end{gather*}
\end{enumerate}
\end{theorem}
\begin{proof}
Dans le cas fini, c'est immédiat car une suite monotone est constante à partir d'un certain rang.
\begin{enumerate}
\item
On définit la suite d'événements \((𝐹_{𝑛})_{𝑛∈ℕ}\) par 
\begin{gather*}
𝐹_{0}\mybydef{=}𝐸_{0},\
∀𝑛∈ℕ^{*},\
𝐹_{𝑛}\mybydef{=}𝐸_{𝑛}∖𝐸_{𝑛-1}
\end{gather*}
Ces événements sont deux à deux disjoints : pour \(0<𝑚<𝑛\)
\begin{align*}
𝐹_{𝑚} ∩ 𝐹_{𝑛}
&{}=
\bigl(𝐸_{𝑚}∖𝐸_{𝑚-1}\bigr)
∩
\bigl(𝐸_{𝑛}∖𝐸_{𝑛-1}\bigr)
\\&{}=
\bigl(𝐸_{𝑚}∩\overline{𝐸_{𝑚-1}}\bigr)
∩
\bigl(𝐸_{𝑛}∩\overline{𝐸_{𝑛-1}}\bigr)
\\&{}=
𝐸_{𝑚}
∩
\overline{𝐸_{𝑛-1}}
⊂𝐸_{𝑚}
∩
\overline{𝐸_{𝑚}}
= ∅
\end{align*}
On peut montrer par récurrence que
\(
∀𝑛∈ℕ,\
𝐸_{𝑛} = 
\bigcup_{𝑖=0}^{𝑛}𝐹_{𝑛}
\).
L'initialisation est immédiate.
L'hérédité vient de
\begin{align*}
\bigcup_{𝑖=0}^{𝑛+1}𝐹_{𝑛} &{}= \bigcup_{𝑖=0}^{𝑛}𝐹_{𝑛}∪𝐹_{𝑛+1}
= 𝐸_{𝑛} ∪ \bigl(𝐸_{𝑛+1}∖𝐸_{𝑛}\bigr)
\\&{}
= \bigl(𝐸_{𝑛} ∩ 𝐸_{𝑛+1}\bigr) ∪ \bigl(𝐸_{𝑛+1}∖𝐸_{𝑛}\bigr)
= 𝐸_{𝑛+1}
\end{align*}
On peut appliquer le théorème précédent pour obtenir la première proposition.
\item
la deuxième proposition se déduit de la première par passage au contraire. En effet, si \((𝐸_{𝑛})_{𝑛∈ℕ}\) est
une suite décroissante d'événements, alors \((\overline{𝐸}_{𝑛})_{𝑛∈ℕ}\) est une suite croissante d'événements\footnote{Le passage au complémentaire définit une application décroissante des parties d'un ensemble.}
donc
\begin{gather*}
ℙ(\overline{𝐸}_{𝑛})
\mathop{⟶}_{𝑛→∞}
ℙ\Bigl(\bigcup_{𝑖∈ℕ}\overline{𝐸}_{𝑖}\Bigl)
\text{ et }
1-ℙ(\overline{𝐸}_{𝑛})
\mathop{⟶}_{𝑛→∞}
1-ℙ\Bigl(\bigcup_{𝑖∈ℕ}\overline{𝐸}_{𝑖}\Bigr).
\intertext{Or}
1-ℙ(\overline{𝐸}_{𝑛})=ℙ(𝐸_{𝑛})
\text{ et }
1-ℙ\Bigl(\bigcup_{𝑖∈ℕ}\overline{𝐸}_{𝑖}\Bigr)=ℙ\Bigl(\bigcap_{𝑖∈ℕ}𝐸_{𝑖}\Bigr).\qedhere
\end{gather*}
\end{enumerate}
\end{proof}
%
\subsubsection{Cas des espaces finis ou dénombrables}
Dans cette section \(Ω\) est fini ou dénombrable.
\paragraph{Événements élémentaires}
\begin{definition}
[Événement élémentaire]
Un événement est \mykeyword{élémentaire} signifie qu'il n'est pas vide et qu'il n'est pas la réunion de
deux événements incompatibles et non vides.
\end{definition}
%
\begin{theorem}
Deux événements élémentaires différents sont incompatibles.
\end{theorem}
\begin{proof}
Soit \(𝐸\), \(𝐹\) deux événements élémentaires, on a
\begin{gather*}
𝐸=(𝐸∩𝐹)∪(𝐸∩\overline{𝐹})\text{ et }∅=(𝐸∩𝐹)∩(𝐸∩\overline{𝐹}),
%\\
%𝐹=(𝐸∩𝐹)∪(\overline{𝐸}∩𝐹)\text{ et }∅=(𝐸∩𝐹)∩(\overline{𝐸}∩𝐹).
\end{gather*}
Donc \(∅=𝐸∩𝐹\) ou bien \(∅=𝐸∩\overline{𝐹}\) et d'une manière symétrique
\(∅=𝐸∩𝐹\) ou bien \(∅=\overline{𝐸}∩𝐹\). Si les événements ne sont pas incompatibles,
alors \(∅=𝐸∩\overline{𝐹}=\overline{𝐸}∩𝐹\) d'où \(𝐸⊂𝐹\) et \(𝐸⊃𝐹\) et l'égalité des événements.
\end{proof}
\begin{theorem}
Dans le cas dénombrable, toute issue appartient à un unique événement élémentaire.
\end{theorem}
\begin{proof}
Si \(𝑥\) est une issue, les événements qui contiennent \(𝑥\) sont en nombre dénombrable, donc leur
intersection est un événement qui contient \(𝑥\), noté ici \(𝐸_{𝑥}\).
Si on a \(𝐸_{𝑥}=𝐸∪𝐹\) avec \(∅=𝐸∩𝐹\), alors \(𝑥∈𝐸\) ou bien \(𝑥∈𝐹\). Dans le premier cas, on obtient \(𝐸_{𝑥}⊂𝐸\) par définition de
\(𝐸_{𝑥}\) et \(𝐸⊂𝐸_{𝑥}\) par définition de \(𝐸\).
On en déduit \(𝐸_{𝑥}=𝐸\) et
\(∅=𝐸∩\overline{𝐸}=(𝐸∪𝐹)∩\overline{𝐸}=(𝐸∩\overline{𝐸})∪(𝐹∩\overline{𝐸})=∅∪(𝐹∩\overline{𝐸})=𝐹∩\overline{𝐸}\).
Ainsi \(𝐹⊂𝐸\) et \(𝐹=𝐹∩𝐸=∅\), donc \(𝐸_{𝑥}\) est élémentaire.
\end{proof}
\begin{theorem}
Dans le cas dénombrable, tout événement est la réunion des événements élémentaires qu'il contient. En
particulier, l'univers est la réunion de tous les événements élémentaires.
\end{theorem}
\begin{proof}
Par définition, tout événement contient les événements élémentaires qu'il contient, donc leur réunion.
L'inclusion réciproque vient de la proposition précédente.
\end{proof}
\begin{theorem}
Si toutes les parties sont des événements, les événements élémentaires sont les singletons.
\end{theorem}
\begin{proof}
Tout singleton est un événement élémentaire. En effet, c'est une partie de l'univers, or toutes ses parties sont des
événements, donc c'est un événement. Les parties de \(\{𝑥\}\) étant \(∅\) et \(\{𝑥\}\), il n'existe pas
deux événements différents et non vides dans \(\{𝑥\}\).

Si un événement \(𝐸\) contient une paire \(\{𝑥,𝑦\}\), alors
\begin{gather*}
𝐸=\bigl(𝐸∩\{𝑥\}\bigr)∪\bigl(𝐸∩\overline{\{𝑥\}}\bigr)=\{𝑥\}∪\bigl(𝐸∩\overline{\{𝑥\}}\bigr).
\end{gather*}
\(𝑦\) est dans \(𝐸\) mais pas \(\{𝑥\}\), il est donc dans \(𝐸∩\overline{\{𝑥\}}\). Ainsi \(𝐸\) n'est pas élémentaire.
Par contraposition, si un événement est élémentaire, alors il ne contient pas de paire. Comme il n'est pas vide, c'est
un singleton.
\end{proof}
\begin{remark}
 on pouvait définir les singletons comme événements élémentaires, la définition ci-dessus
devenant du coup un théorème.
\end{remark}
\begin{theorem}
Une loi de probabilités est entièrement déterminée par la probabilité de chaque événement élémentaire, sauf
éventuellement un.
La somme des probabilités de tous les événements élémentaires vaut 1.
\end{theorem}
\begin{proof}
En exercice...
\end{proof}
\paragraph{Cas des espaces finis}
Dans cette section \(Ω\) est fini.
\begin{definition}
[Loi uniforme]
On appelle loi uniforme toute loi de probabilités qui donne à chaque événement élémentaire la même probabilité.
\end{definition}
\begin{theorem}
Dans le cas d'une loi uniforme sur un espace fini \(Ω\), la probabilité de chaque événement élémentaire
est \(\frac 1{\left|Ω\right|}\).
\end{theorem}
\begin{proof}
Soit \(𝑝\) la probabilité commune à tous les événements élémentaires.
On a
\begin{gather*}
1=ℙ(Ω)=ℙ\Bigl(\bigcup_{𝑥∈Ω}\{𝑥\}\Bigr)=∑_{𝑥∈Ω}ℙ\bigl(\{𝑥\}\bigr)=𝑝\sum
_{𝑥∈Ω}1=𝑝\left|Ω\right|
\qedhere
\end{gather*}
\end{proof}
%
\begin{theorem}
Dans le cas d'une loi uniforme, la probabilité d'un événement est le nombre de cas favorables sur le
nombre de cas possibles.
\end{theorem}
\begin{proof}
\begin{gather*}
ℙ(𝐸)=ℙ\Bigl(\bigcup_{𝑥∈𝐸}\{𝑥\}\Bigr)=
∑_{𝑥∈𝐸}ℙ\bigl(\{𝑥\}\bigr)=\frac
1{\left|Ω\right|}∑_{𝑥∈𝐸}1=\frac{\left|𝐸\right|}{\left|Ω\right|}
\qedhere
\end{gather*}
\end{proof}
%
\subsubsection{Indépendance}
\begin{definition}
[Indépendance]
Deux événements \(𝐸\) et \(𝐹\) sont \mykeyword{indépendants} signifie que
\begin{equation*}
ℙ(𝐸∩𝐹)=ℙ(𝐸)×ℙ(𝐹)
\end{equation*}
On dit aussi que \(𝐸\) est \mykeyword{indépendant} de \(𝐹\).
\end{definition}
\begin{remark}
il n'y a pas de notation symbolique pour deux événements indépendants.
On pourrait noter par exemple \(𝐸⟘𝐹\).
\end{remark}
\begin{lemma}
Tout événement est indépendant d'un événement certain.
\end{lemma}
\begin{proof}
\(𝐸\) et \(𝐹\) deux événements, on a \(𝐸=(𝐸∩𝐹)∪(𝐸∩\overline{𝐹})\) et \((𝐸∩𝐹)∪(𝐸∩\overline{𝐹})= ∅\), d'où
\(ℙ(𝐸)=ℙ(𝐸∩𝐹)+ℙ(𝐸∩\overline{𝐹})\).
Si \(𝐹\) est certain, \(ℙ(𝐹)=1\) et \(ℙ(\overline{𝐹})=0\), et \(ℙ(𝐸∩\overline{𝐹})=0\) puisque
\(𝐸∩\overline{𝐹}⊂\overline{𝐹}\).
Donc
\(ℙ(𝐸∩𝐹)=ℙ(𝐸)=ℙ(𝐸)×ℙ(𝐹)\).
\end{proof}
\begin{theorem}
Tout événement est indépendant d'un événement impossible.
\end{theorem}
\begin{proof}
\(𝐸\) et \(𝐹\) deux événements, \(𝐹\) étant impossible. Comme \((𝐸∩𝐹)⊂𝐹\) et \(0≤ℙ(𝐸∩𝐹)≤ℙ(𝐹)=0\), on a
\(ℙ(𝐸∩𝐹)=0\). Bien sûr, \(ℙ(𝐸)×ℙ(𝐹)=0\).
\end{proof}
\begin{proposition}
Soit \(𝐸\) et \(𝐹\) deux événements. Sont équivalents
\begin{itemize}
\item
\(𝐸\) et \(𝐹\) sont indépendants
\item
\(𝐹\) et \(𝐸\) sont indépendants
\item
\(𝐸\) et \(\overline{𝐹}\) sont indépendants
\end{itemize}
\end{proposition}
\begin{proof}
Comme \(𝐸∩𝐹=𝐹∩𝐸\), on a \(ℙ(𝐸∩𝐹)=ℙ(𝐹∩𝐸)\) d'où l'équivalence des deux premières propositions.

De plus, \(𝐸=(𝐸∩𝐹)∪(𝐸∩\overline{𝐹})\)
et \(∅=(𝐸∩𝐹)∩(𝐸∩\overline{𝐹})\), donc
\(ℙ(𝐸)=ℙ(𝐸∩𝐹)+ℙ(𝐸∩\overline{𝐹})\) et
\begin{align*}
ℙ(𝐸∩𝐹)-ℙ(𝐸)×ℙ(𝐹)&{}=ℙ(𝐸)-ℙ(𝐸∩\overline{𝐹})-ℙ(𝐸)×ℙ(𝐹)
\\
&{}=ℙ(𝐸)×(1-ℙ(𝐹))-ℙ(𝐸∩\overline{𝐹})
\\
&{}=ℙ(𝐸)×ℙ(\overline{𝐹})-ℙ(𝐸∩\overline{𝐹})
\end{align*}
ce qui donne les autres équivalences.
\end{proof}
\begin{example}
Deux événements contraires et possibles ne sont pas indépendants.
\end{example}
%
\subsection{Probabilités composées}
\subsubsection{Restriction.}
On peut changer d'univers en prenant une sous-tribu.
À compléter.
\subsubsection{Probabilités produit dans le cas fini.}
Cela permet de modéliser des combinaisons d'expériences aléatoires indépendantes, comme des répétitions par exemple.

Si \((Ω_1,ℙ_1)\) et \((Ω_2,ℙ_2)\) sont deux univers probabilisés finis, alors \(Ω_1×Ω_2\) est un
univers fini, ses événements élémentaires sont les singletons.
\begin{theorem}
[Probabilité produit]
L'application qui à tout événement élémentaire \(\{(𝑥_1,𝑥_2)\}\) de \(Ω_1×Ω_2\) associe le nombre
\(ℙ_1\bigl(\{𝑥_1\}\bigr)×ℙ_2\bigl(\{𝑥_2\}\bigr)\) définit bien une loi de probabilités sur \(Ω_1×Ω_2\). C'est la \mykeyword{loi produit}.
\end{theorem}
\begin{proof}
On a
\begin{gather*}
∑_{(𝑥_1,𝑥_2)∈Ω_1×Ω_2}ℙ_1\bigl(\{𝑥_1\}\bigr)×ℙ_2\bigl(\{𝑥_2\}\bigr)=∑_{𝑥_1∈Ω_1}ℙ_1\bigl(\{𝑥_1\}\bigr)∑_{𝑥_2∈Ω_2}ℙ_2\bigl(\{𝑥_2\}\bigr)
\intertext{qui vient de}
∑_{(𝑖,𝑗)∈𝐼×𝐽}𝑝_{𝑖}𝑞_{𝑗}=∑_{𝑖∈𝐼}𝑝_{𝑖}∑_{𝑗∈𝐽}𝑞_{𝑗}
 \qedhere
\end{gather*}
\end{proof}
\begin{remark}
Pour les répétitions d'une même expérience modélisée par \((Ω, ℙ)\) une infinité dénombrable de fois,
on a aussi une probabilité produit sur \(Ω^ℕ\). Les événements de références ne sont pas les événements élémentaires
mais des cylindres, c'est-à-dire des suites d'événements de \(Ω\) qui valent \(Ω\) à partir d'un certain rang. À
compléter...
\end{remark}
\subsubsection{Probabilités conditionnelles}
La notion est générale, on ne l'appliquera qu'aux espaces dénombrables.

 \((Ω,𝒯,ℙ)\) est un espace probabilisé.
\begin{theorem}
[Probabilité conditionnelle]
Si \(𝐴\) est un événement possible, l'application \(𝐸↦ℙ(𝐸∩𝐴)/ℙ(𝐴)\) définit une loi de probabilités
sur \((Ω,𝒯)\) qui est notée \(ℙ_{𝐴}\).
\end{theorem}
\begin{proof}
Pour tout événement \(𝐸\), \(𝐸∩𝐴⊂𝐴\), donc \(ℙ(𝐸∩𝐴)≤ℙ(𝐴)\). Comme \(ℙ(𝐴)≠0\),
\(0≤\frac{ℙ(𝐸∩𝐴)}{ℙ(𝐴)}≤1\).
Ensuite
\(\frac{ℙ(Ω∩𝐴)}{ℙ(𝐴)}=\frac{ℙ(𝐴)}{ℙ(𝐴)}=1\).
Pour une famille d'événements \((𝐸_{𝑖})_{𝑖∈𝐼}\) mutuellement incompatibles, les événements
\((𝐸_{𝑖}∩𝐴)_{𝑖∈𝐼}\) sont aussi mutuellement incompatibles car
\((𝐸_{𝑖}∩𝐴)∩(𝐸_{𝑗}∩𝐴)=(𝐸_{𝑖}∩𝐸_{𝑗})∩𝐴\).
de plus, on a
\(\left(\bigcup_{𝑖∈𝐼}𝐸_{𝑖}\right)∩𝐴=\left(\bigcup_{𝑖∈𝐼}𝐸_{𝑖}∩𝐴\right)\),
ce qui donne
\begin{equation*}
ℙ\Bigl(\bigcup_{𝑖∈𝐼}\bigl(𝐸_{𝑖}∩𝐴\bigr)\Bigr)=∑_{𝑖∈𝐼}ℙ\left(𝐸_{𝑖}∩𝐴\right)
\end{equation*}
d'où la σ-additivité en divisant par \(ℙ(𝐴)\).
\end{proof}
\begin{theorem}
Deux événements possibles \(𝐸\) et \(𝐹\) sont indépendants si et seulement si \(ℙ(𝐸)=ℙ_{𝐹}(𝐸)\) ou
de manière symétrique \(ℙ(𝐹)=ℙ_{𝐸}(𝐹)\).
\end{theorem}
\begin{proof}
C'est une application directe de la définition de l'indépendance.
\end{proof}
\begin{definition}
[Système complet d'événements]
Une famille dénombrable \((𝐸_{𝑖})_{𝑖∈𝐼}\) d'événements est un \mykeyword{système complet} d'événements
signifie que
\begin{enumerate}
\item ils sont mutuellement incompatibles : \(∀𝑖,𝑗∈𝐼,\;𝑖≠𝑗⟹𝐸_{𝑖}∩𝐸_{𝑗}=∅\),
\item leur réunion est l'univers : \(\bigcup_{𝑖∈𝐼}𝐸_{𝑖}=Ω\),
\item ils sont tous possibles : \(∀𝑖∈𝐼,\;ℙ(𝐸_{𝑖})≠0\).
\end{enumerate}
\end{definition}
\begin{information}
Dans le contexte dénombrable, les événements élémentaires forment un système complet d'événements. Un tel système peut
être utilisé pour restreindre l'ensemble des événements et simplifier le modèle.
\end{information}
\begin{theorem}
[des probabilités totales]
Pour un système complet d'événements \((𝐸_{𝑖})_{𝑖∈𝐼}\) et un événement \(𝐸\), on a
\begin{gather*}
ℙ(𝐸)=∑_{𝑖∈𝐼}ℙ(𝐸_{𝑖})\;ℙ_{𝐸_{𝑖}}(𝐸).
\end{gather*}
\end{theorem}
\begin{proof}
D'une part
\begin{equation*}
𝐸=𝐸∩Ω=𝐸∩\bigcup_{𝑖∈𝐼}𝐸_{𝑖}=\bigcup_{𝑖∈𝐼}𝐸∩𝐸_{𝑖}
\end{equation*}
et d'autre part les \(𝐸∩𝐸_{𝑖}\) sont mutuellement incompatibles car si \(𝑖≠𝑗\) dans \(𝐼\),
\((𝐸∩𝐸_{𝑖})∩(𝐸∩𝐸_{𝑗})=𝐸∩(𝐸_{𝑖}∩𝐸_{𝑗})=𝐸∩∅=∅\).
Donc
\begin{align*}
ℙ(𝐸)&{}=ℙ\left(\bigcup_{𝑖∈𝐼}𝐸∩𝐸_{𝑖}\right)
\\&{}=∑_{𝑖∈𝐼}ℙ(𝐸∩𝐸_{𝑖})
\\&{}=∑_{𝑖∈𝐼}ℙ(𝐸_{𝑖})\frac{ℙ(𝐸∩𝐸_{𝑖})}{ℙ(𝐸_{𝑖})}
\\&{}=∑_{𝑖∈𝐼}ℙ(𝐸_{𝑖})\;ℙ_{𝐸_{𝑖}}(𝐸).
\qedhere
\end{align*}
\end{proof}
%
\begin{theorem}
[de Bayes]
Pour un système complet d'événements \((𝐸_{𝑖})_{𝑖∈𝐼}\) et un événement \(𝐸\) possible, on a
\begin{gather*}
ℙ_{𝐸}(𝐸_{𝑖})=\frac{ℙ(𝐸_{𝑖})ℙ_{𝐸_{𝑖}}(𝐸)}{∑_{𝑗∈𝐼}ℙ(𝐸_{𝑗})ℙ_{𝐸_{𝑗}}(𝐸)}.
\end{gather*}
\end{theorem}
\begin{proof}
C'est quasi immédiat avec la formule des probabilités totales :
\begin{gather*}
ℙ_{𝐸}(𝐸_{𝑖})\mybydef{=}\frac{ℙ(𝐸∩𝐸_{𝑖})}{ℙ(𝐸)}=\frac{ℙ(𝐸_{𝑖})\;ℙ_{𝐸_{𝑖}}(𝐸)}{∑_{𝑗∈𝐼}ℙ(𝐸_{𝑗})\;ℙ_{𝐸_{𝑗}}(𝐸)}
\qedhere
\end{gather*}
\end{proof}
\begin{remark}
cette formule est tellement simple qu'on ne la retient pas, on la
retrouve selon les besoins.
\end{remark}
