% !TEX encoding = UTF-8
% !TEX program = xelatex
% !TEX root =../Main/main.tex

\section{Relations}
Relations d’ordre et relations d’équivalence.
\subsection{Généralités}
\begin{definition}
[Relation binaire]
Une \mykeyword{relation binaire} \(𝓡\) sur un ensemble \(𝐸\) est donnée par une partie
\(Γ\) de \(𝐸×𝐸\). Pour deux éléments \(𝑥\) et \(𝑦\) de \(𝐸\),
\(𝑥𝓡𝑦\) signifie que le couple \((𝑥,𝑦)\) est dans \(Γ\).

\(𝑥𝓡𝑦\) est lu  : « \(𝑥\) est  en relation avec \(𝑦\) par \(𝓡\)». On omettra « par \(𝓡\) » si le contexte est
suffisamment clair.
\end{definition}
%
\begin{definition}
[Relation binaire réciproque]
À toute relation binaire \(𝓡\) sur un ensemble \(𝐸\) est associée la relation binaire réciproque
\(𝓡'\) définie par
\begin{equation*}
𝑥𝓡'𝑦 \mybydef{⟺}𝑦𝓡𝑥
\end{equation*}
\end{definition}
\begin{definition}
[Réflexivité]
Une relation binaire \(𝓡\) sur un ensemble \(𝐸\) est \mykeyword{réflexive} signifie que
\begin{equation*}
∀𝑥∈𝐸,\;𝑥𝓡𝑥
\end{equation*}
\end{definition}
\begin{definition}
[Symétrie]
Une relation binaire \(𝓡\)sur un ensemble \(𝐸\) est \mykeyword{symétrique} signifie que
\begin{equation*}
∀(𝑥,𝑦)∈𝐸^2\;(𝑥𝓡𝑦⟹𝑦𝓡𝑥).
\end{equation*}
\end{definition}
\begin{terminology}
Dans le cas d'une relation symétrique,
\(𝑥𝓡𝑦\) peut être lu «\(𝑥\) et \(𝑦\) sont en relation par \(𝓡\)».
\end{terminology}
\begin{definition}
[Antisymétrie]
Une relation binaire \(𝓡\) sur un ensemble \(𝐸\) est \mykeyword{antisymétrique} signifie que
\begin{equation*}
∀(𝑥,𝑦)∈𝐸^2\;(𝑥𝓡𝑦\myand𝑦𝓡𝑥⟹𝑥=𝑦).
\end{equation*}
\end{definition}
\begin{definition}
[Transitivité]
Une relation binaire \(𝓡\) sur un ensemble \(𝐸\) est \mykeyword{transitive} signifie que
\begin{equation*}
∀(𝑥,𝑦,𝑧)∈𝐸^3\;(𝑥𝓡𝑦\myand𝑦𝓡𝑧⟹𝑥𝓡𝑧).
\end{equation*}
\end{definition}
\subsection{Ordre}
Voir le chapitre sur les corps totalement ordonnés.
\subsection[Équivalence]{Équivalence}
\subsubsection{Définitions}
\begin{definition}
[Équivalence]
Une relation binaire est une \mykeyword{relation d'équivalence} signifie qu'elle est réflexive,
symétrique et transitive.
\end{definition}
\begin{terminology}
[Classe d'équivalence]
Soit \(𝓡\) une relation d'équivalence sur \(𝐸\), et \(𝑥\) un élément de
E. La \mykeyword{classe d'équivalence} de \(𝑥\) \mykeyword{modulo}
\(𝓡\) est \(\bigl\{𝑦∈𝐸\mathbin{|}𝑥𝓡𝑦\bigr\}\).

Elle est notée \(\overset{∙}{𝑥}\). On pourra omettre
«\mykeyword{modulo} \(𝓡\)» s'il n'y a pas ambiguïté.
Un élément d'une classe d'équivalence en est une \mykeyword{représentant}.
\end{terminology}
%
\subsubsection{Propriétés}
\begin{theorem}
\label{seq:refTheorem0}

La classe d'équivalence d'un élément contient au moins cet élément. En particulier, elle n'est pas vide.
\end{theorem}
\begin{proof}
Immédiat par la réflexivité.
\end{proof}
\begin{theorem}
\label{seq:refTheorem1}

Pour une même relation d'équivalence, deux classes sont disjointes ou bien confondues.
\end{theorem}
\begin{proof}
Soit \(ℛ\) une relation d'équivalence sur \(𝐸\), \(𝑥_0\) et \(𝑦_0\) deux éléments de
\(E\). On a
\begin{equation*}
\overset{∙}{𝑥}_0∩\overset{∙}{𝑦}_0=∅\myxor\overset{∙}{𝑥}_0∩\overset{∙}{𝑦}_0≠∅
\end{equation*}
 Si \(\overset{∙}{𝑥}_0∩\overset{∙}{𝑦}_0≠∅\), soit \(𝑧_0\) élément
commun.
On a \(𝑥_0ℛ𝑧_0\) et \(𝑦_0ℛ𝑧_0\) donc \(𝑥_0ℛ𝑦_0\).
Pour tout \(𝑥\) de \(\overset{∙}{𝑥}_0\),
on a \(𝑥ℛ𝑥_0\) et \(𝑥_0ℛ𝑦_0\) donc \(𝑥ℛ𝑦_0\),
\(𝑥\) est dans \(\overset{∙}{𝑦_0}\).
Pour tout représentant \(𝑦\) de \(\overset{∙}{𝑦}_0\),
on a \(𝑦ℛ𝑦_0\) et \(𝑦_0ℛ𝑥_0\) donc
\(𝑦ℛ𝑥_0\), \(𝑦\) est dans \(\overset{∙}{𝑥}_0\). Ainsi, deux classes non disjointes sont
confondues.

Inversement, deux classes confondues ne sont pas disjointes puisqu'elles ne sont pas vides, ce qui donne
\begin{equation*}
\overset{∙}{𝑥}_0∩\overset{∙}{𝑦}_0≠∅⟺\overset{∙}{𝑥}_0=\overset{∙}{𝑦}_0
\end{equation*}
et le résultat.
\end{proof}
\begin{theorem}
Toute classe d'équivalence est la classe d'équivalence de chacun de ses éléments.
\end{theorem}
\begin{proof}
Soit \(ℛ\) une relation d'équivalence sur \(𝐸\), et \(𝑥_0\) un élément de E.
Pour tout \(𝑥\) de \(\overset{∙}{𝑥}_0\),
\(\overset{∙}{𝑥}\) et \(\overset{∙}{𝑥}_0\)
ont en commun \(𝑥\) donc sont confondues.
\end{proof}
\begin{lemma}
Soit \(𝓡\) une relation d'équivalence sur \(𝐸\), deux éléments de \(𝐸\) sont en relation
si et seulement s'ils ont la même classe d'équivalence.
\end{lemma}
\begin{proof}
On a
\(𝑥𝓡𝑦⟺(𝑥∈\overset{∙}{𝑥}\myand𝑥∈\overset{∙}{𝑦})⟺\overset{∙}{𝑥}=\overset{∙}{𝑦}\).
\end{proof}

%
\subsubsection{Projection canonique}
\begin{definition}
[Ensemble quotient]
Soit \(𝓡\) une relation d'équivalence sur \(𝐸\). L'ensemble des classes d'équivalence est
l'\mykeyword{ensemble quotient} de \(𝐸\) par \(𝓡\), noté \(𝐸⁄𝓡\).
L'application de \(𝐸\) dans \(𝐸⁄𝓡\) qui à tout élément de \(𝐸\)
associe sa classe d'équivalence est l'\mykeyword{application} (ou \mykeyword{projection})
\mykeyword{canonique}.
\end{definition}
\begin{theorem}
La projection canonique est surjective.
\end{theorem}
\begin{proof}
En exercice...
\end{proof}
\begin{remark}
les propriétés de 𝐸 qui ne dépendent que de la classe d'équivalence se
transportent à l'ensemble quotient.
\end{remark}
\subsubsection{Partitions}
\begin{theorem}
Soit \(𝓡\) une relation d'équivalence sur \(𝐸\). Les classes d'équivalence définissent une partition de \(𝐸\).
\end{theorem}
\begin{proof}
Par les lemmes \ref{seq:refTheorem0} et \ref{seq:refTheorem1}, les classes d'équivalences sont non
vides et mutuellement disjointes. Comme tout élément de \(𝐸\) est dans sa classe d'équivalence, la réunion des classes est
\(𝐸\).
\end{proof}
\begin{proposition}
Toute partition de \(𝐸\) définit sur \(𝐸\) une relation d'équivalence dont les classes
coïncident avec les éléments de la partition.
\end{proposition}
\begin{proof}
Soit \(𝑃\) une partition de\textsubscript{ }𝐸. On pose
\begin{equation*}
𝑥ℛ𝑦 \mybydef{⟺}∃𝐹∈𝑃,\ (𝑥∈𝐹\myand𝑦∈𝐹)
\end{equation*}
À finir en exercice...
\end{proof}
\begin{theorem}
[Relation et application]
Étant donnée une application \(𝑓\) de \(𝛦\) dans \(𝐹\),
\begin{equation*}
𝑥𝓡𝑦 \mybydef{⟺}𝑓(𝑥)=𝑓(𝑦)
\end{equation*}
est une relation d'équivalence dans \(𝛦\).
\end{theorem}
\begin{proof}
Pour appliquer le résultat précédent : \(\bigcup_{𝑦∈𝑓(𝛦)}\overset{-1}{𝑓}(𝑦)\) est une partition
de \(𝛦\).
\begin{enumerate}
\item
Tout \(𝑦\) de \(𝑓(𝛦)\) est un \(𝑓(𝑥)\), donc \(\overset{-1}{𝑓}(𝑦)\) contient \(𝑥\) et
\(\overset{-1}{𝑓}(𝑦)≠∅\).
\item
Tout \(𝑥\) de \(𝛦\) est dans \(\overset{-1}{𝑓}(𝑓(𝑥))\). Or \(𝑓(𝑥)\) est dans \(𝑓(𝛦)\), donc
\(\bigcup_{𝑦∈𝑓(𝛦)}?\overset{-1}{?}(𝑦)\), qui est une partie de \(𝛦\), contient
\(\overset{-1}{𝑓}(𝑓(𝑥))\) et \(𝑥\). Donc \(𝐸=\bigcup_{𝑦∈𝑓(𝛦)}\overset{-1}{𝑓}(𝑦)\).
\item
Pour tous \(𝑦\) et \(𝑦'\) de \(𝑓(𝛦)\), si \(\overset{-1}{𝑓}(𝑦)∩\overset{-1}{𝑓}(𝑦')\) contient \(𝑥\) alors
\(𝑓(𝑥)=𝑦=𝑦'\). Par contraposition, si \(𝑦\) et \(𝑦'\) sont différents,
\(\overset{-1}{𝑓}(𝑦)∩\overset{-1}{𝑓}(𝑦')\) est vide.
\end{enumerate}
\end{proof}
