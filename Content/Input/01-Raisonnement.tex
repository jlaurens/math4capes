% !TEX program = xelatex
% !TEX encoding = UTF8
% !TEX root = ../Main/main.tex

\section{Éléments de logique}
On utilise la définition d'ensemble ainsi que les calculs algébriques sur les petits entiers. On présente un peu de
calcul propositionnel.

\subsection{Bases}
\begin{axiom}
[Proposition]
\(\mytrue\) et \(\myfalse\) sont deux objects mathématiques dénommés valeurs logiques.
\end{axiom}
\begin{vocabulary}
\par\noindent
\begin{itemize}
\item
valeur de vérité est synonyme de valeur logique.
\item
\(𝗩\) et \(1\) sont synonymes de \(\mytrue\).
\item
\(𝗙\) et \(0\) sont synonymes de \(\myfalse\).
\end{itemize}
\end{vocabulary}
En particulier, \(\myfalse=1-\mytrue\) ,
\(\myfalse<\mytrue\).

%
\begin{definition}
[Proposition]
Une proposition est un objet mathématique auquel est associée une valeur logique
\end{definition}
\begin{definition}
\par\noindent
\begin{itemize}
\item
\(\mytrue\) est une proposition à laquelle est associée la valeur logique \(\mytrue\).
\item
\(\myfalse\) est une proposition à laquelle est associée la valeur logique \(\myfalse\).
\end{itemize}
\end{definition}
\begin{notation}
\(𝚟𝚕(𝖯)\) désigne la valuer logique de la proposition \(𝖯\).
\end{notation}
En particulier, \(𝚟𝚕=𝚟𝚕^2\)
ou \(𝚟𝚕×(1-𝚟𝚕)=0\)
\begin{definition}
[Théorie]
Une théorie est un ensemble de propositions.
\end{definition}
\begin{definition}
[contraire, et, ou]
Si \(𝖯\) et \(𝖰\) sont des propositions,
\begin{itemize}
\item \(\mynot 𝖯\) est une proposition appelée contraire de
\(𝖯\), aussi notée \(¬𝖯\) ou \(\overline{𝖯}\),
\item \(𝖯 \myand 𝖰\) est une proposition appelée conjonction de
\(𝖯\) et \(𝖰\), aussi notée \(𝖯\mathbin{∧}𝖰\),
\item \(𝖯 \myor 𝖰\) est une proposition appelée disjonction de
\(𝖯\) et \(𝖰\), aussi notée \(𝖯 \mathbin{∨} 𝖰\).
\end{itemize}
\end{definition}
\begin{vocabulary}
[Propositions composées, atomiques]
Une proposition est complexe signifie qu'elle est sous une des formes \(\mynot 𝖯\), \(𝖯 \myand 𝖰\) ou \(𝖯 \myor 𝖰\).
Un proposition est atomique signifie qu'elle n'est pas sous l'une de ces formes.
\end{vocabulary}
\begin{definition}
[Priorité]
Par ordre de priorité décroissante des opérateurs, il vient \mynot , puis
\(\myand\) et \(\myor\).
\end{definition}

\begin{remark}
à rapprocher de certains langages informatiques.
\end{remark}

\begin{definition}
[Table de vérité]
Soit \(𝖯\) une proposition construite à partir d'un nombre fini de propositions atomiques
\(𝖰_{𝑖}\).
La table de vérité de \(𝖯\) est l'application qui associe à chaque combinaison possible des valeurs de vérité des
\(𝖰_{𝑖}$ une valeur de vérité à \(𝖯\) correspondante.
\end{definition}

\begin{axiom}[Tables de vérité de la négation, de la conjonstion et de la disjonction]
\begin{equation*}
\setlength{\extrarowheight}{0.25ex}
\begin{array}{|*{2}{>{\(}wc{1cm}<{\)}|}}
\hline
𝖯 &
¬𝖯
\\\hline
𝗩 &
𝗙
\\\hline
𝗙 &
𝗩
\\\hline
\end{array},\
\begin{array}{|*{2}{>{\(}wc{1cm}<{\)}|}>{\(}wc{1.5cm}<{\)}|}
\hline
𝖯 &
𝖰 &
𝖯\mathbin{∧}𝖰
\\\hline
𝗩 &
𝗩 &
𝗩
\\\hline
𝗩 &
𝗙 &
𝗙
\\\hline
𝗙 &
𝗩 &
𝗙
\\\hline
𝗙 &
𝗙 &
𝗙
\\\hline
\end{array},\
\begin{array}{|*{2}{>{\(}wc{1cm}<{\)}|}>{\(}wc{1.5cm}<{\)}|}
\hline
𝖯 &
𝖰 &
𝖯\mathbin{∨}𝖰
\\\hline
𝗩 &
𝗩 &
𝗩
\\\hline
𝗩 &
𝗙 &
𝗩
\\\hline
𝗙 &
𝗩 &
𝗩
\\\hline
𝗙 &
𝗙 &
𝗙
\\\hline
\end{array}
\end{equation*}
\end{axiom}
%
\begin{remark}De manière synthétique, on a
\begin{equation*}
\setlength{\extrarowheight}{0.25ex}
\begin{array}{|*{5}{>{\(}wc{1.5cm}<{\)}|}}
\hline
𝖯 &
𝖰 &
¬𝖯 &
𝖯\mathbin{∧}𝖰 &
𝖯\mathbin{∨}𝖰
\\\hline
𝗩 &
𝗩 &
𝗙 &
𝗩 &
𝗩
\\\hline
𝗩 &
𝗙 &
𝗙 &
𝗙 &
𝗩
\\\hline
𝗙 &
𝗩 &
𝗩 &
𝗙 &
𝗩
\\\hline
𝗙 &
𝗙 &
𝗩 &
𝗙 &
𝗙
\\\hline
\end{array}
\end{equation*}
\end{remark}

En termes de valeur logique :
\begin{equation*}
𝚟𝚕(¬𝖯)=1-𝚟𝚕(𝖯)
\end{equation*}
\begin{equation*}
𝚟𝚕(𝖯\mathbin{∧}𝖰)=𝚟𝚕(𝖯).𝚟𝚕(𝖰)
\end{equation*}
\begin{equation*}
𝚟𝚕(𝖯\mathbin{∨}𝖰)=𝚟𝚕(𝖯)+𝚟𝚕(𝖰)-𝚟𝚕(𝖯).𝚟𝚕(𝖰)
\end{equation*}

%
\begin{definition}
[Tautologie, antinomie]
Une tautologie est une proposition qui n'a que la valeur logique $\mytrue$ ,
une antinomie est une proposition qui n'a que la valeur logique $\myfalse$ .
\end{definition}
%
\begin{theorem}
Soit \(𝖯\) et \(𝖰\) des propositions.
Si \(𝖯\) est une tautologie, alors \(𝖯 \myor 𝖰\) est une tautologie.
Si en plus \(𝖰\) est une tautologie, alors \(𝖯\) \myand \(𝖰\) est une tautologie.

Si \(𝖯\) est une antinomie, alors \(𝖯 \myand 𝖰\) est une antinomie.
Si en plus \(𝖰\) est une antinomie, alors \(𝖯 \myor 𝖰\) est une tautologie.
\end{theorem}
\begin{proof}
Dans le premier cas, cela correspond aux deux premières lignes du tableau précédent.
Uniquement la première dans le deuxième cas.
\end{proof}

De manière symétrique.
\begin{theorem}
Soit \(𝖯\) et \(𝖰\) des propositions.
Si \(𝖯\) est une antinomie, alors \(𝖯 \myand 𝖰\) est une antinomie.
Si en plus \(𝖰\) est une antinomie, alors \(𝖯 \myor 𝖰\) est une antinomie.
\end{theorem}
\begin{proof}
Dans le premier cas, ce sont les deux dernières lignes du tableau précédent.
Uniquement la dernière dans le deuxième
cas.
\end{proof}

\begin{theorem}
[Tiers exclu]
\(𝖯 \myor \mynot 𝖯\) est une tautologie.
\end{theorem}

\begin{theorem}
[Non contradiction]
\(𝖯 \myand \mynot𝖯\) est une antinomie.
\end{theorem}
\begin{proof}
On a regroupé ci-dessous les tables de vérités de
\(𝖯 \myand \mynot 𝖯\) et \(𝖯 \myor \mynot 𝖯\).
\begin{equation*}
\begin{array}{|*{4}{>{\(}wc{2.0cm}<{\)}|}}
\hline
𝖯 &
¬𝖯 &
𝖯\mathbin{∧}¬𝖯 &
𝖯\mathbin{∨}¬𝖯\\\hline
𝗩 &
𝗙 &
𝗙 &
𝗩
\\\hline
𝗙 &
𝗩 &
𝗙 &
𝗩
\\\hline
\end{array}
\end{equation*}

On peut le lire dans le tableau précédent, mais on a aussi

\begin{align*}
𝚟𝚕(𝖯\mathbin{∧}¬𝖯)
&{}=𝚟𝚕(𝖯).𝚟𝚕(¬𝖯)
\\&{}=𝚟𝚕(𝖯).(1-𝚟𝚕(𝖯))
\\&{}=0
\end{align*}
\begin{align*}
𝚟𝚕(𝖯\mathbin{∨}¬𝖯)
&{}=𝚟𝚕(𝖯)+𝚟𝚕(¬𝖯)-𝚟𝚕(𝖯).𝚟𝚕(¬𝖯)
\\&{}=𝚟𝚕(𝖯)+1-𝚟𝚕(𝖯)-𝚟𝚕(𝖯).(1-𝚟𝚕(𝖯))
\\&{}=1
\qedhere
\end{align*}
\end{proof}
%
\subsection{Implications}
\begin{definition}
[Implication]
Si \(𝖯\) et \(𝖰\) sont des propositions, on pose :
\begin{equation*}
(𝖯⇒𝖰)\mybydef{=}(\bigl(\mynot 𝖯)\myor𝖰\bigr)
\end{equation*}
\begin{equation*}
(𝖰⇐𝖯)\mybydef{=}(𝖯⇒𝖰)
\end{equation*}
\(⇒\) est lu «implique» ou entraine.
\(⇐\) est lu «est impliqué par» ou «est entrainé par».
La proposition complexe \(𝖯⇒𝖰\) est une implication. \(𝖯\) et la prémisse, \(𝖰\) est la
conclusion.
\end{definition}

\subsection{Équivalences}
\begin{definition}[Équivalence]
Si \(𝖯\) et \(𝖰\) sont des propositions, on pose
\begin{equation*}
𝖯⟺𝖰
\overset{\text{déf}}{=}(𝖯\myand𝖰)\myor\bigl((\mynot 𝖯)\myand(\mynot 𝖰)\bigr)
\end{equation*}
\end{definition}
\begin{vocabulary}
\(𝖯\) et \(𝖰\) sont équivalentes signifie que \(𝖯⟺𝖰\) est vraie.
De même pour \(𝖯\) équivaut à \(𝖰\).
Les équivalences sont les propositions complexes de la forme \(𝖯⟺𝖰\).
\end{vocabulary}
%
\begin{theorem}[Tautologies]
Si \(𝖯\) et \(𝖰\) sont des propositions, les équivalences suivantes sont des tautologies :
\begin{itemize}
\item
\(\bigl(\mynot (\mynot 𝖯)\bigr)⟺𝖯\)
\item
\((𝖯\myand𝖯)⟺𝖯\),
\item
\((𝖯\myor𝖯)⟺𝖯\)
\item
\((𝖯\myand𝖰)⟺(𝖰\myand𝖯)\)
\item
\(\displaystyle
\mynot(𝖯\myand𝖰)⟺\bigl((\mynot𝖯)\myor(\mynot𝖰)\bigr)
\)
\item
\(\displaystyle
\mynot(𝖯\myor𝖰)⟺\mynot𝖯)\myand(\mynot 𝖰)
\)
\end{itemize}
\end{theorem}
\begin{proof}
Pour les deux premières lignes :
\begin{equation*}
\begin{array}{|*{3}{>{\(}wc{1cm}<{\)}|}*{3}{>{\(}wc{1.5cm}<{\)}|}}
\hline
𝖯 &
𝖯 &
¬𝖯 &
¬(¬𝖯) &
𝖯\mathbin{∧}𝖯 &
𝖯\mathbin{∨}𝖯
\\\hline
1 &
1 &
0 &
1 &
1 &
1\\\hline
0 &
0 &
1 &
0 &
0 &
0\\\hline
\end{array}
\end{equation*}
Il est facile de voir à partir de leurs définitions que \(\myand\) et
\(\myor\) sont commutatifs par équivalence. Pour leurs négations :
\begin{equation*}
\begin{array}{|*4{>{\(}wc{0.5cm}<{\)}|}>{\(}wc{1cm}<{\)}|*6{>{\(}wc{1.5cm}<{\)}|}}
\hline
𝖯 & ¬𝖯 & 𝖰 & ¬𝖰 &
𝖯\mathbin{∧}𝖰 &
¬(𝖯\mathbin{∧}𝖰) &
¬𝖯\mathbin{∨}¬𝖰 &
𝖯\mathbin{∨}𝖰 &
¬(𝖯\mathbin{∨}𝖰) &
¬𝖯\mathbin{∧}¬𝖰\\\hline
1 &
0 &
1 &
0 &
1 &
0 &
0 &
1 &
0 &
0\\\hline
1 &
0 &
0 &
1 &
0 &
1 &
1 &
1 &
0 &
0\\\hline
0 &
1 &
1 &
0 &
0 &
1 &
1 &
1 &
0 &
0\\\hline
0 &
1 &
0 &
1 &
0 &
1 &
1 &
0 &
1 &
1\\\hline
\end{array}
\end{equation*}
\begin{equation*}
\begin{array}{|>{\(}wc{0.5cm}<{\)}|
>{\(}wc{0.75cm}<{\)}|
>{\(}wc{0.5cm}<{\)}|
>{\(}wc{0.75cm}<{\)}|
>{\(}wc{1.25cm}<{\)}|
>{\(}wc{1.5cm}<{\)}|
>{\(}wc{1.25cm}<{\)}|
>{\(}wc{1.5cm}<{\)}|
*2{>{\(}wc{1.5cm}<{\)}|}}
\hline
𝖯 &
¬𝖯 &
𝖰 &
¬𝖰 &
𝖯\mathbin{∧}𝖰 &
¬𝖯\mathbin{∧}¬𝖰 &
𝖯⇔𝖰 &
𝖯⇒𝖰 &
𝖰⇒𝖯 &
\stackrel{(𝖯⇒𝖰)\mathbin{∧}}{(𝖰⇒𝖯)}\\\hline
1 &
0 &
1 &
0 &
1 &
0 &
1 &
1 &
1 &
1\\\hline
1 &
0 &
0 &
1 &
0 &
0 &
0 &
0 &
1 &
0\\\hline
0 &
1 &
1 &
0 &
0 &
0 &
0 &
1 &
0 &
0\\\hline
0 &
1 &
0 &
1 &
0 &
1 &
1 &
1 &
1 &
1\\\hline
\end{array}
\end{equation*}
\end{proof}
%
\begin{definition}
Étant donné \(𝖯\) et \(𝖰\) deux propositions.
\(𝖯⟺𝖰\) signifie que \(𝚟𝚕(𝖯)=𝚟𝚕(𝖰)\).
\(𝖯⟺𝖰\) est lu «\(𝖯\) équivaut à \(𝖰\)» ou «\(𝖯\) est équivalent à \(𝖰\)».
\end{definition}

\begin{proof}
Par la table de vérité :

\begin{equation*}
\setlength{\extrarowheight}{0.25ex}
\mynewcolumntype{A}{0.5cm}
\mynewcolumntype{B}{1.25cm}
\mynewcolumntype{C}{2.75cm}
\begin{array}{|AABC}
\hline
𝖯 &
𝖰 &
𝖯⇔𝖰 &
𝚟𝚕(𝖯)=𝚟𝚕(𝖰)\\\hline
1 &
1 &
1 &
1\\\hline
1 &
0 &
0 &
0\\\hline
0 &
1 &
0 &
0\\\hline
0 &
0 &
1 &
1\\\hline
\end{array}
\end{equation*}
et par le calcul :
\begin{align*}
𝚟𝚕(𝖯⇔𝖰)
&{}=
𝚟𝚕((𝖯\mathbin{∧}𝖰)𝚟(¬𝖯\mathbin{∧}¬𝖰))
\\&{}=
𝚟𝚕(𝖯\mathbin{∧}𝖰)+𝚟𝚕(¬𝖯\mathbin{∧}¬𝖰)-𝚟𝚕(𝖯\mathbin{∧}𝖰).𝚟𝚕(¬𝖯\mathbin{∧}¬𝖰)
\\&{}=
𝚟𝚕(𝖯).𝚟𝚕(𝖰)+𝚟𝚕(¬𝖯).𝚟𝚕(¬𝖰)-𝚟𝚕(𝖯).𝚟𝚕(𝖰).𝚟𝚕(¬𝖯).𝚟𝚕(¬𝖰)
\\&{}=
𝚟𝚕(𝖯).𝚟𝚕(𝖰)+(1-𝚟𝚕(𝖯)).(1-𝚟𝚕(𝖰))
\\&{}=
1+2.𝚟𝚕(𝖯).𝚟𝚕(𝖰)-𝚟𝚕(𝖯)-𝚟𝚕(𝖰)
\\&{}=
1+2.𝚟𝚕(𝖯).𝚟𝚕(𝖰)-𝚟𝚕^2(𝖯)-𝚟𝚕^2(𝖰)
\\&{}=
1-\left(𝚟𝚕(𝖯)-𝚟𝚕(𝖰)\right)^2
\end{align*}
\end{proof}
%
\begin{theorem}
[Substitution]
Dans une proposition complexe, on peut remplacer une proposition argument d'un des opérateurs par une proposition qui
lui est équivalente et obtenir ainsi une nouvelle proposition qui est équivalente à la celle de départ.
\end{theorem}
\begin{proof}
On pourrait le démontrer modulo un formalisme adapté, mais c'est lourd...
\end{proof}
\begin{remark}
Dans la suite, on n'utilise plus \(𝗩\) et \(𝗙\), seulement \(1\) et \(0\).
\end{remark}
%
\begin{theorem}
[Tautologies]
Si \(𝖯\) et \(𝖰\) sont des propositions, les propositions suivantes sont des tautologies :
\begin{itemize}
\item
\(𝖯⇒𝖯\),
\item
\(𝖯⇒(𝖯\myor𝖰)\)
\end{itemize}
\end{theorem}
\begin{proof}
\begin{equation*}
\mynewcolumntype{A}{1cm}
\mynewcolumntype{B}{1.5cm}
\mynewcolumntype{C}{2.5cm}
\begin{array}{|AAABBC}
\hline
𝖯 &
𝖰 &
¬𝖯 &
(¬𝖯)\mathbin{∨}𝖯 &
𝖯\mathbin{∨}𝖰 &
(¬𝖯)\mathbin{∨}(𝖯\mathbin{∨}𝖰)\\\hline
1 &
1 &
0 &
1 &
1 &
1\\\hline
1 &
0 &
0 &
1 &
1 &
1\\\hline
0 &
1 &
1 &
1 &
1 &
1\\\hline
0 &
0 &
1 &
1 &
0 &
1\\\hline
\end{array}
\qedhere
\end{equation*}
\end{proof}
%
\begin{theorem}
Si \(𝖯\) et \(𝖰\) sont des propositions, la valeur de vérité de
\(𝖯⇒𝖰$ est $\mytrue$ si
et seulement si la valeur de vérité de \(𝖯\) est inférieure à celle de \(𝖰\).
En plus court, \(
(𝖯⇒𝖰)⟺(𝚟𝚕(𝖯)≤𝚟𝚕(𝖰))
\)
est une tautologie.
\end{theorem}
\begin{proof}
\begin{equation*}
\setlength{\extrarowheight}{0.25ex}
\mynewcolumntype{A}{1cm}
\mynewcolumntype{B}{2cm}
\mynewcolumntype{C}{3cm}
\begin{array}{|AAABC}
\hline
𝖯 &
𝖰 &
¬𝖯 &
(¬𝖯)\mathbin{∨}𝖰 &
𝚟𝚕(𝖯)≤𝚟𝚕(𝖰)\\\hline
1 &
1 &
0 &
1 &
1\\\hline
1 &
0 &
0 &
0 &
0\\\hline
0 &
1 &
1 &
1 &
1\\\hline
0 &
0 &
1 &
1 &
1\\\hline
\end{array}
\end{equation*}
\end{proof}
%
\begin{theorem}
Pour montrer qu'une implication est une tautologie,
il suffit de la démontrer dans le cas particulier où la prémisse est vraie.
\end{theorem}

\begin{theorem}
[Simplification de et]
Si \(𝖯\) et \(𝖰\) sont des propositions, $(𝖯\myand𝖰)⟹𝖯$ est une tautologie.
\end{theorem}
\begin{proof}
\begin{equation*}
\setlength{\extrarowheight}{0.25ex}
\mynewcolumntype{A}{1cm}
\mynewcolumntype{B}{1cm}
\mynewcolumntype{C}{2cm}
\mynewcolumntype{D}{3cm}
\begin{array}{|AABCD}
\hline
𝖯 &
𝖰 &
𝖯\mathbin{∧}𝖰 &
¬(𝖯\mathbin{∧}𝖰) &
¬(𝖯\mathbin{∧}𝖰)\mathbin{∨}𝖯\\\hline
1 &
1 &
1 &
0 &
1\\\hline
1 &
0 &
0 &
1 &
1\\\hline
0 &
1 &
0 &
1 &
1\\\hline
0 &
0 &
0 &
1 &
1\\\hline
\end{array}
\end{equation*}
\end{proof}
%
\begin{theorem}
[Syllogisme disjonctif]
Si \(𝖯\) et \(𝖰\) sont des propositions,
\(𝖯\myor𝖰)\myand\mynot𝖯⟹𝖰\) est une tautologie.
\end{theorem}
\begin{proof}
\begin{equation*}
\setlength{\extrarowheight}{0.25ex}
\mynewcolumntype{A}{1cm}
\mynewcolumntype{B}{2.5cm}
\begin{array}{|AAAAB|}
\hline
𝖯 &
𝖰 &
𝖯\mathbin{∨}𝖰 &
¬𝖯 &
(𝖯\mathbin{∨}𝖰)\mathbin{∨}¬𝖯\\\hline
1 &
1 &
1 &
0 &
0\\\hline
1 &
0 &
1 &
0 &
0\\\hline
0 &
1 &
1 &
1 &
1\\\hline
0 &
0 &
0 &
1 &
0\\\hline
\end{array}
\end{equation*}
Avec l'associativité de \(\myor\) vue ci-dessous, on a
\begin{align*}
((𝖯\mathbin{∨}𝖰)\mathbin{∧}¬𝖯⇒𝖰)
&{}⟺¬((𝖯\mathbin{∨}𝖰)\mathbin{∧}¬𝖯)\mathbin{∨}𝖰
\\&{}⟺
(¬(𝖯\mathbin{∨}𝖰)\mathbin{∨}𝖯)\mathbin{∨}𝖰
\\&{}⟺¬(𝖯\mathbin{∨}𝖰)\mathbin{∨}(𝖯\mathbin{∨}𝖰)
\end{align*}
\end{proof}
%
\begin{theorem}
[Tautologies]
Si \(𝖯\) et \(𝖰\) sont des propositions, les propositions complexes suivantes sont toutes des
tautologies :
\begin{itemize}
\item
\(
(𝖯⇔𝖰)⟺((𝖯⇒𝖰)\myand(𝖰⇒𝖯))
\)
\item
\(
(𝖯⇔𝖰)⟺(𝖰⇔𝖯)
\)
\item
\(
\mynot (𝖯⇔𝖰)⟺((\mynot 𝖯)⇔𝖰)
\)
\item
\(
\mynot (𝖯⇔𝖰)⟺(𝖯⇔(\mynot 𝖰))
\)
\item
\(
\mynot (𝖯⇒𝖰)⟺(𝖯\myand\mynot 𝖰)
\)
\end{itemize}
\end{theorem}
%
\begin{proof}
Par substitution,
\begin{align*}
((𝖯⇒𝖰)\myand(𝖰⇒𝖯))
&{}⟺((𝚟𝚕(𝖯){\leq}𝚟𝚕(𝖰))\myand(𝚟𝚕(𝖰){\leq}𝚟𝚕(𝖯)))
\\&{}⟺(𝚟𝚕(𝖯)=𝚟𝚕(𝖰))
\end{align*}
on termine en utilisant la commutativité par équivalence de \myand ainsi que la
caractérisation de l'équivalence par la valeur logique.

Pour la deuxième ligne,
\begin{equation*}
\setlength{\extrarowheight}{0.25ex}
\mynewcolumntype{A}{1cm}
\mynewcolumntype{B}{1cm}
\mynewcolumntype{C}{1.5cm}
\mynewcolumntype{D}{2cm}
\begin{array}{|ABABCDDD}
\hline
𝖯 &
¬𝖯 &
𝖰 &
¬𝖰 &
𝖯⇔𝖰 &
¬(𝖯⇔𝖰) &
(¬𝖯)⇔𝖰 &
𝖯⇔(¬𝖰)\\\hline
1 &
0 &
1 &
0 &
1 &
0 &
0 &
0\\\hline
1 &
0 &
0 &
1 &
0 &
1 &
1 &
1\\\hline
0 &
1 &
1 &
0 &
0 &
1 &
1 &
1\\\hline
0 &
1 &
0 &
1 &
1 &
0 &
0 &
0\\\hline
\end{array}
\end{equation*}
Pour finir, par substitutions successives :
\begin{align*}
¬(𝖯⇒𝖰)
&{}\overset{\hphantom{\text{subs}}}{⟺}¬(¬𝖯\mathbin{∨}𝖰
\\
&{}\overset{\text{subs}}{⟺}
(¬(¬𝖯)\mathbin{∧}¬𝖰)
\\&{}\overset{\text{subs}}{⟺}
(𝖯\mathbin{∧}¬𝖰)
\end{align*}
\end{proof}
%
\begin{theorem}
[Contraposition]
Si \(𝖯\) et \(𝖰\) sont des propositions, on a les tautologies :
\begin{gather*}
(𝖯⇔𝖰)⟺\bigl((\mynot 𝖯)⇔(\mynot 𝖰)\bigl)
\\
(𝖯⇒𝖰)⟺\bigl((\mynot 𝖰)⇒(\mynot 𝖯)\bigr)
\end{gather*}
\end{theorem}
\begin{proof}
\begin{equation*}
\setlength{\extrarowheight}{0.25ex}
\mynewcolumntype{A}{0.75cm}
\mynewcolumntype{B}{1.25cm}
\mynewcolumntype{C}{2cm}
\begin{array}{|AAAABCBC}
\hline
𝖯 &
¬𝖯 &
𝖰 &
¬𝖰 &
𝖯⟺ 𝖰 &
¬𝖯⟺ ¬𝖰 &
𝖯⇒𝖰 &
¬𝖰⇒¬𝖯\\\hline
1 &
0 &
1 &
0 &
1 &
1 &
1 &
1\\\hline
1 &
0 &
0 &
1 &
0 &
0 &
0 &
0\\\hline
0 &
1 &
1 &
0 &
0 &
0 &
1 &
1\\\hline
0 &
1 &
0 &
1 &
1 &
1 &
1 &
1\\\hline
\end{array}
\end{equation*}
\end{proof}
%
En fait, il y a 16 combinaisons possibles pour des opérations binaires :
\begin{equation*}
\setlength{\extrarowheight}{0.25ex}
\mynewcolumntype{A}{1cm}
\newcolumntype{A}{>{\(}wc{0.45cm}<{\)}|}
\def\mycell#1{\)\rotatebox{90}{\(#1\)}\(}
\begin{array}{|*{18}{A}}
\hline
𝖯 &
𝖰 &
1 &
2 &
3 &
4 &
5 &
5 &
7 &
8 &
9 &
10 &
11 &
12 &
13 &
14 &
15 &
16\\\hline
1 &
1 &
1 &
1 &
1 &
1 &
1 &
1 &
1 &
1 &
0 &
0 &
0 &
0 &
0 &
0 &
0 &
0\\\hline
1 &
0 &
1 &
1 &
1 &
1 &
0 &
0 &
0 &
0 &
1 &
1 &
1 &
1 &
0 &
0 &
0 &
0\\\hline
0 &
1 &
1 &
1 &
0 &
0 &
1 &
1 &
0 &
0 &
1 &
1 &
0 &
0 &
1 &
1 &
0 &
0\\\hline
0 &
0 &
1 &
0 &
1 &
0 &
1 &
0 &
1 &
0 &
1 &
0 &
1 &
0 &
1 &
0 &
1 &
0\\\hline
~
 &
~
 &
\mycell{𝖯\mathbin{∨}¬𝖯} &
\mycell{𝖯\mathbin{∨}𝖰} &
\mycell{𝖯\mathbin{∨}¬𝖰} &
\mycell{𝖯} &
\mycell{¬𝖯\mathbin{∨}𝖰} &
\mycell{𝖰} &
\mycell{¬𝖯⇔¬𝖰} &
\mycell{𝖯\mathbin{∧}𝖰} &
\mycell{¬(𝖯\mathbin{∧}𝖰)} &
\mycell{¬(𝖯⇔𝖰)} &
\mycell{¬𝖰} &
\mycell{¬(𝖯⇒𝖰)} &
\mycell{¬𝖯} &
\mycell{¬(𝖰⇒𝖯)} &
\mycell{¬(𝖯\mathbin{∨}𝖰)} &
%\mycell{¬(𝖰⇒𝖯)} &
\mycell{𝖯\mathbin{∧}¬𝖯}\\\hline
\end{array}
\end{equation*}
\begin{remark}%
On peut introduit l'opérateur binaire \(\myxor\) aussi noté \(\mathbin{∨∨}\)
en sorte que \(¬(𝖯⇔𝖰)\) et
\(𝖯\mathbin{∨∨}𝖰\) sont équivalents.
\end{remark}

\begin{theorem}
[\(⇔\) est une relation d'équivalence]

L'équivalence des propositions définit une relation d'équivalence entre les propositions.
\end{theorem}
\begin{proof}
\par\noindent
\begin{itemize}
\item
Réflexivité :
\begin{align*}
𝖯⇔𝖯
&\mybydef{=}
(𝖯\myand𝖯)\myor\bigl((\mynot 𝖯)\myand(\mynot 𝖯)\bigr)
\\&\overset{\text{subs}}{⟺}
\bigl(𝖯\myor(\mynot 𝖯)\bigr)
\end{align*}
\item
Symétrie : c'est la première tautologie du théorème précédent.
\item
Transitivité :
\begin{equation*}
\setlength{\extrarowheight}{0.25ex}
\mynewcolumntype{A}{0.5cm}
\mynewcolumntype{B}{1.25cm}
\mynewcolumntype{C}{3.75cm}
\begin{array}{|AAABBCB}
\hline
𝖯 &
𝖰 &
𝖱 &
𝖯⇔𝖰 &
𝖰⇔𝖱 &
(𝖯⇔𝖰)\mathbin{∧}(𝖰⇔𝖱) &
𝖯⇔𝖱\\\hline
1 &
1 &
1 &
1 &
1 &
1 &
1\\\hline
1 &
1 &
0 &
1 &
0 &
0 &
0\\\hline
1 &
0 &
1 &
0 &
0 &
0 &
0\\\hline
1 &
0 &
0 &
0 &
1 &
0 &
0\\\hline
0 &
1 &
1 &
0 &
1 &
0 &
0\\\hline
0 &
1 &
0 &
0 &
0 &
0 &
0\\\hline
0 &
0 &
1 &
1 &
0 &
0 &
0\\\hline
0 &
0 &
0 &
1 &
1 &
1 &
1\\\hline
\end{array}
\qedhere
\end{equation*}
\end{itemize}
\end{proof}
\begin{remark}
On peut écrire les suites d'équivalence en ligne.
\end{remark}
%
\begin{theorem}
[Associativités]
Soient \(𝖯\), \(𝖰\) et \(𝖱\) des propositions, on a les tautologies :
\begin{itemize}
\item
\(
(𝖯\myand(𝖰\myand𝖱))⟺((𝖯\myand𝖰)\myand𝖱)
\)
\item
\(
(𝖯\myor(𝖰\myor𝖱))⟺((𝖯\myor𝖰)\myor𝖱)
\)
\item
\(
(𝖯⇔(𝖰⇔𝖱))⟺((𝖯⇔𝖰)⇔𝖱)
\)
\item
\(
(𝖯⇒(𝖰⇒𝖱))⟹((𝖯⇒𝖰)⇒𝖱)
\)
\end{itemize}
\end{theorem}

\begin{attention}
\(⇒\) n'est pas associatifs !
\end{attention}

\begin{remark}
(moyen presque mnémotechnique) commutatif donne associatif.
\end{remark}

\begin{proof}
\begin{equation*}
\setlength{\extrarowheight}{0.25ex}
\mynewcolumntype{A}{0.5cm}
\mynewcolumntype{B}{1cm}
\mynewcolumntype{C}{2.25cm}
\begin{array}{|AAABCBC}
\hline
𝖯 &
𝖰 &
𝖱 &
𝖰\mathbin{∧}𝖱 &
𝖯\mathbin{∧}(𝖰\mathbin{∧}𝖱) &
𝖯\mathbin{∧}𝖰 &
(𝖯\mathbin{∧}𝖰)\mathbin{∧}𝖱\\\hline
1 &
1 &
1 &
1 &
1 &
1 &
1\\\hline
1 &
1 &
0 &
0 &
0 &
1 &
0\\\hline
1 &
0 &
1 &
0 &
0 &
0 &
0\\\hline
1 &
0 &
0 &
0 &
0 &
0 &
0\\\hline
0 &
1 &
1 &
1 &
0 &
0 &
0\\\hline
0 &
1 &
0 &
0 &
0 &
0 &
0\\\hline
0 &
0 &
1 &
0 &
0 &
0 &
0\\\hline
0 &
0 &
0 &
0 &
0 &
0 &
0\\\hline
\end{array}
\end{equation*}
On en déduit :
\begin{equation*}
\begin{split}
¬(𝖯\mathbin{∨}(𝖰\mathbin{∨}𝖱))⟺¬
𝖯\mathbin{∧}¬
(𝖰\mathbin{∨}𝖱)⟺¬
𝖯\mathbin{∧}(¬𝖰\mathbin{∧}¬
𝖱)⟺...
\\
...⟺(¬𝖯\mathbin{∧}¬𝖰)\mathbin{∧}¬
𝖱⟺¬
(𝖯\mathbin{∨}𝖰)\mathbin{∧}¬
𝖱⟺¬
((𝖯\mathbin{∨}𝖰)\mathbin{∨}𝖱)
\end{split}
\end{equation*}
Pour les deux dernières lignes :
\begin{equation*}
\setlength{\extrarowheight}{0.25ex}
\newcolumntype{X}{>{\(}wc{1.25cm}<{\)}|>{\(}wc{2.5cm}<{\)}|}
\begin{array}{|
*{3}{>{\(}wc{0.5cm}<{\)}|}
*{2}{X}
}
\hline
𝖯 &
𝖰 &
𝖱 &
𝖰⇔𝖱 &
𝖯⇔(𝖰⇔𝖱) &
𝖯⇔𝖰 &
(𝖯⇔𝖰)⇔𝖱\\\hline
1 &
1 &
1 &
1 &
1 &
1 &
1\\\hline
1 &
1 &
0 &
0 &
0 &
1 &
0\\\hline
1 &
0 &
1 &
0 &
0 &
0 &
0\\\hline
1 &
0 &
0 &
1 &
1 &
0 &
1\\\hline
0 &
1 &
1 &
1 &
0 &
0 &
0\\\hline
0 &
1 &
0 &
0 &
1 &
0 &
1\\\hline
0 &
0 &
1 &
0 &
1 &
1 &
1\\\hline
0 &
0 &
0 &
1 &
0 &
1 &
0\\\hline
\end{array}
\end{equation*}
%
\begin{equation*}
\setlength{\extrarowheight}{0.25ex}
\newcolumntype{X}{>{\(}wc{1.5cm}<{\)}|>{\(}wc{2.5cm}<{\)}|}
\begin{array}{|
*{3}{>{\(}wc{0.3cm}<{\)}|}
*{2}{X}
}
\hline
𝖯 &
𝖰 &
𝖱 &
𝖰⇒𝖱 &
𝖯⇒(𝖰⇒𝖱) &
𝖯⇒𝖰 &
(𝖯⇒𝖰)⇒𝖱\\\hline
1 &
1 &
1 &
1 &
1 &
1 &
1\\\hline
1 &
1 &
0 &
0 &
0 &
1 &
0\\\hline
1 &
0 &
1 &
0 &
0 &
0 &
1\\\hline
1 &
0 &
0 &
0 &
0 &
0 &
1\\\hline
0 &
1 &
1 &
1 &
0 &
1 &
1\\\hline
0 &
1 &
0 &
0 &
0 &
1 &
0\\\hline
0 &
0 &
1 &
0 &
0 &
1 &
1\\\hline
0 &
0 &
0 &
0 &
0 &
1 &
0\\\hline
\end{array}
\end{equation*}
\end{proof}
%
\begin{theorem}
[Distributivités \(\myand\) \(\myor\)]
Soient \(𝖯\), \(𝖰\) et \(𝖱\) des propositions, on a les tautologies :
\begin{itemize}
\item
\(
(𝖯\myor(𝖰\myand𝖱))⟺((𝖯\myor𝖰)\myand(𝖯\myor𝖱))
\)
\item
\(
(𝖯\myand(𝖰\myor𝖱))⟺((𝖯\myand𝖰)\myor(𝖯\myand𝖱))
\)
\end{itemize}
\end{theorem}
%
\begin{proof}
\begin{equation*}
\setlength{\extrarowheight}{0.25ex}
\mynewcolumntype{A}{0.5cm}
\mynewcolumntype{B}{1cm}
\mynewcolumntype{C}{3.25cm}
\begin{array}{|AAABCBBC}
\hline
𝖯 &
𝖰 &
𝖱 &
𝖰\mathbin{∧}𝖱 &
𝖯\mathbin{∨}(𝖰\mathbin{∧}𝖱) &
𝖯\mathbin{∨}𝖰 &
𝖯\mathbin{∨}𝖱 &
(𝖯\mathbin{∨}𝖰)\mathbin{∧}(𝖯\mathbin{∨}𝖱)\\\hline
1 &
1 &
1 &
1 &
1 &
1 &
1 &
1\\\hline
1 &
1 &
0 &
0 &
1 &
1 &
1 &
1\\\hline
1 &
0 &
1 &
0 &
1 &
1 &
1 &
1\\\hline
1 &
0 &
0 &
0 &
1 &
1 &
1 &
1\\\hline
0 &
1 &
1 &
1 &
1 &
1 &
1 &
1\\\hline
0 &
1 &
0 &
0 &
0 &
1 &
0 &
0\\\hline
0 &
0 &
1 &
0 &
0 &
0 &
1 &
0\\\hline
0 &
0 &
0 &
0 &
0 &
0 &
0 &
0\\\hline
\end{array}
\end{equation*}

On pourrait faire le même tableau en échangeant les deux opérateurs, on peut aussi utiliser la négation :

\begin{align*}
¬(𝖯\mathbin{∧}(𝖰\mathbin{∨}𝖱))
&{}⟺
¬𝖯\mathbin{∨}¬(𝖰\mathbin{∨}𝖱)
\\&{}⟺
¬𝖯\mathbin{∨}(¬𝖰\mathbin{∧}¬𝖱)
\\&{}⟺
(¬𝖯\mathbin{∨}¬𝖰)\mathbin{∧}(¬𝖯\mathbin{∨}¬𝖱)
\\&{}⟺
¬(𝖯\mathbin{∧}𝖰)\mathbin{∧}¬(𝖯\mathbin{∧}𝖱)
\\&{}⟺
¬\bigl((𝖯\mathbin{∧}𝖰)\mathbin{∨}(𝖯\mathbin{∧}𝖱)\bigr)
\qedhere
\end{align*}
\end{proof}

\begin{theorem}
[Distributivités à gauche de $⇒$]
Soient \(𝖯\), \(𝖰\) et \(𝖱\) des propositions, on a les tautologies :
\begin{itemize}
\item
\(
\bigl(𝖯⇒(𝖰\myand𝖱)\bigr)⟺\bigl((𝖯⇒𝖰)\myand(𝖯⇒𝖱)\bigr)
\)
\item
\(
\bigl(𝖯⇒(𝖰\myor𝖱)\bigr)⟺\bigl((𝖯⇒𝖰)\myor(𝖯⇒𝖱)\bigr)
\)
\end{itemize}
\end{theorem}
\begin{proof}
Par la définition de l'implication et le théorème précédent,
\begin{align*}
¬\bigl(𝖯\mathbin{∨}(𝖰\mathbin{∧}𝖱)\bigr)
&{}⟺
(¬𝖯\mathbin{∨}𝖰)\mathbin{∧}(¬𝖯\mathbin{∨}𝖱)
\\
¬𝖯\mathbin{∨}(𝖰\mathbin{∨}𝖱)
&{}⟺
((¬𝖯\mathbin{∨}𝖰)\mathbin{∨}(¬𝖯\mathbin{∨}𝖱))
\end{align*}
On a terminé en utilisant associativité et commutativité de \(\mathbin{∨}\).
\end{proof}
%
\begin{theorem}
[Séparation]
Soient \(𝖯\), \(𝖰\) et \(𝖱\) des propositions, on a les tautologies :

\begin{itemize}
\item
\(
((𝖯\myand𝖰)⇒𝖱)⟺(𝖯⇒(𝖰⇒𝖱))
\)
\item
\(
((𝖯\myor𝖰)⇒𝖱)⟺((𝖯⇒𝖱)\myand(𝖰⇒𝖱))
\)
\end{itemize}
\end{theorem}
\begin{proof}
Par la définition de l'implication et l'associativité de \myor,
\begin{align*}
((𝖯\mathbin{∧}𝖰)⇒𝖱)
&{}⟺
¬(𝖯\mathbin{∧}𝖰)\mathbin{∨}𝖱
\\&{}⟺
(¬𝖯\mathbin{∨}¬𝖰)\mathbin{∨}𝖱
\\&{}⟺
¬𝖯\mathbin{∨}(¬𝖰\mathbin{∨}𝖱)
\\&{}⟺
¬𝖯\mathbin{∨}(𝖰⇒𝖱)
\\&{}⟺
(𝖯⇒(𝖰⇒𝖱))
\end{align*}
et par la définition de l'implication et le théorème \ref{seq:refTheorem16},
\begin{align*}
((𝖯\mathbin{∨}𝖰)⇒𝖱)
&{}⟺
¬(𝖯\mathbin{∨}𝖰)\mathbin{∨}𝖱
\\&{}⟺
(¬𝖯\mathbin{∧}¬𝖰)\mathbin{∨}𝖱
\\&{}⟺
(¬𝖯\mathbin{∨}𝖱)\mathbin{∧}(¬𝖰\mathbin{∨}𝖱)
\\&{}⟺
(𝖯⇒𝖱)\mathbin{∧}(𝖰⇒𝖱)
\qedhere
\end{align*}
\end{proof}
%
\begin{theorem}
Soient \(𝖯\), \(𝖰\) et \(𝖱\) des propositions.
\begin{itemize}
\item
\(
\bigl(𝖯⇔(𝖰\myand𝖱)\bigr)⟸\bigl((𝖯⇔𝖰)\myand(𝖯⇔𝖱)\bigr)
\)
\item
\(
\bigl(𝖯⇔(𝖰\myor𝖱)\bigr)⟹\bigl((𝖯⇔𝖰)\myor(𝖯⇔𝖱)\bigr)
\)
\end{itemize}
\end{theorem}
\begin{remark}
Ce ne sont pas des équivalences !
\end{remark}
\begin{proof}
Pour la première implication, on compare les colonnes en gras :
\begin{equation*}
\setlength\extrarowheight{0.25ex}
\mynewcolumntype{A}{0.5cm}
\mynewcolumntype{B}{1.2cm}
\mynewcolumntype{C}{2.25cm}
\mynewcolumntype{D}{3.5cm}
\begin{array}{|AAABCBBD}
\hline
𝖯 &
𝖰 &
𝖱 &
𝖰\mathbin{∧}𝖱 &
𝖯⇔(𝖰\mathbin{∧}𝖱) &
𝖯⇔𝖰 &
𝖯⇔𝖱 &
(𝖯⇔𝖰)\mathbin{∧}(𝖯⇔𝖱)\\\hline
1 &
1 &
1 &
1 &
1 &
1 &
1 &
1\\\hline
1 &
1 &
0 &
0 &
0 &
1 &
0 &
0\\\hline
1 &
0 &
1 &
0 &
0 &
0 &
1 &
0\\\hline
1 &
0 &
0 &
0 &
0 &
0 &
0 &
0\\\hline
0 &
1 &
1 &
1 &
0 &
0 &
0 &
0\\\hline
0 &
1 &
0 &
0 &
1 &
0 &
1 &
0\\\hline
0 &
0 &
1 &
0 &
1 &
1 &
0 &
0\\\hline
0 &
0 &
0 &
0 &
1 &
1 &
1 &
1\\\hline
\end{array}
\end{equation*}

Les cases grises montrent qu'il n'y a pas d'équivalences.
Pour la deuxième implication, sont équivalentes :
\begin{itemize}
\item
\(
(𝖯⇔(𝖰\mathbin{∧}𝖱))⟸((𝖯⇔𝖰)\mathbin{∧}(𝖯⇔𝖱))
\)
\item
\(
¬((𝖯⇔𝖰)\mathbin{∧}(𝖯⇔𝖱))⟹¬(𝖯⇔(𝖰\mathbin{∧}𝖱))
\)
\item
\(
(¬(𝖯⇔𝖰)\mathbin{∨}¬(𝖯⇔𝖱))⟹(¬𝖯⇔¬(𝖰\mathbin{∧}𝖱))
\)
\item
\(
((¬𝖯⇔¬𝖰)\mathbin{∨}(¬𝖯⇔¬𝖱))⟹(¬𝖯⇔(¬𝖰\mathbin{∨}¬𝖱))
\)
\end{itemize}
Il suffit de substituer à chaque proposition sa négation.
\end{proof}
%
\begin{theorem}
[Ordre]
\(⇒\) est une relation d'ordre à équivalence près.
\end{theorem}
\begin{proof}
À équivalence près signifie que c'est une relation d'ordre sur les classes d'équivalence.

Soient \(𝖯\), \(𝖰\) et \(𝖱\) des propositions.
\begin{itemize}
\item
$⇒$ est réflexive :
Par définition, $𝖯⇒𝖯$ est la tautologie \((\mynot 𝖯)\myor𝖯\).
\item
\(⇒\) est anti-symétrique à équivalence près :
on a déjà vu la tautologie
\begin{equation*}
(𝖯⇒𝖰\myand𝖯\Leftarrow𝖰)⟺(𝖯⇔𝖰)
\end{equation*}
\item
$⇒$ est transitive :
\begin{equation*}
\setlength\extrarowheight{0.25ex}
\mynewcolumntype{A}{0.75cm}
\mynewcolumntype{B}{1.25cm}
\mynewcolumntype{C}{3.5cm}
\begin{array}{|AAABBCB}
\hline
𝖯 &
𝖰 &
𝖱 &
𝖯⇒𝖰 &
𝖰⇒𝖱 &
(𝖯⇒𝖰)\mathbin{∧}(𝖰⇒𝖱) &
𝖯⇒𝖱\\\hline
1 &
1 &
1 &
1 &
1 &
1 &
1\\\hline
1 &
1 &
0 &
1 &
0 &
0 &
0\\\hline
1 &
0 &
1 &
0 &
1 &
0 &
1\\\hline
1 &
0 &
0 &
0 &
1 &
0 &
0\\\hline
0 &
1 &
1 &
1 &
1 &
1 &
1\\\hline
0 &
1 &
0 &
1 &
0 &
0 &
1\\\hline
0 &
0 &
1 &
1 &
1 &
0 &
1\\\hline
0 &
0 &
0 &
1 &
1 &
0 &
1\\\hline
\end{array}
\end{equation*}
\end{itemize}
\end{proof}

\section{Quantificateurs}
Cette partie repose sur la notion d'ensembles détaillée au chapitre suivant.

\begin{axiom}
[Quantificateurs existentiel et universel]
Pour tout ensemble \(𝐸\), les propositions suivantes sont des tautologies
\begin{itemize}
\item
\(
𝐸≠∅⟺∃𝑥(𝑥∈𝐸)
\)
\item
\(
𝐸=∅⟺∀𝑥(𝑥∉𝐸)
\)
\end{itemize}
\end{axiom}
\begin{remark}
Ce ne sont pas des définitions car la notion d'ensemble est pré-requise, qui utilise elle-même les quantificateurs.
\end{remark}

\begin{definition}
[Quantificateurs]
Étant donnés un ensemble \(𝛺\) et une proposition dont l'écriture dépend formellement d'un paramètre \(𝑥\)
notée $𝖯(𝑥)$,
\begin{itemize}
\item
\(\displaystyle
∃𝑥∈𝛺\;(𝖯(𝑥))
\overset{\text{déf}}{⟺}
∃𝑥\;(𝑥∈𝛺\myand𝖯(𝑥))
\)
\item
\(\displaystyle
∀𝑥∈𝛺\;(𝖯(𝑥))
\overset{\text{déf}}{⟺}
\mynot (∃𝑥∈𝛺\;(\mynot 𝖯(𝑥)))
\)
\end{itemize}
\end{definition}
\begin{communication}
 \(∃𝑥∈𝛺\;...\) est lu «~il existe 𝑥 élément de (ou dans)
\(𝛺\) tel que ...~».
\(∀𝑥∈𝛺\;...\) est lu «quel que soit \(𝑥\) élément de (ou dans) \(𝛺\) tel que ...~»
ou «~pour tout \(𝑥\) élément de (ou dans) \(𝛺\) tel que ...~».
\end{communication}

\begin{remark}
En fait ce n'est pas une seule définition mais un gabarit de définitions.
Cela permet de donner autant de définitions que de \(𝛺\) et $𝖯(𝚡)$
possibles.
\end{remark}

\begin{theorem}
Étant donnés un ensemble \(𝛺\) et une proposition dont l'écriture dépend formellement d'un paramètre \(𝑥\)
notée \(𝖯(𝑥)\), on a les tautologies
\begin{itemize}
\item
\(\displaystyle
∃𝑥∈𝛺\;𝖯(𝑥)⟺\{𝑧∈𝛺\;/\;𝖯(𝑧)\}≠∅⟹𝛺≠∅
\)
\item
\(\displaystyle
∀𝑥∈𝛺\;(𝖯(𝑥))⟺\{𝑧∈𝛺\;/\;¬𝖯(𝑧)\}=∅⟺\{𝑧∈𝛺\;/\;𝖯(𝑧)\}=𝛺
\)
\end{itemize}
\end{theorem}
\begin{proof}
Avec le quantificateur existentiel, c'est une application directe de l'axiome ci-dessus sachant que
\begin{equation*}
𝑥∈\{𝑧∈𝛺\;/\;𝖯(𝑧)\}
\overset{\text{déf}}{⟺}
𝑥∈𝛺\myand𝖯(𝑥)
\end{equation*}
Avec le quantificateur universel, on utilise les négations. On a
\begin{equation*}
∃𝑥∈𝛺\;(¬𝖯(𝑥))⟺\{𝑧∈𝛺\;/\;¬𝖯(𝑧)\}≠∅
\end{equation*}
d'où
\begin{equation*}
∀𝑥∈𝛺\;(𝖯(𝑥))⟺¬(∃𝑥∈𝛺\;(¬𝖯(𝑥)))⟺\{𝑧∈𝛺\;/\;¬𝖯(𝑧)\}=∅
\end{equation*}
\end{proof}

\begin{theorem}
Étant donnés un ensemble \(𝛺\) et
une proposition dont l'écriture dépend formellement d'un paramètre notée \(𝖯\)(𝚡).
Les ensembles
\(\{𝑧∈𝛺\;/\;𝖯(𝑧)\}\) et \(\{𝑧∈𝛺\;/\;¬𝖯(𝑧)\}\)
forment une partition de \(𝛺\), en particulier
\begin{equation*}
\{𝑧∈𝛺\;/\;𝖯(𝑧)\}=𝛺⟺\{𝑧∈𝛺\;/\;¬𝖯(𝑧)\}=∅
\end{equation*}
\end{theorem}
\begin{proof}
On a pour tout \(𝑥\),
\begin{equation*}
𝑥∈\{𝑧∈𝛺\;/\;𝖯(𝑧)\}∩\{𝑧∈𝛺\;/\;¬𝖯(𝑧)\}
⟺𝑥∈𝛺\myand𝖯(𝑥)\myand¬𝖯(𝑥)
⟺𝑥∈∅
\end{equation*}
et
\begin{equation*}
𝑥∈\{𝑧∈𝛺\;/\;¬𝖯(𝑧)\}∪\{𝑧∈𝛺\;/\;𝖯(𝑧)\}
⟺𝑥∈𝛺\myand(𝖯(𝑥)\myor¬𝖯(𝑥))
⟺𝑥∈𝛺
\end{equation*}
\end{proof}

\begin{theorem}
[Distributivité]
Étant donnés un ensemble \(𝛺\), une proposition dont l'écriture dépend formellement d'un paramètre 𝚡
notée \(𝖯\)(𝚡), et une proposition \(𝖰\), on a

\begin{equation*}
(∃𝑥∈𝛺\;𝖯(𝑥))\myand𝖰⟺∃𝑥∈𝛺\;(𝖯(𝑥)\myand𝖰)
\end{equation*}
et par contraposition et négation
\begin{equation*}(∀𝑥∈𝛺\;𝖯(𝑥))\myor𝖰⟺∀𝑥∈𝛺\;(𝖯(𝑥)\myor𝖰)
\end{equation*}
De plus,
\begin{equation*}
(∃𝑥∈𝛺\;𝖯(𝑥))\myor𝖰⟸∃𝑥∈𝛺\;(𝖯(𝑥)\myor𝖰)
\end{equation*}
et par contraposition et négation
\begin{equation*}
(∀𝑥∈𝛺\;𝖯(𝑥))\myand𝖰⟹∀𝑥∈𝛺\;(𝖯(𝑥)\myand𝖰)
\end{equation*}
Si en plus \(𝛺\) n'est pas vide, les deux dernières implications sont des équivalence.
\end{theorem}
\begin{proof}
Soit $𝑥_0$ tel que $𝖯(𝑥_0)$.
Si \(𝖰\) est vrai alors la proposition
\(𝖯(𝑥_0)\myand𝖰$ aussi d'où $∃𝑥∈𝛺 \;(𝖯(𝑥)\myand𝖰)$.
Inversement, soit $𝑥_0$ tel que \(𝖯(𝑥_0)\myand𝖰$ est vrai.
Alors $𝖯(𝑥_0)$ est vrai donc $∃𝑥∈𝛺 \;𝖯(𝑥)$ l'est aussi aussi.

Soit $𝑥_0$ tel que \(𝖯(𝑥_0)\myor𝖰$ . On a
\begin{equation*}
𝖯(𝑥_0)⟹(∃𝑥∈𝛺\;𝖯(𝑥))⟹(∃𝑥∈𝛺\;𝖯(𝑥))\myor𝖰
\end{equation*}
\begin{equation*}
𝖰⟹(∃𝑥∈𝛺\;𝖯(𝑥))\myor𝖰\end{equation*}
Par le théorème \ref{seq:refTheorem18}:
\begin{equation*}
𝖯(𝑥_0)\myor𝖰⟹(∃𝑥∈𝛺 ,\;𝖯(𝑥))\myor𝖰
\end{equation*}
Inversement, supposons $∃𝑥∈𝛺\;𝖯(𝑥)$ vrai,
soit donc $𝑥_0$ tel que $𝖯(𝑥_0)$.
On a \(𝖯(𝑥_0)\myor𝖰\) donc \(∃𝑥∈𝛺 \;(𝖯(𝑥)\myor𝖰)\) est vrai.
Supposons maintenant \(𝖰\) vrai et \(𝛺\) non vide. Soit
\(𝑥_0$ un élément de \(𝛺\), on a encore
\(𝖯(𝑥_0)\myor𝖰\) d'où le résultat.
\end{proof}
%
\section{Types de raisonnements}
\subsection{Modus ponens}
\begin{equation*}
(𝖯\myand(𝖯⇒𝖰))⟹𝖰
\end{equation*}
\subsection{Contraposition ou modus tollens}
\begin{equation*}
(¬𝖰⇒¬𝖯)⟹(𝖯⇒𝖰)
\end{equation*}
\subsection{Par l'absurde}
\begin{equation*}
𝖯\myand((𝖯\myand¬𝖰)⇒\myfalse)⟹𝖰
\end{equation*}
\subsection{Séparation}
Le principe repose sur le théorème \ref{seq:refTheorem18} :
\begin{equation*}
((𝖯\myand𝖰)⇒𝖱)⟺(𝖯⇒(𝖰⇒𝖱))
\end{equation*}
\begin{equation*}
((𝖯\myor𝖰)⇒𝖱)⟺((𝖯⇒𝖱)\myand(𝖰⇒𝖱))
\end{equation*}
\subsection{Contre-exemple}
Sachant que
\begin{equation*}
¬(∀𝑥∈𝛺,\;𝖯(𝑥))⟺∃𝑥∈𝛺,\;¬𝖯(𝑥)
\end{equation*}
pour montrer que \(∀𝑥∈𝛺,\;𝖯(𝑥)\) est fausse, il suffit de trouver
\(𝑥_0∈𝛺\) tel que \(𝖯(𝑥_0)\)
est faux. \(𝑥_0\) est le contre-exemple.
%
\subsection{Disjonction des cas}
Le principe repose sur
\begin{equation*}
∀𝑥∈\mathop{∪}\limits_{𝑖∈𝐼}𝛺_{𝑖}\ 𝖯(𝑥))⟺\mathop{∧}\limits_{𝑖∈𝐼}(∀𝑥∈𝛺_{𝑖}\;𝖯(𝑥))
\end{equation*}
La démonstration est adaptée à l'ensemble $𝛺_i$ étudié.

\subsection{Syllogisme}
\begin{equation*}
∀𝑥\;(𝑥∈𝛺\myand𝛺{\subset}𝛺'⟹𝑥∈𝛺')
\end{equation*}
\subsection{Syllogisme disjonctif}
\begin{equation*}
((𝖯\myor𝖰)\myand¬𝖯)⟹𝖰
\end{equation*}
\subsection{Déduction}
À compléter...

\subsection{Récurrence}
À compléter...

\subsection{Descente}
À compléter...

\subsection{Abduction}
\begin{remark}
Ce n'est pas du tout une règle d'inférence logique mais c'est fondamental.
\end{remark}
\begin{equation*}
((𝖯⇒𝖰)\myand𝖰)⟹𝖯
\end{equation*}
L'argument pour justifier un tel raisonnement est le suivant :
si \(𝖰\) est une proposition complexe et s'il y a très peu de prémisses simples qui l'impliquent, il y a de forte chance que \(𝖯\) soit vrai.
\subsection{Induction}
\begin{remark}
Ce n'est pas du tout une règle d'inférence logique.
\end{remark}
Passer du cas particulier au cas général : voir conjecturer...
\begin{equation*}
∃𝑥∈𝛺,\;𝖯(𝑥)⟹∀𝑥∈𝛺\;𝖯(𝑥)
\end{equation*}
Extension :
\begin{equation*}
∃𝛺'⊂𝛺, \; 𝛺'≠∅,\;\myand ∀𝑥∈𝛺'\;𝖯(𝑥)⟹∀𝑥∈𝛺\;𝖯(𝑥)
\end{equation*}
Cette proposition est vrai si \(𝛺'=𝛺\), pas toujours dans le cas contraire.
Dans le doute, on peut se baser sur ce schéma si \(𝛺'\) est suffisamment grand.
Cela peut être une aide pour trouver un raisonnement mathématique rigoureux,
ou bien une aide à la prise de décision sans raisonnement mathématique rigoureux.
Ce dernier cas intervient de manière essentielle dans la construction
du raisonnement chez l'enfant et l'adolescent.
