% !TEX encoding = UTF-8
% !TEX program = xelatex
% !TEX root =../Main/main.tex

Dans la suite \(𝐺\), \(𝐹\) et \(𝐸\) désignent des ensembles.
%
\section{Dénombrement}
Cardinal d’un ensemble fini.
\subsection{Ensembles finis ou infinis}
\subsubsection{Définitions}
\begin{definition}
\par\noindent
\begin{enumerate}
\item Un ensemble \(𝐸\) est \mykeyword{infini} signifie qu'il existe une injection de \(𝐸\) dans une
partie stricte de \(𝐸\).
\item Un ensemble \(𝐸\) est \mykeyword{fini} signifie qu'il n'est pas infini.
\end{enumerate}
\end{definition}
\begin{theorem}
\par\noindent
\begin{enumerate}
\item Un ensemble dont une partie est infinie est infini.
\item Toute partie d'un ensemble fini est finie.
\end{enumerate}
\end{theorem}
\begin{proof}
\par\noindent
\begin{enumerate}
\item Si \(𝐸⊂𝐺\) et \(𝜑~:𝐸→𝐸\) est une injection. L'application
\begin{equation*}
\begin{matrix}
𝜓:&𝐺&⟶&𝐺
\\
&𝑥&⟼&\begin{cases}
𝑥\text{ si }𝑥∈𝐺∖𝐸
\\
𝜑(𝑥)\text{ sinon}
\end{cases}
\end{matrix}
\end{equation*}
est une injection dont l'image est \(\operatorname{Im}𝜑∪𝐺∖𝐸\), qui est une partie stricte de \(𝐺\) dès que \(\operatorname{Im}𝜑\) est
une partie stricte de \(𝐺\).
\item Par contraposition de ce qui précède.
\end{enumerate}
\end{proof}
\begin{theorem}
Un ensemble infini privé d'un seul élément est encore infini.
\end{theorem}
\begin{proof}
Soit \(𝐹⊊𝐸\), \(𝜑\) une injection de \(𝐸\) dans \(𝐹\) et \(𝑥∈𝐸\). Soit 𝜏 la transposition qui échange \(𝑥\) et
\(𝜑(𝑥)\). \(𝜏(𝐹)\) est une partie stricte de \(𝐸\) car elle ne contient pas \(𝜏(𝐸∖𝐹)\). \(𝜏∘𝜑\) est une
injection de \(𝐸\) dans \(𝜏(𝐹)\) telle que \(𝜏∘𝜑(𝑥)=𝑥\). Elle induit une injection de \(𝐸∖\{𝑥\}\) dans
\(𝜏(𝐹)∖\{𝑥\}\) où \(𝜏(𝐹)∖\{𝑥\}⊊𝐸∖\{𝑥\}\). Donc \(𝐸∖\{𝑥\}\) est infini.
\end{proof}
\subsubsection[Intervalles entiers]{Intervalles entiers}
\begin{theorem}
Soit \(𝑛∈ℕ\), \(⟦1;𝑛⟧\) est un ensemble fini.
\end{theorem}
\begin{proof}
Soit \(𝐸\) l'ensemble des entiers tels que \(⟦1;𝑛⟧\) est infini.
Supposons par l'absurde que \(𝐸\) n'est pas vide. \(𝐸\) ne contient pas \(0\) car
\(⟦1;0⟧=\mathsf{ ∅}\) et il n'existe pas de partie stricte de \(⟦1;0⟧\) donc il n'existe pas d'injection d'une
partie stricte de \(⟦1;0⟧\) dans \(⟦1;0⟧\). Le minimum de \(𝐸\), non nul, est noté \(𝑚\). Comme \(⟦1;𝑚⟧\) est
infini, et \(⟦1;𝑚⟧=⟦1;𝑚-1⟧∪\{𝑚\}\), par le lemme précédent, \ \(⟦1;𝑚-1⟧\) est infini, ce qui contredit la
minimalité de \(𝑚\).
\end{proof}
\begin{theorem}
Soient \(𝑛,𝑚∈ℕ\).
\begin{enumerate}
\item S'il existe une injection de \(⟦1;𝑛⟧\) dans \(⟦1;𝑚⟧\) alors \(𝑛⩽𝑚\).
\item S'il existe une surjection de \(⟦1;𝑛⟧\) sur \(⟦1;𝑚⟧\) alors \(𝑛⩾𝑚\).
\end{enumerate}
\end{theorem}
\begin{proof}
\par\noindent
\begin{enumerate}
\item Par récurrence sur \(𝑛\).
\begin{itemize}
\item
Initialisation.
Pour \(𝑛=0\), comme \(𝑚∈ℕ\), \(𝑛=0⩽𝑚\).
\item
Hérédité.
On montre d'abord que s'il existe une injection, notée \(𝜑\), de \(⟦1;𝑛+1⟧\) dans \(⟦1;𝑚+1⟧\) alors il
existe une injection de \(⟦1;𝑛⟧\) dans \(⟦1;𝑚⟧\).
\\
Si \(𝑚+1∉\operatorname{Im}𝜑\), \(𝜑_{\left|⟦1;𝑛⟧\right.}\) est une injection \(⟦1;𝑛⟧\) dans \(⟦1;𝑚⟧\).
\\
Si \(𝑚+1∈\operatorname{Im}𝜑\), soit \(𝜏\) la transposition qui échange \(𝑛+1\) et \(𝜑^{-1}(𝑚+1)\),
\(𝜑∘𝜏_{\left|⟦1;𝑛⟧\right.}\) est une injection de \(⟦1;𝑛⟧\) dans \(⟦1;𝑚⟧\).

On combine avec une hypothèse de récurrence pour obtenir \(𝑛⩽𝑚\) et \(𝑛+1⩽𝑚+1\), ce qui donne l'hérédité.
\end{itemize}
\item Par récurrence sur \(𝑚\).
\begin{itemize}
\item
Initialisation. Pour \(𝑚=0\), comme \(𝑛∈ℕ\), \(𝑛⩾0=𝑚\).
\item
Hérédité. On montre d'abord que s'il existe une surjection, notée \(𝜑\), de \(⟦1;𝑛+1⟧\) sur \(⟦1;𝑚+1⟧\) alors il
existe une surjection de \(⟦1;𝑛⟧\) sur \(⟦1;𝑚⟧\).
\\
Si \(𝜑(𝑛+1)≠𝑚+1\), l'application
\begin{equation*}
\begin{matrix}
𝜓:&⟦1;𝑛⟧&⟶&⟦1;𝑚⟧
\\
&𝑖&⟼&\begin{cases}
𝜑(𝑛+1)\text{ si }𝜑(𝑖)=𝑚+1
\\
𝜑(𝑖)\text{ sinon}
\end{cases}
\end{matrix}
\end{equation*}
est une surjection. En effet, il existe \(𝑖\) de \(⟦1;𝑛+1⟧\) tel que \(𝜑(𝑖)=𝑚+1\). Par hypothèse, \(𝑖≠𝑛+1\), d'où
 \(𝜓(i)=𝜑(𝑛+1)\), donc \(𝜑(𝑛+1)∈\operatorname{Im}𝜓\). De plus, pour tout entier 𝑗 de \(⟦1;𝑚⟧\) autre que \(𝜑(𝑛+1)\), il
existe \(𝑖\) de \(⟦1;𝑛+1⟧\) tel que \(𝜑(𝑖)=𝑗\). Comme on a \(𝑗≠𝜑(𝑛+1)\), on a \(𝑖≠𝑛+1\). Comme on a \(𝑗≠𝑚+1\)
, on a \(𝜑(𝑖)≠𝑚+1\) et par définition de \(𝜓\), \(𝜑(𝑖)=𝜓(i)\). Ainsi \(𝑗∈\operatorname{Im}𝜓\) et \(𝜓\) est surjective.
\par\noindent
Si \(𝜑(𝑛+1)=𝑚+1\), l'application
\begin{equation*}
\begin{matrix}
𝜓:&⟦1;𝑛⟧&⟶&⟦1;𝑚⟧
\\
&𝑖&⟼&\begin{cases}
1\text{ si }𝜑(i)=𝑚+1
\\
𝜑(i)\text{ sinon}
\end{cases}
\end{matrix}
\end{equation*}
est une surjection. En effet, pour tout entier 𝑗 de \(⟦1;𝑚⟧\), il existe \(𝑖\) de \(⟦1;𝑛+1⟧\) tel que \(𝜑(𝑖)=𝑗\)
. Comme on a \(𝑗≠𝑚+1\), on a \(𝑖≠𝑛+1\) et d'où \(𝜓(i)=𝜑(𝑖)=𝑗\). Ainsi \(𝑗∈\operatorname{Im}𝜓\) et \(𝜓\) est
surjective.\qedhere
\end{itemize}
\end{enumerate}
\end{proof}
%
\begin{theorem}
S'il existe une bijection entre \(⟦1;𝑛⟧\) et \(⟦1;𝑚⟧\) alors \(𝑛=𝑚\).
\end{theorem}
\begin{proof}
S'il existe une bijection entre \(⟦1;𝑛⟧\) et \(⟦1;𝑚⟧\) alors en particulier :
\begin{enumerate}
\item il existe une injection de \(⟦1;𝑛⟧\) dans \(⟦1;𝑚⟧\), et \(𝑛⩽𝑚\),
\item il existe une surjection de \(⟦1;𝑛⟧\) sur \(⟦1;𝑚⟧\), et \(𝑚⩽𝑛\).
\end{enumerate}
Par conséquent, \(𝑛=𝑚\).
\end{proof}
%
\begin{theorem}
\par\noindent
\begin{enumerate}
\item Toute injection de \(⟦1;𝑛⟧\) dans lui-même est une bijection.
\item Toute surjection de \(⟦1;𝑛⟧\) sur lui-même est une bijection.
\end{enumerate}
\end{theorem}
%
\begin{proof}
\par\noindent
\begin{enumerate}
\item Par récurrence sur \(𝑛\).
\begin{itemize}
\item
Initialisation. Toute application de \( ∅\) dans lui-même est une bijection.
\item
Hérédité. Soit \(𝜑\) une injection de \(⟦1;𝑛+1⟧\) dans lui-même.
\\
Si \(𝑛+1=𝜑(𝑛+1)\), \(𝜑_{\left|⟦1;𝑛⟧\right.}\) est une injection de \(⟦1;𝑛⟧\) dans \(⟦1;𝑛+1⟧\) qui
n'atteint pas \(𝑛+1\), c'est donc une injection de \(⟦1;𝑛⟧\) dans lui-même. Par hypothèse de récurrence, c'est une
bijection, donc une surjection. Ainsi \(\Im 𝜑\), qui contient \(\Im 𝜑_{\left|⟦1;𝑛⟧\right.}\), contient \(⟦1;𝑛⟧\).
Comme il contient par ailleurs \(𝑛+1\), il contient \(⟦1;𝑛+1⟧\) donc \(𝜑\) est surjective : c'est une bijection.
\\
Si \(𝑛+1≠𝜑(𝑛+1)\), l'application \(𝜏\) de \(⟦1;𝑛+1⟧\) dans lui-même qui échange \(𝑛+1\) et \(𝜑(𝑛+1)\) est
une bijection :
\begin{equation*}
\begin{matrix}
𝜏~:&⟦1;𝑛+1⟧&⟶&⟦1;𝑛+1⟧
\\
&𝑖&⟼&\begin{cases}
𝑛+1~\text{si }~𝑖=𝜑(𝑛+1)
\\
𝜑(𝑛+1)~\text{si }~𝑖=𝑛+1
\\
𝑖\text{sinon}
\end{cases}
\end{matrix}
\end{equation*}
On a \(𝜏∘𝜑(𝑛+1)=𝑛+1\) et \(𝜏∘𝜑\) est une injection, par le point précédent, \(𝜏∘𝜑\) est aussi une bijection. Par
conséquent, \(𝜏∘𝜏∘𝜑\), qui vaut \(𝜑\), en est aussi une.
\end{itemize}%
\item Par récurrence sur \(𝑛\).
\begin{itemize}
\item
Initialisation.
Toute application de \( ∅\) dans lui-même est une bijection, ainsi que toute application de
\(⟦1;1⟧\) dans lui-même.
\item
Hérédité.
Pour tout \(𝑛⩾0\), soit \(𝜑\) une surjection de \(⟦1;𝑛+1⟧\) dans lui-même.
\\
Si \(𝑛+1=𝜑(𝑛+1)\). Pour tout 𝑦 de \(⟦1;𝑛⟧\), \(?\overset{-1}{?}(𝑦)\) est une partie non vide de \(⟦1;𝑛+1⟧\)
 qui ne contient pas \(𝑛+1\), c'est donc une partie non vide de \(⟦1;𝑛⟧\). Donc \(𝜑_{\left|⟦1;𝑛⟧\right.}\) a pour
image \(⟦1;𝑛⟧\) et par hypothèse de récurrence, elle induit une bijection de \(⟦1;𝑛⟧\) dans lui-même. Cela fait de
\(𝜑\) une bijection de \(⟦1;𝑛+1⟧\) dans lui-même.
\\
Si \(𝑛+1≠𝜑(𝑛+1)\), on a \(𝜏∘𝜑(𝑛+1)=𝑛+1\) et \(𝜏∘𝜑\) est une surjection, par le point précédent, \(𝜏∘𝜑\)
est aussi une bijection. Par conséquent, \(𝜏∘𝜏∘𝜑\), qui vaut \(𝜑\), en est aussi une.\qedhere
\end{itemize}
\end{enumerate}
\end{proof}
%
\begin{theorem}
\(ℕ\) est infini.
\end{theorem}
\begin{proof}
Soit
\begin{equation*}
\begin{matrix}
𝜑:&ℕ&⟶&ℕ^{\ast }
\\
&𝑛&⟼&𝑛+1
\end{matrix},\ \begin{matrix}
𝜓:&ℕ^{\ast
}&⟶&ℕ
\\
&𝑛&⟼&𝑛-1
\end{matrix}
\end{equation*}
On a \(∀𝑛∈ℕ,\ 𝜓∘𝜑(𝑛)=𝜓(𝑛+1)=𝑛\).
Donc, \(𝜑\) est une injection de \(ℕ\) dans \(ℕ^{\ast }\) qui est une partie stricte de \(ℕ\).
\end{proof}
%
\subsubsection[Cas général]{Cas général}
\begin{theorem}
\label{seq:refTheorem7}
L'ensemble \(𝐸\) est infini si et seulement s'il existe une injection de \(ℕ\) dans \(𝐸\).
\end{theorem}
\begin{proof}
Si \(𝜑\) est une injection de \(ℕ\) dans \(𝐸\), l'application
\begin{equation*}
\begin{matrix}
𝜓:&𝐸&⟶&𝐸
\\
&𝑥&⟼&\begin{cases}
𝑥\text{ si }𝑥∉\operatorname{Im}𝜑
\\
𝜑\bigl(1+𝜑^{-1}(𝑥)\bigr)\text{ sinon}
\end{cases}
\end{matrix}
\end{equation*}
est une injection de \(𝐸\) dans \(𝐸∖\{𝜑(0)\}\).
\end{proof}
%
%
\begin{theorem}
Soit \(𝑓:𝛦→𝐹\).
\begin{enumerate}
\item Si \(𝑓(𝛦)\) est infini alors \(𝐸\) est infini.
\item Si \(𝐸\) est fini alors \(𝑓(𝛦)\) est fini.
\item Si \(𝑓\) est injective, alors \(𝛦\) est infini si et seulement si \(𝑓(𝛦)\) est infini.
\item Si \(𝑓\) est injective, alors \(𝛦\) est fini si et seulement si \(𝑓(𝛦)\) est fini.
\end{enumerate}
\end{theorem}
\begin{proof}
\par\noindent
\begin{enumerate}
\item Soit \(𝜑:ℕ→𝑓(𝐸)\) une injection. On sait que \(𝑓\) induit une surjection de \(𝛦\) sur \(𝑓(𝛦)\), encore
notée \(𝑓\), qui admet une réciproque à droite, notée \(𝑔:𝑓(𝐸)→𝐸\), injective et telle que
\(𝑓∘𝑔=\operatorname{Id}_{𝑓(𝐸)}\). \(𝑔∘𝜑\) est une injection de \(ℕ\) dans \(𝐸\).
\item \ C'est la contraposée du point précédent.
\item Si \(𝜑:ℕ→𝐸\) est une injection, alors \(𝑓∘𝜑:ℕ→𝑓(𝐸)\) est une injection : si \(𝛦\) est infini, \(𝑓(𝛦)\)
aussi. Avec i) on a l'équivalence.
\item C'est la contraposée du point précédent.
\qedhere
\end{enumerate}
\end{proof}
\begin{theorem}
\label{seq:refTheorem9}
Il existe une injection de \(ℕ\) dans \(𝐸\) si et seulement si pour tout entier naturel \(𝑛\) il existe une injection
de \(⟦1;𝑛⟧\) dans \(𝐸\).
\end{theorem}
\begin{proof}
\par\noindent
\begin{enumerate}
\item
Condition nécessaire.
Soit \(𝜑\) une injection de \(ℕ\) dans \(𝐸\). Alors, pour tout \(𝑛∈ℕ\), \(𝜑_{\left|⟦1;𝑛⟧\right.}\) est une
injection de \(⟦1;𝑛⟧\) dans \(𝐸\) (y compris \(𝑛=0\) ).
\item
Condition suffisante.
Soit \(𝜑_𝑛\) l'injection de \(⟦1;𝑛⟧\) dans \(𝐸\). On définit l'injection de \(ℕ\) dans \(𝐸\) par récurrence.
On pose \(𝑒_1\mybydef{=}𝜑_1(1)\). Pour \(𝑛>1\), si \(𝑒_1\),..., \(𝑒_𝑛\) sont mutuellement
distincts, on n'a pas \(\operatorname{Im}𝜑_{𝑛+1}⊂\{𝑒_1,...,𝑒_𝑛\}\) sinon \(\{𝑒_𝑖↦𝑖\}∘𝜑_{𝑛+1}\) serait une injection
de \(⟦1;𝑛+1⟧\) dans \(⟦1;𝑛⟧\). Ainsi, il existe un élément de \(𝐸\) différent de \ \(𝑒_1\),..., \(𝑒_𝑛\), c'est
\(𝑒_{𝑛+1}\).
À préciser.
\end{enumerate}
\end{proof}
\begin{theorem}
\label{seq:refTheorem10}
Un ensemble \(𝐸\) est fini s'il existe une bijection de \(𝐸\) vers \(⟦1;𝑛⟧\) où \(𝑛∈ℕ\).
\end{theorem}
\begin{proof}
Par l'absurde, soit \(𝜑_𝑛\) une bijection de \(𝐸\) vers \(⟦1;𝑛⟧\) et \(𝜓\) une injection de \(𝐸\) vers \(𝐹\) où
\(𝐹⊊𝐸\). L'application \(𝜑_𝑛^{-1}∘𝜓∘𝜑_𝑛\) est injective. Son image est incluse dans \(𝜑_𝑛^{-1}(𝐹)\), c'est
donc une partie stricte de \(⟦1;𝑛⟧\). Contradiction avec la finitude de \(⟦1;𝑛⟧\).
\end{proof}
%
\subsection{Cardinal}
\subsubsection{Définition}
\begin{lemma}
Soit \(𝐸\) un ensemble. L'ensemble des entiers naturels \(𝑛\) tels qu'il existe une injection de \(⟦1;𝑛⟧\) dans
\(𝐸\) est non vide, s'il est majoré alors \(𝐸\) est fini, s'il n'est pas majoré alors \(𝐸\) est infini.
\end{lemma}
\begin{proof}
Notons \(𝐴\) l'ensemble des entiers naturels en question. Pour \(𝑛=0\), on a \(⟦1;0⟧=\mathsf{ ∅}\) et l'unique
application de \(\mathsf{ ∅}\) dans \(𝐸\) est injective. Ainsi, \(𝐴\) est non vide.

Si \(𝐴\) est majoré, notons \(𝑚\) son maximum et \(𝜑_𝑚\) une injection correspondante. Si \(𝜑_𝑚\) n'est pas
surjective, \(𝐸∖\operatorname{Im}𝜑_𝑚\) contient un élément \(𝑦\) et l'application
\begin{equation*}
\begin{matrix}
𝜓:&⟦0;𝑚+1⟧&⟶&𝐸
\\
&𝑛&⟼&\begin{cases}
𝑦\text{ si }𝑛=𝑚+1
\\
𝜑_𝑚(𝑛)\text{ sinon}
\end{cases}
\end{matrix}
\end{equation*}
est injective, ce qui contredit la maximalité de \(𝑚\). Donc \(𝜑_𝑚\) est une bijection et par le lemme
\ref{seq:refTheorem10}, \(𝐸\) est fini.

Si A n'est pas majoré alors pour tout \(𝑛∈ℕ^{ *}\) il existe une injection de \(⟦1;𝑚⟧\) dans \(𝐸\) avec
\(𝑛⩽𝑚\), qui par restriction, donne une injection de \(⟦1;𝑛⟧\) dans \(𝐸\). Par le lemme \ref{seq:refTheorem9} il y
a une injection de \(ℕ\) dans \(𝐸\) et par le lemme \ref{seq:refTheorem7} \(𝐸\) est infini.
\end{proof}
\begin{definition}
Soit \(𝐸\) un ensemble fini. \(𝑛\) est le \mykeyword{cardinal}, ou \mykeyword{nombre d'éléments}, de \(𝐸\) signifie
qu'il existe une bijection de \(⟦1;𝑛⟧\) dans \(𝐸\). Il est noté \(\operatorname{card}(𝐸)\), ou \(|𝐸|\).
\end{definition}
\begin{vocabulary}
\par\noindent
\begin{itemize}
\item
\(𝐸\) a \(𝑛\) éléments signifie \(\operatorname{card}(𝐸)=𝑛\).
\item
\(𝐸\) a au moins, ou plus de, \(𝑛\) éléments signifie \(\operatorname{card}(𝐸)⩾𝑛\).
\item
\(𝐸\) a strictement plus de \(𝑛\) éléments signifie \(\operatorname{card}(𝐸)>𝑛\).
\item
\(𝐸\) a au plus, ou moins de, \(𝑛\) éléments signifie \(\operatorname{card}(𝐸)⩽𝑛\).
\item
\(𝐸\) a strictement moins de \(𝑛\) éléments signifie \(\operatorname{card}(𝐸)<𝑛\).
\end{itemize}
On peut remplacer «a» par «possède», «contient» et autres synonymes.
\end{vocabulary}
\begin{theorem}
\begin{equation*}
|∅|=0
\end{equation*}
\end{theorem}
\begin{proof}
L'application vide est une bijection de \(⟦1;0⟧\) dans \( ∅\).
\end{proof}
\begin{theorem}
Le cardinal d'un singleton est \(1\).
\end{theorem}
\begin{proof}
Soit \(\{𝑥\}\) le singleton, l'unique application de \(⟦1;1⟧\) dans \(\{𝑥\}\) est une bijection.
\end{proof}
\subsubsection{Applications et cardinal}
\begin{theorem}
Soient \(𝐸\) et \(𝐹\) deux ensembles finis. Il existe une bijection de \(𝐸\) dans \(𝐹\) si et seulement si
\(\operatorname{card}(𝐸)=\operatorname{card}(𝐹)\).
\end{theorem}
\begin{proof}
 \(𝐸\) et \(𝐹\) sont finis donc il existe une bijection \(𝑓\) de \(⟦1;|𝐸|⟧\) dans \(𝐸\) et une bijection \(𝑔\) de
\(⟦1;|𝐹|⟧\) dans \(𝐹\).
\\
Si \(ℎ\) est une bijection de \(𝐸\) vers \(𝐹\), alors \(𝑔^{-1}\circ ℎ\circ 𝑓\) est une bijection de \(⟦1;|𝐸|⟧\)
vers \(⟦1;|𝐹|⟧\), donc, \(\operatorname{card}(𝐸)=\operatorname{card}(𝐹)\).
\\
Si \(\operatorname{card}(𝐸)=\operatorname{card}(𝐹)\), \(𝑔\circ 𝑓^{-1}\) est une bijection de \(𝐸\) vers F.
\end{proof}
\begin{theorem}
\par\noindent
\begin{enumerate}
\item S'il existe une injection de \(𝐸\) dans \(𝐹\) fini, \(𝐸\) est fini et
\(\operatorname{card}(𝐸)⩽\operatorname{card}(𝐹)\).
\item S'il existe une surjection de \(𝐸\) fini sur \(𝐹\), \(𝐹\) est fini et
\(\operatorname{card}(𝐸)⩾\operatorname{card}(𝐹)\).
\end{enumerate}
\end{theorem}
\begin{proof}
\par\noindent
\begin{enumerate}
\item Si \(𝜑\) est une injection de \(𝐸\) dans \(𝐹\), \(𝑔^{-1}∘𝜑∘𝑓\) est une injection de \(⟦1;|𝐸|⟧\) dans
\(⟦1;|𝐹|⟧\), donc \(\operatorname{card}(𝐸)⩽\operatorname{card}(𝐹)\).
\item Si \(𝜑\) est une surjection de \(𝐸\) sur \(𝐹\), \(𝑔^{-1}∘𝜑∘𝑓\) est une surjection de \(⟦1;|𝐸|⟧\) sur
\(⟦1;|𝐹|⟧\), donc \(\operatorname{card}(𝐸)⩾\operatorname{card}(𝐹)\).
\end{enumerate}
\end{proof}
%
%
\begin{theorem}
Soient \(𝐸\) et \(𝐹\) deux ensembles finis de même cardinal.
\begin{enumerate}
\item Toute injection de \(𝐸\) dans \(𝐹\) est une bijection.
\item Toute surjection de \(𝐸\) sur \(𝐹\) est une bijection.
\end{enumerate}
\end{theorem}
\begin{proof}
Soit \(𝑛\) le cardinal commun, \(𝑓\) une bijection de \(⟦1;𝑛⟧\) dans \(𝐸\), \(𝑔\) une bijection de \(⟦1;𝑛⟧\) dans
\(𝐹\) et \(𝜑:𝐸→𝐹\).
\begin{enumerate}
\item \(𝑔^{-1}∘𝜑∘𝑓\) est une injection de \(⟦1;𝑛⟧\) dans lui-même, c'est une bijection, donc \(𝜑\) aussi puisque
\(𝜑=𝑔∘(𝑔^{-1}∘𝜑∘𝑓)∘𝑓^{-1}\).
\item \(𝑔^{-1}∘𝜑∘𝑓\) est une surjection de \(⟦1;𝑛⟧\) dans lui-même, c'est une bijection, donc \(𝜑\) aussi puisque
\(𝜑=𝑔∘(𝑔^{-1}∘𝜑∘𝑓)∘𝑓^{-1}\).\qedhere
\end{enumerate}
\end{proof}
%
\begin{theorem}
Soit \(𝐸\) un ensemble fini et \(𝐹\) une partie de \(𝐸\), alors \(\operatorname{card}(𝐹)⩽\operatorname{card}(𝐸)\)
avec égalité si et seulement si \(𝐸\) et \(𝐹\) sont égaux.
\end{theorem}
\begin{proof}
On applique le lemme à l'injection canonique pour obtenir l'inégalité. En cas d'égalité des cardinaux, l'injection
canonique est aussi une surjection et son image vaut à la fois \(𝐸\) et \(𝐹\).
\end{proof}
\begin{theorem}
Soit \(𝑓:𝐸→𝐹\). Si \(𝐸\) est fini alors \(\operatorname{card}(𝑓(𝐸))⩽\operatorname{card}(𝐸)\) avec égalité si et
seulement si \(𝑓\) est injective
\end{theorem}
\begin{proof}
Comme \(𝑓\) induit une surjection de \(𝐸\) sur \(𝑓(𝐸)\) on a l'inégalité. On a égalité si et seulement si \(𝑓\)
induit une bijection de \(𝐸\) sur \(𝑓(𝐸)\), id est \(𝑓\) injective.
\end{proof}
\subsubsection{Opérations binaires}
\begin{theorem}
Étant donnés \(𝐸\) et \(𝐹\) deux ensembles finis disjoints, \(𝐸∪𝐹\) est fini et
\(\operatorname{card}(𝐸∪𝐹)=\operatorname{card}(𝐸)+\operatorname{card}(𝐹)\).
\end{theorem}
\begin{proof}
\(𝐸\) étant fini, il existe une bijection \(𝑓\) de \(⟦1;|𝐸|⟧\) dans \(𝐸\). De même, il existe une bijection \(𝑔\)
de \(⟦1;|𝐹|⟧\) dans \(𝐹\). Les applications
\begin{equation*}
\begin{matrix}
ℎ:&⟦1;|𝐸|+|𝐹|⟧&⟶&𝐸∪𝐹\phantom{A^A}
\\
&𝑖&⟼&\begin{cases}
𝑓(𝑖)\text{ si }1⩽𝑖⩽|𝐸|
\\
𝑔(𝑖-|𝐸|)\text{ si }|𝐸|<𝑖⩽|𝐸|+|𝐹|\phantom{A^A}
\end{cases}
\end{matrix}
\end{equation*}
et
\begin{equation*}
\begin{matrix}
𝑘:&𝐸∪𝐹&⟶&⟦1;|𝐸|+|𝐹|⟧\phantom{A^A}
\\
&𝑥&⟼&\begin{cases}
𝑓^{-1}(𝑥)\text{ si }𝑥∈𝐸
 \\
|𝐸|+𝑔^{-1}(𝑥)\text{ si }𝑥∈𝐹\phantom{A^A}
 \end{cases}
\end{matrix}
\end{equation*}
sont bien définies.

Pour \(ℎ\), d'une part \(⟦1;|𝐸|⟧\) et \(⟧|𝐸|;|𝐸|+|𝐹|⟧\) sont les deux composantes d'une partition de
\(⟦1;|𝐸|+|𝐹|⟧\) et d'autre part, si \(1⩽𝑖⩽|𝐸|\), \(𝑓\) est définie en 𝑖 \({}\) et \(𝑓(𝑖)∈𝐸∪𝐹\) et si
\(|𝐸|<𝑖⩽|𝐸|+|𝐹|\), on a \(0<𝑖-|𝐸|⩽|𝐹|\), \(𝑔\) définie en \(𝑖-|𝐸|\) et \(𝑔(𝑖-|𝐸|)∈𝐸∪𝐹\).

Pour \(𝑘\), d'une part \(𝐸\) et \(𝐹\) sont les deux composantes d'une partition de \(𝐸∪𝐹\) et d'autre part, si on a
 \(𝑥∈𝐸\), \(𝑓^{-1}\) est définie en \(𝑥\) et on a \(𝑓^{-1}(𝑥)∈⟦1;|𝐸|+|𝐹|⟧\) et si \(𝑥∈𝐹\), \(𝑔^{-1}\) est
définie en \(𝑥\) et on a \(⟦1;|𝐸|+𝑔^{-1}(𝑥)∈⟦1;|𝐸|+|𝐹|⟧\).

Montrons que \(ℎ\) et \(𝑘\) sont réciproques l'une de l'autre.
\begin{itemize}
\item
On a \(∀𝑖∈⟦1;|𝐸|+|𝐹|⟧,\ 𝑘∘ℎ(𝑖)=𝑖\).

Si \(1⩽𝑖⩽|𝐸|\) alors \(ℎ(𝑖)∈𝐸\), car \ \(ℎ(𝑖)\mybydef{=}𝑓(𝑖)\) et~ \(𝑓(𝑖)∈𝐸\), d'où
\(𝑘\bigl(𝑓(𝑖)\bigr)=𝑓^{-1}\bigl(𝑓(𝑖)\bigr)=𝑖\).

Si \(|𝐸|<𝑖⩽|𝐸|+|𝐹|\) alors \(ℎ(𝑖)∈𝐹\), car \ \(ℎ(𝑖)\mybydef{=}𝑔(𝑖-|𝐸|)\) et~ \(𝑔(𝑖-|𝐸|)∈𝐹\),
d'où il vient \(𝑘(𝑔(𝑖-𝑚))=|𝐸|+𝑔^{-1}\bigl(𝑔(𝑖-|𝐸|)\bigr)=|𝐸|+𝑖–|𝐸|=𝑖\).
\item
On a \(∀𝑥∈𝐸∪𝐹,\ ℎ∘𝑘(𝑥)=𝑥\).

Si \(𝑥∈𝐸\) alors \(𝑘(𝑥)∈⟦1;|𝐸|⟧\), car \ \(𝑘(𝑥)\mybydef{=}𝑓^{-1}(𝑥)\) et \(𝑓^{-1}(𝑥)∈⟦1;|𝐸|⟧\),
d'où il vient \(ℎ∘𝑘(𝑥)=ℎ\bigl(𝑓^{-1}(𝑥)\bigr)=𝑓\bigl(𝑓^{-1}(𝑥)\bigr)=𝑥\).

Si \(𝑥∈𝐹\) alors \(𝑘(𝑥)∈⟧|𝐸|;|𝐸|+|𝐹|⟧\), \(𝑘(𝑥)\mybydef{=}|𝐸|+𝑔^{-1}(𝑥)\) et
\(𝑔^{-1}(𝑥)∈⟦1;|𝐹|⟧\), d'où il vient \(ℎ∘𝑘(𝑥)=ℎ\bigl(|𝐸|+𝑔^{-1}(𝑥)\bigr)=𝑔\bigl(|𝐸|+𝑔^{-1}(𝑥)-|𝐸|\bigr)=𝑔\bigl(𝑔^{-1}(𝑥)\bigr)=𝑥\).
\end{itemize}
Ainsi \(ℎ\) est bijective et \(\operatorname{card}(𝐸∪𝐹)=\operatorname{card}(𝐸)+\operatorname{card}(𝐹)\).
\end{proof}
%
\begin{theorem}
Soit \(𝐹\) partie de \(𝐸\) fini, on a \(\operatorname{card}(𝐸∖
𝐹)=\operatorname{card}(𝐸)–\operatorname{card}(𝐹)\).
\end{theorem}
\begin{proof}
Sachant que \(𝐸=(𝐸∖𝐹)∪𝐹\) et \((𝐸∖𝐹)∩𝐹=∅\), on applique la proposition précédente d'où
\(\operatorname{card}(𝐸)=\operatorname{card}(𝐸∖𝐹)+\operatorname{card}(𝐹)\).
\end{proof}
%
\begin{theorem}
Soient \(𝐸\) et \(𝐹\) deux ensembles finis. \(𝐸∪𝐹\) et \(𝐸∩𝐹\) sont finis et
 \(\operatorname{card}(𝐸∪𝐹)=\operatorname{card}(𝐸)+\operatorname{card}(𝐹)–\operatorname{card}(𝐸∩𝐹)\).
\end{theorem}
\begin{proof}
 \(𝐸∩𝐹\) et \(𝐸∖𝐹\) sont finis en tant que parties de \(𝐸\) qui est fini. De plus, on a \(𝐸∪𝐹=(𝐸∖𝐹)∪𝐹\)
 avec \((𝐸∖𝐹)∩𝐹=∅\). Par ce qui précède, \(𝐸∪𝐹\) est fini et
\(\operatorname{card}(𝐸∪𝐹)=\operatorname{card}(𝐸∖𝐹)+\operatorname{card}(𝐹).\) Par ailleurs, on a
\(𝐸=(𝐸∖𝐹)∪(𝐸∩𝐹)\) avec \((𝐸∖𝐹)∩(𝐸∩𝐹)=∅\), d'où par ce qui précède
\(\operatorname{card}(𝐸)=\operatorname{card}(𝐸∖𝐹)+\operatorname{card}(𝐸∩𝐹)\). En combinant, il vient le résultat.
\end{proof}
\subsubsection{Principe des bergers}
\begin{theorem}
Soit \((𝐸_𝑖)_{𝑖∈𝐼}\) avec \(𝐼\) fini, telle que
\begin{enumerate}
\item \(∀𝑖∈𝐼,\ 𝐸_𝑖\text{ est fini}\),
\item \(
∀𝑖,𝑗∈𝐼,\ 𝑖≠𝑗⇒𝐸_𝑖∩𝐸_𝑗=∅
\)
\end{enumerate}
alors \(\operatorname{card}\Bigl(\bigcup_{𝑖∈𝐼}𝐸_𝑖\bigr)=\sum
_{𝑖∈𝐼}\operatorname{card}(𝐸_𝑖)\) \footnote{Où sont définies les unions, intersections, sommes et produits finis ?}.
\end{theorem}
\begin{proof}
Par récurrence sur le cardinal de \(𝐼\).
\begin{itemize}
\item
Initialisation.
Si \(|𝐼|=0\), c'est-à-dire \(𝐼\) est vide, d'une part la réunion est vide, donc son cardinal est
 \(0\), et d'autre part la somme est vide, donc vaut aussi \(0\).
\item
Hérédité.
Si \(|𝐼|=𝑛+1\), \(𝐼\) n'est pas vide et contient \(𝑖_0\). On a
\begin{equation*}
𝐼=𝐼∖\{𝑖_0\}∪\{𝑖_0\}
\text{ et }
𝐼∖\{𝑖_0\}∩\{𝑖_0\}= ∅
\end{equation*}
d'après ce qui précède,
\begin{equation*}
𝑛+1=|𝐼|=|𝐼∖\{𝑖_0\}|∪|\{𝑖_0\}|=|𝐼∖\{𝑖_0\}|+1
\end{equation*}
d'où \(|𝐼∖\{𝑖_0\}|=𝑛\).

Par ailleurs,
\begin{equation*}
\bigcup_{𝑖∈𝐼}𝐸_{𝑖}=𝐸_{𝑖_0}∪\left(\bigcup_{𝑖∈𝐼∖\{𝑖_0\}}𝐸_{𝑖}\right)
\end{equation*}
et
\begin{equation*}
𝐸_{𝑖_0}∩\Bigl(\bigcup_{𝑖∈𝐼∖\{𝑖_0\}}𝐸_{𝑖}\Bigr)=\bigcup_{𝑖∈𝐼∖\{𝑖_0\}}\left(𝐸_{𝑖_0}∩𝐸_{𝑖}\right)=\bigcup_{𝑖∈𝐼∖\{𝑖_0\}}
∅= ∅,
\end{equation*}
donc
\(\Bigl|\bigcup_{𝑖∈𝐼}𝐸_{𝑖}\Bigr|=|𝐸_{𝑖_0}|+\Bigr|\bigcup_{𝑖∈𝐼∖\{𝑖_0\}}𝐸_{𝑖}\Bigr|=|𝐸_{𝑖_0}|+\sum
_{𝑖∈𝐼∖\{𝑖_0\}}|𝐸_{𝑖}|=\sum _{𝑖∈𝐼}|𝐸_{𝑖}|\).\qedhere
\end{itemize}
\end{proof}
\begin{lemma}
[des bergers]
Soit \(𝐸\) un ensemble et \(𝐸_{1,}…,𝐸_𝑟\) une partition de \(𝐸\).
Si \(\operatorname{card}𝐸_1=...=\operatorname{card}𝐸_𝑟=𝑝\) alors \(\operatorname{card}𝐸=𝑟\times 𝑝\).
\end{lemma}
\begin{proof}
\(\operatorname{card}𝛦=\operatorname{card}\left(\bigcup_{𝑖=1}^{𝑟}𝐸_𝑖\right)=\sum
_{𝑖=1}^{𝑟}\operatorname{card}(𝐸_𝑖)=𝑝\sum _{𝑖=1}^{𝑟}1=𝑟×𝑝\)
\end{proof}
%
\begin{theorem}
[des bergers]
Soient \(𝐸\), \(𝐹\) deux ensembles finis et \(𝑓:𝐸→𝐹\) une application surjective telle que tout élément de \(𝐹\) a
exactement \(𝑛\) antécédents dans \(𝐸\). Alors on a \(\operatorname{card}(𝐸)=𝑛\times \operatorname{card}(𝐹)\).
\end{theorem}
\begin{proof}
On sait que \(\bigcup_{𝑦∈𝑓(𝛦)}\overset{-1}{𝑓}(𝑦)\) est une partition\footnote{Voir le chapitre
sur les relations d'équivalence.} de \(𝛦\) et par surjection \(𝑓(𝛦)=𝐹\), donc
\begin{equation*}
\bigl|𝛦\bigr|=\Bigl|\bigcup_{𝑦∈𝐹}\overset{-1}{𝑓}(𝑦)\Bigr|=\sum
_{𝑦∈𝐹}\Bigl|\overset{-1}{𝑓}(𝑦)\Bigr|=𝑛\sum _{𝑦∈𝐹}1=𝑛×\bigl|𝐹\bigr|.\qedhere
\end{equation*}
\end{proof}
%
\begin{theorem}
Soit \(𝑓:𝐸→𝐹\) avec \(𝐸\) fini.
\begin{equation*}
\operatorname{card}(𝛦)=\sum _{𝑦∈𝐹}\operatorname{card}\bigl(\overset{-1}{𝑓}(𝑦)\bigr)
\end{equation*}
\end{theorem}
\begin{proof}
C'est une partie de la preuve précédente.
\end{proof}
%
\subsubsection{Produit cartésien}
\begin{theorem}
Soient \(𝐺\) et \(𝐹\) deux ensembles finis alors \(𝐺×𝐹\) est fini et
\begin{equation*}
\operatorname{card}(𝐺×𝐹)=\operatorname{card}(𝐺)×\operatorname{card}(𝐹)
\end{equation*}
\end{theorem}
\begin{proof}
Soit \(𝑔\) une bijection de \(⟦1;|𝐺|⟧\) dans \(𝐺\) et \(𝑓\) une bijection de \(⟦1;|𝐹|⟧\) dans \(𝐹\). Les
applications
\begin{gather*}
\begin{matrix}
ℎ:&⟦1;|𝐺|⟧×⟦1;|𝐹|⟧&⟶&𝐺\text{𝑥}𝐹
\\
&(𝑖;𝑗)&⟼&\bigl(𝑔(𝑖);𝑓(𝑗)\bigr)
\end{matrix}
\intertext{et}
\begin{matrix}
𝑘:&𝐺×𝐹&⟶&⟦1;|𝐺|⟧×⟦1;|𝐹|⟧
\\
&(𝑥;𝑦)&⟼&\bigl(𝑔^{-1}(𝑥);𝑓^{-1}(𝑦)\bigr)
\end{matrix}
\end{gather*}
sont bien définies et réciproques l'une de l'autre car
\begin{multline*}
∀(𝑖;𝑗)∈⟦1;|𝐺|⟧×⟦1;|𝐹|⟧,
\\
𝑘∘ℎ(𝑖;𝑗)=𝑘\bigl(𝑔(𝑖),𝑓(𝑗)\bigr)=\bigl(𝑔^{-1}∘𝑔(𝑖),𝑓^{-1}∘𝑓(𝑗)\bigr)=(𝑖;𝑗)
\end{multline*}
\begin{multline*}
∀(𝑖;𝑗)∈⟦1;|𝐺|⟧×⟦1;|𝐹|⟧,
\\
𝑘∘ℎ(𝑖;𝑗)=𝑘\bigl(𝑔(𝑖),𝑓(𝑗)\bigr)=\bigl(𝑔^{-1}∘𝑔(𝑖),𝑓^{-1}∘𝑓(𝑗)\bigr)=(𝑖;𝑗)
\end{multline*}
\begin{multline*}
∀(𝑥,𝑦)∈𝐺×𝐹,
\\
ℎ∘𝑘(𝑥,𝑦)=ℎ\bigl(𝑔^{-1}(𝑥),𝑓^{-1}(𝑦)\bigr)=\bigl(𝑔∘𝑔^{-1}(𝑥),𝑓∘𝑓^{-1}(𝑦)\bigr)=(𝑥,𝑦)
\end{multline*}
Cela montre que \(ℎ\) est bijective, donc \(𝐺×𝐹\) et \(⟦1;|𝐺|⟧×⟦1;|𝐹|⟧\) sont équipotents.

De plus, les applications
\begin{gather*}
\begin{matrix}
ℎ :&⟦1;|𝐺|⟧×⟦1;|𝐹|⟧&⟶&⟦1;|𝐺|×|𝐹|⟧
\\
&(𝑖;𝑗)&⟼&(𝑖-1)|𝐹|+𝑗
\end{matrix}
\intertext{et}
\begin{matrix}
𝑘\text
:&⟦1;|𝐺|×|𝐹|⟧&⟶&⟦1;|𝐺|⟧×⟦1;|𝐹|⟧
\\
&𝑛&⟼&(1+(𝑛-1)÷|𝐹|;1+(𝑛-1)\mathbin{\operatorname{\%}}|𝐹|)
\end{matrix}
\end{gather*}
sont bien définies et réciproques l'une de l'autre. Pour \(ℎ\), l'ensemble d'arrivée est justifié par
\begin{equation*}
1⩽𝑖⩽|𝐺|\myand1⩽𝑗⩽|𝐹|⇒1⩽(𝑖-1)|𝐹|+𝑗⩽(|𝐺|-1)|𝐹|+|𝐹|=|𝐺||𝐹|
\end{equation*}
et pour \(𝑔\), l'ensemble d'arrivée est justifié par
\begin{align*}
1⩽𝑛⩽|𝐺||𝐹|
&{}
⇒0⩽𝑛-1⩽|𝐺||𝐹|-1⇒0⩽(𝑛-1)÷|𝐹|⩽|𝐺|-1
\\&{}
⇒1⩽1+(𝑛-1)÷|𝐹|⩽|𝐺|
\end{align*}
et \(0⩽(𝑛-1)\mathbin{\operatorname{\%}}|𝐹|⩽|𝐹|-1⇒1⩽1+(𝑛-1)\mathbin{\operatorname{\%}}|𝐹|⩽|𝐹|\).

Pour la réciprocité, on a
\begin{align*}
𝑘∘ℎ(𝑖;𝑗)&{}=𝑘((𝑖-1)|𝐹|+𝑗)
\\&{}
=(1+((𝑖-1)|𝐹|+𝑗-1)÷|𝐹|;1+((𝑖-1)|𝐹|+𝑗-1)\mathbin{\operatorname{\%}}|𝐹|)
\\&{}
=(1+(𝑖-1)+(𝑗-1)÷|𝐹|;1+(𝑗-1)\mathbin{\operatorname{\%}}|𝐹|)
\\&{}
=(𝑖;𝑗)
\end{align*}
sachant que
\begin{equation*}
1⩽𝑗⩽|𝐹|⇒0⩽𝑗-1⩽|𝐹|-1⇒(𝑗-1) ÷|𝐹|=0
\end{equation*}
et \((𝑗-1) ÷|𝐹|=𝑗-1\) et
\begin{align*}
ℎ∘𝑘(𝑛)
&{}
=ℎ((1+(𝑛-1)÷|𝐹|;1+(𝑛-1)\mathbin{\operatorname{\%}}|𝐹|))
\\&{}
=(1+(𝑛-1)÷|𝐹|-1)|𝐹|+1+(𝑛-1)\mathbin{\operatorname{\%}}|𝐹|
\\&{}
=1+((𝑛-1)÷|𝐹|)|𝐹|+(𝑛-1)\mathbin{\operatorname{\%}}|𝐹|
\end{align*}
qui fait apparaître la division euclidienne de \(𝑛-1\) par \(|𝐹|\) et donne \(ℎ∘𝑘(𝑛)=ℎ\).

Ainsi \(𝑘\) est bijective, donc \(⟦1;|𝐺|⟧×⟦1;|𝐹|⟧\) est fini et son cardinal est \ \(|𝐺||𝐹|\). Comme il est
équipotent à \(𝐺×𝐹\), ce dernier est fini et de même cardinal.
\end{proof}
%
\begin{theorem}
Pour tout \(𝑗∈ℕ^{\ast }\), \(\operatorname{card}(𝐸^𝑗)=(\operatorname{card}(𝐸))^𝑗\).
\end{theorem}
\begin{proof}
Par récurrence sur \(𝑗\) dans \(ℕ^{\ast}\).
\begin{itemize}
\item Initialisation. Pour \(𝑗=1\), on a
\(\operatorname{card}(𝐸^1)=\operatorname{card}(𝐸)=\bigl(\operatorname{card}(𝐸)\bigr)^1\).
\item Hérédité. Par ce qui précède, \(|𝐸^{𝑗+1}|=|𝐸^{𝑗}×𝐸|=|𝐸^{𝑗}||𝐸|=|𝐸|^{𝑗}|𝐸|=|𝐸|^{𝑗+1}\).
\end{itemize}
\end{proof}
