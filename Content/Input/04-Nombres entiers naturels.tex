% !TEX encoding = UTF-8
% !TEX program = xelatex
% !TEX root = ../Main/main.tex

\section{Nombres entiers naturels}
Construction de \(ℕ\).
\subsection{Présentation axiomatique}
\subsubsection{Définition}
%
\begin{definition} 
\(ℕ\) est un ensemble dont les éléments sont appelés \mykeyword{entiers naturels}.
\end{definition}
%
\begin{axiom}
[de Peano]
\par\noindent
\begin{enumerate}[
    label=\textcolor{ocre}{\textbf{P\arabic*)}},
    ref=P\arabic*,
    itemindent=1.5\parindent
  ]
\item
\label{axm.peano.1}
\(0\) est un entier naturel, lu «~zéro~».
\item
\label{axm.peano.2}
Tout entier naturel \(𝑛\) a un unique \mykeyword{successeur} noté \(𝑛_+\).
\item
\label{axm.peano.3}
Aucun entier naturel n'a \(0\) pour successeur.
\item
\label{axm.peano.4}
Deux entiers naturels ayant même successeur sont égaux.
\item
\label{axm.peano.5}
\mykeyword{Récurrence}. Si un ensemble d'entiers naturels contient \(0\) et contient le successeur de
chacun de ses éléments alors cet ensemble est égal à \(ℕ\).
\begin{equation*}
\begin{cases}
𝐸⊂ℕ\\0∈𝐸\\∀𝑛∈ℕ,\ 𝑛∈𝐸⇒𝑛_+∈𝐸
\end{cases}
⟹𝐸=ℕ
\end{equation*}
\end{enumerate}
\end{axiom}
\begin{remark}
On peut reformuler ces axiomes en utilisant la notion d'application.
\end{remark}
%
\begin{definition} 
\(ℕ^{\ast }\mybydef{=}ℕ∖\{0\}\), lu «~n étoile~».
\end{definition}
%
\begin{definition} 
\mykeyword{Suivant} est synonyme de \mykeyword{successeur}.
\(𝑝\) 
\mykeyword{succède à} \(𝑞\) signifie que \(𝑝\) est \ le successeur de \(𝑞\).
\end{definition}
%
\begin{definition}
[Nombres] 
\par\noindent
\begin{itemize}
\item
\(1\mybydef{=}0₊\), lu «~un~»,
\item
\(2\mybydef{=}1₊\), lu «~deux~», 
\item
\(3\mybydef{=}2₊\), lu «~trois~»,
\item
\(4\mybydef{=}3₊\), lu «~quatre~»,
\item
\(5\mybydef{=}4₊\), lu «~cinq~»,
\item
\(6\mybydef{=}5₊\), lu «~six~»,
\item
\(7\mybydef{=}6₊\), lu «~sept~»,
\item
\(8\mybydef{=}7₊\), lu «~huit~»,
\item
\(9\mybydef{=}8₊\), lu «~neuf~».
\end{itemize}
\end{definition}
%
\begin{notation}
\label{ntn.N.successeur}
\(𝑛₊\) est noté \(𝑛+1\)
\end{notation}
%
\subsubsection{Principe de récurrence}
Si un ensemble \(𝐸\) inclus dans \(ℕ\) est donné en compréhension, \emph{id est} 
\(𝐸\mybydef{=}\bigl\{𝑛∈ℕ\mathbin|𝖯(𝑛)\bigr\}\), \(𝐸\) contient \(𝑛\) signifie \(𝖯(𝑛)\) est vraie :
on spécialise l'axiome \ref{axm.peano.5}.
%
\begin{axiom}
[Récurrence]
Si \(𝖯(0)\) est vraie et si pour tout entier naturel \(𝑛\), \(𝖯(𝑛)⇒𝖯(𝑛+1)\) est vraie alors pour tout entier
naturel \(𝑛\), \(𝖯(𝑛)\) est vraie.
\end{axiom}
%
\begin{attention}
Ceci n'est pas un théorème, c'est exactement l'axiome \ref{axm.peano.5} dans le cas particulier des
ensembles définis en compréhension.
\end{attention}
%
\begin{theorem} 
Pour démontrer \(∀𝑛∈ℕ,\ 𝖯(𝑛)\),
\begin{enumerate}
\item on démontre \(𝖯(0)\), c'est \mykeyword{l'initialisation,}
\item on démontre \(∀𝑛∈ℕ,\ 𝖯(𝑛)⇒𝖯(𝑛+1)\), c'est \mykeyword{l'hérédité}.
\end{enumerate}
\end{theorem}
%
\begin{theorem}
[Récurrence à deux niveaux]
Si
\begin{enumerate}
\item
\(𝖯(0)\) est vraie,
\item
\(𝖯(1)\) est vraie et
\item
pour tout entier naturel \(𝑛\), on a \(𝖯(𝑛)\myand𝖯(𝑛+1)⇒𝖯(𝑛+2)\) est vraie
\end{enumerate}
alors pour tout entier naturel \(𝑛\), \(𝖯(𝑛)\) est vraie.
\end{theorem}
%
\begin{proof}
Soit \(𝖰(𝑛)\mybydef{=}𝖯(𝑛)\text{ et }𝖯(𝑛+1)\). On a \(𝖰(0)=𝖯(0)\text{ et }𝖯(1)\),
%
\begin{equation*}
(𝖯(𝑛)\text{ et }𝖯(𝑛+1)⇒𝖯(𝑛+2))⟺(𝖯(𝑛)\text{ et }𝖯(𝑛+1)⇒𝖯(𝑛+1)\text{ et }𝖯(𝑛+2))
\end{equation*}
ainsi que
\(∀𝑘∈ℕ,\ 𝖯(𝑛)⟺∀𝑘∈ℕ,\ 𝖰(𝑛)\).
Donc les deux formulations de récurrence sont équivalentes.
\end{proof}
%
\begin{remark}
Les récurrences à plus de niveaux sont laissées en exercice.
\end{remark}
%
Si nous anticipons sur la définition de la relation d'ordre, nous avons
%
\begin{theorem}
[Récurrence forte]
Si \(𝖯(0)\) est vraie et si pour tout entier naturel \(𝑛\), \((∀0≤𝑘≤𝑛,𝖯(𝑘))⇒𝖯(𝑛+1)\) est vraie alors pour tout
entier naturel \(𝑛\), \(𝖯(𝑛)\) est vraie.
\end{theorem}
%
\begin{proof}
Soit \(𝖰(𝑛)\mybydef{=}\bigl(∀0≤𝑘≤𝑛,𝖯(𝑘)\bigr)\). On a \(𝖰(0)=𝖯(0)\),
\begin{align*}
\lefteqn{\bigl(∀0≤𝑘≤𝑛,𝖯(𝑘)\bigr)⇒𝖯(𝑛+1)}\hspace{4em}
\\&{}⟺
(∀0≤𝑘≤𝑛,\ 𝖯(𝑘))⇒(∀0≤𝑘≤𝑛+1,𝖯(𝑘))
\\&{}⟺
𝖰(𝑛)⇒𝖰(𝑛+1)
\end{align*}
ainsi que
\(\bigl(∀𝑛∈ℕ,\ 𝖯(𝑛)\bigr)⟺\bigl(∀𝑛∈ℕ,\ 𝖰(𝑛)\bigr)\).
Donc les deux formulations de récurrence sont équivalentes.
\end{proof}
%
\begin{definition} 
Un prédécesseur de \(𝑛\) est un entier naturel, noté \(𝑛₋\), dont le successeur est \(𝑛\).
\end{definition}
%
\begin{theorem} 
Tout entier naturel \(𝑛\) admet au plus un prédécesseur.
\end{theorem}
%
\begin{proof}
Par définition, deux prédécesseurs de \(𝑛\) ont le même successeur \(𝑛\). Par l'axiome \ref{axm.peano.4}
ils sont égaux.
\end{proof}
%
\begin{theorem} 
Tout entier naturel est le prédécesseur de son successeur.
%
\begin{equation*}
∀𝑛∈ℕ,\ 𝑛=(𝑛₊)₋
\end{equation*}
\end{theorem}
%
\begin{corollary}
En particulier, tout successeur admet un prédécesseur.
\end{corollary}
%
\begin{proof}
Par définition du prédécesseur on a \(\bigl((𝑛₊)₋\bigr)₊=𝑛₊\),
ainsi \(𝑛\) et \((𝑛₊)₋\) ont pour successeur \(𝑛₊\), ils sont donc égaux par l'axiome \ref{axm.peano.4}.
\end{proof}
%
\begin{theorem} 
Les entiers naturels non nuls sont les successeurs.
\end{theorem}
%
\begin{corollary}
En particulier, tout entier naturel non nul admet un prédécesseur.
\end{corollary}
%
\begin{proof}
Soit \(𝐸\) l'ensemble des entiers naturels successeurs et \(𝐹\mybydef{=}\{0\}∪𝐸\). Par l'axiome
\ref{axm.peano.2}, tout élément de \(𝐹\) admet un successeur qui est dans \(𝐸\) et \emph{a fortiori} dans \(𝐹\). Ainsi, \(𝐹\) est une partie de \(ℕ\) qui contient \(0\) et le successeur de chacun de ses éléments,
c'est donc \(ℕ\) lui-même par l'axiome \ref{axm.peano.5}. Par l'axiome \ref{axm.peano.3}, \(𝐸\)
ne contient pas \(0\), donc \(ℕ∖\{0\}=\left(\{0\}∪𝐸\right)∖\{0\}=
∅∪\left(𝐸∖\{0\}\right)=𝐸\).
\end{proof}
%
\begin{theorem} 

Tout entier naturel non nul est le successeur de son prédécesseur.
%
\begin{equation*}
∀𝑛∈ℕ^{ *},\ 𝑛=(𝑛₋)₊
\end{equation*}
\end{theorem}
%
\begin{proof}
\item 

C'est une application directe de la définition et des lemmes précédents.
\end{proof}
%
\begin{theorem} 
L'application \(𝑛↦𝑛₊\) est un bijection de \(ℕ\) sur \(ℕ^{\ast}\), elle est \mykeyword{nommée décalage à
droite}. Sa réciproque est nommée \mykeyword{décalage à gauche}.
\end{theorem}
%
\begin{proof}
Immédiat. À préciser...
\end{proof}
%
\begin{definition} 
Soit \(𝐸\) une partie de \(ℕ\), \(𝑝\) est un \mykeyword{élément minimal} de \(𝐸\) signifie qu'il appartient à \(𝐸\) et que son
prédécesseur, s'il existe, n'appartient pas à \(𝐸\).

\(𝐸\) admet un élément minimal signifie qu'il existe un élément minimal de \(𝐸\).
%
\end{definition}
%
\begin{examples}
\par\noindent
\begin{enumerate}
\item Si \(𝐸\) contient \(0\), celui-ci est minimal.
\item Dans l'ensemble des entiers naturels pairs, tout entier est minimal.
\end{enumerate}
\end{examples}
%
\begin{theorem} 
Toute partie non vide de \(ℕ\) continet un élément minimal.
\end{theorem}
%
\begin{proof}
Soit \(𝐸\) une telle partie. Si \(𝐸\) contient \(0\), c'est immédiat. Sinon le complémentaire de \(𝐸\) contient \(0\), mais
n'est pas \(ℕ\) : par l'axiome \ref{axm.peano.5} il ne peut pas contenir le successeur de chacun de ses
éléments. Il existe donc un entier qui n'appartient pas à \(𝐸\) mais dont le successeur appartient à \(𝐸\). Ce
successeur est un élément minimal de \(𝐸\).
\end{proof}
%
\begin{attention}
Le théorème suivant est \mykeyword{fondamental~}! Il est utilisé notamment pour les suites récurrentes.
La preuve de l'unicité est facile, celle de l'existence est plus technique sans être réellement difficile.
\end{attention}
%
\begin{theorem}
[Définition par récurrence]
Soit un ensemble \(𝐸\) qui contient \(𝑎\) et \(𝐻~:ℕ×𝐸→𝐸\). Il existe une application \(𝑓\) de \(ℕ\) dans \(𝐸\), et
une seule, telle que
%
\begin{enumerate}
\item %
\(
𝑓(0)=𝑎
\)
\item \(∀𝑛∈ℕ,\ 𝑓(𝑛₊)=𝐻\bigl(𝑛,𝑓(𝑛)\bigr)\).
\end{enumerate}
\end{theorem}
%
\begin{proof}
\par\noindent
\begin{enumerate}
\item Unicité. Soient \(𝑓_1\) et \(𝑓_2\) de telles applications et 
\begin{equation*}
𝐹\mybydef{=}\left\{𝑛∈ℕ\mathbin|𝑓_1(𝑛)=𝑓_2(𝑛)\right\}
\end{equation*}
Par la propriété 1) du théorème,
\(𝑓_1(0)=𝑎=𝑓_2(0)\) donc \(𝐹\) contient \(0\). Par la propriété 2),
\begin{equation*}
𝑓_1(𝑛)=𝑓_2(𝑛)⇒𝐻\bigl(𝑛,𝑓_1(𝑛)\bigr)=𝐻\bigl(𝑛,𝑓_2(𝑛)\bigr)⇒𝑓_1(𝑛₊)=𝑓_2(𝑛₊)
\end{equation*}
et \(𝐹\) contient le successeur de chacun de ses éléments. Par l'axiome de récurrence, \(𝐹\) vaut \(ℕ\) et 
\(𝑓_1=𝑓_2\).
\item Existence : par construction du graphe. On considère les parties \(𝑀\) de \(ℕ×𝐸\) qui ont les deux propriétés
suivantes :
\begin{itemize}
\item
\(
(0,𝑎)∈𝑀
\)
\item
\((𝑛,𝑦)∈𝑀⇒\bigl(𝑛₊,𝐻(𝑛,𝑦)\bigr)∈𝑀\). 
\end{itemize}
\(ℕ×𝐸\) est un tel ensemble. L'intersection \(𝛤\) de tous ces ensembles a encore ces propriétés et c’est le plus
petit sous-ensemble de \(ℕ×𝐸\) qui les a. Montrons que c'est le graphe d'une application.

On considère
\(𝐹\mybydef{=}\bigl\{𝑛∈ℕ\mathbin|∃𝑦∈𝐸,(𝑛,𝑦)∈𝛤\bigr\}\). 
\(𝐹\) contient \(0\) puisque \((0,𝑎)∈𝛤\).
Si \(𝐹\) contient \(𝑛\), alors \((𝑛,𝑦)∈𝛤\) pour un certain \(𝑦\) de \(𝐸\) et par la propriété 2) on a
\(\bigl(𝑛₊,𝐻(𝑛,𝑦)\bigr)∈𝛤\), donc \(𝐹\) contient \(𝑛₊\). Ainsi, \(𝐹\) vaut \(ℕ\) et \(∀𝑛∈ℕ,\ ∃𝑦∈𝐸,\ (𝑛,𝑦)∈𝛤\).

On considère maintenant
\begin{equation*}
𝐹\mybydef{=}\bigl\{𝑛∈ℕ\mathbin|∃𝑦,𝑧∈𝐸,𝑦≠𝑧,(𝑛,𝑦)∈𝛤,(𝑛,𝑧)∈𝛤\bigr\}.
\end{equation*}
Si \(𝐹\) contient \(0\), on a \((0,𝑥)∈𝛤\) avec \(𝑥≠𝑎\). \(𝛤∖\{(0,𝑥)\}\) a les deux propriétés ci-dessus et il est
strictement inclus dans \(𝛤\), ce qui contredit la minimalité de \(𝛤\). Donc \ \(𝐹\) ne contient pas 0.

Si \(𝐹\) n'est
pas vide, il admet un élément minimal non nul \(𝑝\). Comme \(𝑝₋\) n'est pas dans \(𝐹\), il existe un unique \(𝑡\) 
tel que \((𝑝₋,𝑡)∈𝛤\). Comme \(𝑝\) et dans \(𝐹\), on a \((𝑝,𝑥)∈𝛤\) avec \(𝑥≠𝐻(𝑝₋,𝑡)\). Puisque 
\((0,𝑎)≠(𝑝,𝑥)\), \(𝛤∖\{(𝑝,𝑥)\}\) contient \((0,𝑎)\). De plus, si \ \(𝛤∖\{(𝑝,𝑥)\}\) contient \((𝑛,𝑦)\) alors
on a
%
\begin{equation*}
\bigl(𝑛₊,𝐻(𝑛,𝑦)\bigr)=(𝑝,𝑧)⇒
\begin{cases}
𝑛=𝑝₋
\\𝑦=𝑡
\\𝑧≠𝑥
\end{cases}
\end{equation*}
c'est-à-dire \(𝛤∖\{(𝑝,𝑥)\}\) contient \(bigl(𝑛₊,𝐻(𝑛,𝑦)\bigr)\). \(𝛤∖\{(𝑝,𝑥)\}\) a les deux propriétés ci-dessus et
il est strictement inclus dans \(𝛤\), cela contredit la minimalité de \(𝛤\).

Donc \(𝐹\) est vide et \(𝛤\) est le
graphe d’une application \(𝑓\) de \(ℕ\) dans \(𝐸\) pour laquelle on a bien \(𝑓(0)=𝑎\), \(𝑓(𝑛₊)=𝐻(𝑛,𝑓(𝑛))\) 
pour chaque entier \(𝑛\).\qedhere
\end{enumerate}
\end{proof}
%
\begin{theorem}
[Intervalles entiers]
Il existe une unique application de \(ℕ\) dans \( 𝒫(ℕ)\), notée \(𝑛↦⟦0;𝑛⟧\) telle que
%
\begin{enumerate}
\item \(⟦0;0⟧=\{0\}\),
\item \(∀𝑛∈ℕ,\ ⟦0;𝑛₊⟧=⟦0;𝑛⟧∪\{𝑛₊\}\).
\end{enumerate}
\end{theorem}
%
\begin{proof}
On applique le théorème de définition par récurrence à \(𝑎=\{0\}\), \(𝐸= 𝒫(ℕ)\) et 
\(𝐻(𝑛,𝑦)\mybydef{=}𝑦∪\{𝑛₊\}\).
\end{proof}
%
\begin{lemma}
%
\(
ℕ=\bigcup_{𝑛=0}^∞⟦0;𝑛⟧
\)
\end{lemma}
%
\begin{proof}
C'est immédiat.
\end{proof}
\subsection{Addition des entiers naturels}
\subsubsection{Définition}
%
\begin{theorem} 
\label{thm.N.addition0}
Pour tout entier naturel \(𝑛\), il existe une unique application de \(ℕ\) dans \(ℕ\), notée \(𝑝↦+_{𝑛}(𝑝)\) telle que
%
\begin{enumerate}
\item \(+_{𝑛}(0)=𝑛\),
\item \(∀𝑝∈ℕ,\ +_{𝑛}(𝑝₊)=\bigl(+_{𝑛}(𝑝)\bigr)₊\).
\end{enumerate}
\end{theorem}
%
\begin{proof}
C'est une définition par récurrence.
À compléter.
\end{proof}
\begin{proposition}
\par\noindent
\begin{enumerate}
\item
\(∀𝑝∈ℕ,\ +_{0}(𝑝)=𝑝\),
\item
\(∀𝑝∈ℕ,\ +_{1}(𝑝)=𝑝₊=p+1\),
\end{enumerate}
\end{proposition}
\begin{proof}
À compléter.
\end{proof}
%
\begin{definition}
[addition des entiers naturels]
\label{seq:refDefinition6} 
%
\begin{equation*}
∀𝑛,𝑝∈ℕ,\ 𝑛+𝑝\mybydef{=}+_{𝑛}(𝑝)
\end{equation*}
\end{definition}
%
\subsubsection{Propriétés}
%
\begin{theorem}
\par\noindent
%
\begin{enumerate}
\item
Élément neutre : \(∀𝑛∈ℕ,𝑛+0=0+𝑛=𝑛\) 
\item
Successeur : \(∀𝑛∈ℕ,𝑛₊=𝑛+1=1+𝑛\) 
\item
Associativité : \(∀𝑝,𝑞,𝑟∈ℕ,\ (𝑝+𝑞)+𝑟=𝑝+(𝑞+𝑟)\) 
\item
Commutativité : \(∀𝑝,𝑞∈ℕ,\ 𝑝+𝑞=𝑞+𝑝\) 
\item %
\(
∀𝑝,𝑞∈ℕ,\ 𝑝+𝑞=0⟺𝑝=𝑞=0
\)
\item Régularité : \(∀𝑝,𝑞,𝑟∈ℕ,\ 𝑝+𝑟=𝑞+𝑟⟹𝑝=𝑞\) 
\end{enumerate}
\end{theorem}
\begin{remark}
\vspace{-\baselineskip}
\begin{itemize}
\item
Par 1), 2), 4) et 6), \((ℕ,+)\) est un \mykeyword{monoïde régulier commutatif}.
\item
Par 2), la notation \ref{ntn.N.successeur} est consistante avec la definition \ref{def.N.addition}
\end{itemize}
\end{remark}
%
\begin{proof}
\par\noindent
%
\begin{enumerate}
\item Par le theorème \ref{thm.N.addition0}, pour tout entier naturel \(𝑛\), on a
 \(𝑛+0=𝑛\). L'autre égalité se montre par récurrence.
 On a \(0+0=0\) et
 \(0+𝑛=𝑛⇒𝑛₊=(0+𝑛)₊=0+𝑛₊\)
%
\item Pour la première égalité, \(𝑛₊=(𝑛+0)₊=𝑛+(0₊)=𝑛+1\). Par récurrence pour la deuxième, \(0+1=1+0=1\) et
%
\begin{equation*}
𝑛+1=1+𝑛⇒(𝑛+1)+1=(1+𝑛)+1=1+(𝑛+1)
\end{equation*}
\item Par récurrence sur \(𝑟\). Initialisation : \((𝑝+𝑞)+0=𝑝+𝑞=𝑝+(𝑞+0)\). Hérédité : si \ on a 
\((𝑝+𝑞)+𝑟=𝑝+(𝑞+𝑟)\), alors
%
\begin{equation*}
(𝑝+𝑞)+𝑟₊=\bigl((𝑝+𝑞)+𝑟\bigr)₊=\bigl(𝑝+(𝑞+𝑟)\bigr)₊=𝑝+(𝑞+𝑟)₊=𝑝+(𝑞+𝑟₊)
\end{equation*}
\item Par récurrence sur \(𝑝\) à \(𝑛\) fixé. Initialisation : voir 1) ci-dessus. Hérédité : si on a \(𝑛+𝑝=𝑝+𝑛\). alors
%
\begin{equation*}
𝑛+(𝑝+1)=(𝑛+𝑝)+1=(𝑝+𝑛)+1=𝑝+(𝑛+1)=𝑝+(1+𝑛)=(𝑝+1)+𝑛
\end{equation*}
\item \(0+0=0\) donne un sens de l'équivalence. Par ailleurs, si \(𝑞\) est un successeur, \(𝑝+𝑞\) aussi. Par
contraposition, si \(𝑝+𝑞\) est nul alors \(𝑞\) aussi. Il en va de même pour \(𝑝\) par commutativité.
\item Par récurrence sur \(𝑟\). Initialisation : immédiat. Hérédité : essentiellement

 \(𝑝+(𝑟+1)=𝑞+(𝑟+1)⇒(𝑝+𝑟)+1=(𝑞+𝑟)+1⇒𝑝+𝑟=𝑞+𝑟\).
\end{enumerate}
\end{proof}
%
\begin{proposition} 
On obtient une égalité vraie en ajoutant membre à membre des égalités vraies.
\end{proposition}
%
\begin{proof}
Par le \emph{principe de Leibnitz},
\begin{equation*}
\begin{cases}
𝑎=𝑏\\𝑐=𝑑\\𝑎+𝑐=𝑎+𝑐
\end{cases}⇒\begin{cases}
𝑎=𝑏\\𝑎+𝑐=𝑎+𝑑
\end{cases}⇒𝑎+𝑐=𝑏+𝑑
\end{equation*}
\end{proof}
%
\subsection[Ordre]{Ordre}
\subsubsection{Définitions}
%
\begin{definition}
\label{dfn.N.ordre}
Pour tous entiers \(𝑝\) et \(𝑞\) 
%
\begin{enumerate}
\item \(𝑝⩽𝑞\) signifie \(∃𝛿∈ℕ,\ 𝑝+𝛿=𝑞\),
\item \(𝑝<𝑞\) signifie en plus \(𝛿≠0\),
\item \(𝑞⩾𝑝\) signifie \(𝑝⩽𝑞\) et
\item \(𝑞>𝑝\) signifie \(𝑝<𝑞\).
\end{enumerate}
\begin{itemize}
\item
1) et 3) sont des \mykeyword{inégalités larges}, les autres sont des \mykeyword{inégalités
strictes}.
\item
1) et 2) sont des inégalités de \mykeyword{sens croissant}, les deux autres sont des inégalités de
\mykeyword{sens décroissant}.
\item
\(𝑝\) \mykeyword{minore} \(𝑞\), \(𝑝\) \mykeyword{est minorant de} \(𝑞\) sont des
synonymes de \(𝑝⩽𝑞\).
\item
\(𝑝\) \mykeyword{majore} \(𝑞\), \(𝑝\) \mykeyword{est majorant de} \(𝑞\) sont des
synonymes de \(𝑝⩾𝑞\).
\item
\(𝑝⩽𝑞⩽𝑟\) signifie \(𝑝⩽𝑞\myand𝑞⩽𝑟\). De même avec des inégalités strictes, et avec des inégalités dans
l'autre sens.
\end{itemize}
\end{definition}
%
\begin{definition}
\par\noindent
\begin{itemize}
\item
\(𝑝\) est \mykeyword{positif ou nul} signifie \(0⩽𝑝\).
\item
\(𝑝\) est \mykeyword{positif} signifie \(0<𝑝\).
\item
\(𝑝\) est \mykeyword{négatif ou nul} signifie \(𝑝⩽0\).
\item
\(𝑝\) est \mykeyword{négatif} signifie \(𝑝<0\).
\end{itemize}
\end{definition}
\subsubsection{Propriétés}
%
\begin{proposition} 
Tout entier naturel est positif ou nul.
\end{proposition}
%
\begin{proof}
Pour tout entier naturel \(𝑝\), on a \(0+𝑝=𝑝\), c'est la définition de \(0⩽𝑝\) avec \(𝛿=𝑝\).
\end{proof}
%
\begin{theorem} 

Pour tous entiers naturels \(𝑝\) et \(𝑞\), \(𝑝⩽𝑝+𝑞\). Si en plus \(𝑞\) n'est pas nul, \(𝑝<𝑝+𝑞\). En particulier 
\(𝑝<𝑝+1\) 
\end{theorem}
%
\begin{proof}
\item 

On a \(𝑝+𝑞=𝑝+𝑞\), c'est la définition de \(𝑝⩽𝑝+𝑞\) ou \(𝑝<𝑝+𝑞\) avec \(𝛿=𝑞\).
\end{proof}
%
\begin{theorem} 

Pour tous entiers \(𝑝\) et \(𝑞\),
%
\begin{enumerate}
\item %
\(
𝑝⩽𝑞⟺𝑝<𝑞\text{ ou }𝑝=𝑞
\)
\item
\(𝑝<𝑞⟺∃𝛿∈ℕ,\ 𝑝+𝛿+1=𝑞\).
\end{enumerate}
\end{theorem}
%
\begin{proof}
\par\noindent
\begin{enumerate}
\item
On a par définitions \(𝑝⩽𝑞⟸𝑝<𝑞\myor𝑝=𝑞\). De plus,
%
\begin{align*}
𝑝⩽𝑞\myand𝑝≠𝑞
&{}⟹∃𝛿∈ℕ,\ 𝑝+𝛿=𝑞,\ 𝑝≠𝑞
\\&{}⟹
∃𝛿∈ℕ,\ (𝛿=0,\ 𝑝+𝛿=𝑞,\ 𝑝≠𝑞)
\\
&\hspace{8em}\myor(𝛿≠0,\ 𝑝+𝛿=𝑞,\ 𝑝≠𝑞)
\\&{}⟹
∃𝛿∈ℕ,\ (𝛿=0,\ 𝑝=𝑞,\ 𝑝≠𝑞)
\\
&\hspace{8em}\myor(𝛿≠0,\ 𝑝+𝛿=𝑞,\ 𝑝≠𝑞)
\\&{}⟹
∃𝛿∈ℕ,\ 𝛿≠0,\ 𝑝+𝛿=𝑞
\\&{}⟹
𝑝<𝑞
\end{align*}
\item On a
\end{enumerate}
%
\begin{align*}
\lefteqn{∃𝛿∈ℕ,\ 𝛿≠0,\ 𝑝+𝛿=𝑞}
\hspace{5em}
\\&{}⟺
∃𝛿∈ℕ,\ ∃𝛿'∈ℕ,\ 𝛿=𝛿'+1,\ 𝑝+𝛿=𝑞
\\&{}⟺
∃𝛿∈ℕ,\ ∃𝛿'∈ℕ,\ 𝛿=𝛿'+1,\ 𝑝+(𝛿'+1)=𝑞
\\&{}⟺
∃𝛿'∈ℕ,\ 𝑝+(𝛿'+1)=𝑞,\ ∃𝛿∈ℕ,\ 𝛿=𝛿'+1
\\&{}⟺
∃𝛿'∈ℕ,\ 𝑝+(𝛿'+1)=𝑞
\end{align*}
\end{proof}
%
\begin{lemma} 
Pour tout entier naturel \(𝑝\), on a \(𝑝⩽𝑝\).
\end{lemma}
%
\begin{proof}
\item 

On a \(𝑝+0=𝑝\), c'est la définition de \(𝑝⩽𝑝\) avec \(𝛿=0\).
\end{proof}
%
\begin{theorem} 

Pour tous entiers naturels \(𝑝\) et \(𝑞\), on a \(𝑝⩽𝑞 \myand𝑞⩽𝑝⟹𝑝=𝑞\).
\end{theorem}
%
\begin{proof}
\item 

%
\begin{equation*}
%
\begin{matrix}𝑝⩽𝑞 \myand𝑞⩽𝑝&⟹&∃𝛿∈ℕ,\ 𝑝+𝛿=𝑞 \myand∃𝛿'∈ℕ,\ 𝑞+𝛿'=𝑝
\\&⟹&∃𝛿,𝛿'∈ℕ,\ 𝑝+𝛿=𝑞,\ (𝑝+𝛿)+𝛿'=𝑝\\&⟹&∃𝛿,𝛿'∈ℕ,\ 𝑝+𝛿=𝑞,\ 𝑝+(𝛿+𝛿')=𝑝\\&⟹&∃𝛿,𝛿'∈ℕ,\ 𝑝+𝛿=𝑞,\ 𝛿+𝛿'=0\\&⟹&∃𝛿,𝛿'∈ℕ,\ 𝑝+𝛿=𝑞,\ 𝛿=𝛿'=0\\&⟹&∃𝛿,𝛿'∈ℕ,\ 𝑝+0=𝑞,\ 𝛿=𝛿'=0\\&⟹&𝑝+0=𝑞,\ ∃𝛿,𝛿'∈ℕ,\ 𝛿=𝛿'=0\\&⟹&𝑝=𝑞
\end{matrix}
\end{equation*}
\end{proof}
%
\begin{theorem} 

Pour tous entiers naturels \(𝑝\), \(𝑞\) et \(𝑟\), on a

 \(𝑝⩽𝑞 \myand𝑞⩽𝑟⟹𝑝⩽𝑟\).
\end{theorem}
%
\begin{proof}
\begin{align*}
\lefteqn{𝑝⩽𝑞 \myand𝑞⩽𝑟 \myand𝑟⩽𝑝}
\hspace{5em}&
\\&{}⟹∃𝛿,𝛿',𝛿''∈ℕ,\ 𝑝+𝛿=𝑞 \myand𝑞+𝛿'=𝑟 \myand𝑟+𝛿''=𝑝
\\&{}⟹
∃𝛿,𝛿',𝛿''∈ℕ,\
\begin{cases}
\bigl((𝑝+𝛿)+𝛿'\bigr)+𝛿''=𝑝
\\
𝑝+𝛿=𝑞 \myand𝑞+𝛿'=𝑟 \myand𝑟+𝛿''=𝑝
\end{cases}
\end{align*}
Or,
%
\begin{align*}
%
((𝑝+𝛿)+𝛿')+𝛿''=𝑝
&{}⟹
𝑝+((𝛿+𝛿')+𝛿'')=𝑝
\\&{}⟹((𝛿+𝛿')+𝛿'')=0
\\&{}⟹
𝛿+𝛿'=0,\ 𝛿''=0
\\&{}⟹
𝛿=𝛿'=𝛿''=0
\end{align*}
Cela donne directement le résultat après
%
\begin{align*}
%
\lefteqn{𝑝⩽𝑞 \myand𝑞⩽𝑟 \myand𝑟⩽𝑝}
\hspace{5em}
\\&{}⟹
∃𝛿,𝛿',𝛿''∈ℕ,\%
\begin{cases}
𝛿=𝛿'=𝛿''=0
\\𝑝+𝛿=𝑞 \myand𝑞+𝛿'=𝑟 \myand𝑟+𝛿''=𝑝
\end{cases}
\\&{}⟹
∃𝛿,𝛿',𝛿''∈ℕ,\ %
\begin{cases}
𝛿=𝛿'=𝛿''=0
\\
𝑝=𝑞\myand𝑞=𝑟\myand𝑟=𝑝
\end{cases}
\end{align*}
\end{proof}
%
\begin{lemma} 
Pour tous entiers naturels \(𝑝\), \(𝑞\) et \(𝑟\), on a \(𝑝⩽𝑞 \myand𝑞⩽𝑟 \myand𝑟⩽𝑝⟹𝑝=𝑞=𝑟\) 
\end{lemma}
%
\begin{proof}
\begin{align*}
%
𝑝⩽𝑞 \myand𝑞⩽𝑟 \myand𝑟⩽𝑝
&{}⟹
𝑝⩽𝑞 \myand𝑞⩽𝑟 \myand𝑝⩽𝑟 \myand𝑟⩽𝑝
\\&{}⟹
𝑝⩽𝑞 \myand𝑞⩽𝑟 \myand𝑝=𝑟
\\&{}⟹
𝑝⩽𝑞 \myand𝑞⩽p \myand𝑝=𝑟
\\&{}⟹
𝑝=𝑞 \myand𝑝=𝑟
\end{align*}
\end{proof}
%
%
\begin{lemma} 
Pour tous entiers naturels \(𝑝\), \(𝑞\) et \(𝑟\), on a
\begin{itemize}
\item
\(𝑝<𝑞 \myand𝑞⩽𝑟⟹𝑝<𝑟\),
\item
\(𝑝⩽𝑞 \myand𝑞<𝑟⟹𝑝<𝑟\).
\end{itemize}
\end{lemma}
%
\begin{proof}
\begin{align*}
𝑝<𝑞 \myand𝑞⩽𝑟
&{}⟹
𝑝<𝑞 \myand𝑝⩽𝑞 \myand𝑞⩽𝑟
\\&{}⟹
𝑝<𝑞 \myand𝑞⩽𝑟\myand𝑝⩽𝑟
\\&{}⟹
𝑝<𝑞 \myand𝑞⩽𝑟 \myand𝑝=𝑟
\myor
𝑝<𝑞 \myand𝑞⩽𝑟 \myand𝑝<𝑟
\\&{}⟹
𝑟<𝑞\myand𝑞⩽𝑟
\myor
𝑝<𝑟
\\&{}⟹
𝑟≠𝑞 \myand𝑟⩽𝑞 \myand𝑞⩽𝑟
\myor𝑝<𝑟
\\&{}⟹
𝑟≠𝑞 \myand𝑟=𝑞
\myor
𝑝<𝑟
\\&{}⟹
𝑝<𝑟
\end{align*}
\end{proof}
%
\begin{lemma} 
Pour tous entiers \(𝑝\) et \(𝑞\), on a \(𝑝<𝑞\) ou bien \(𝑝=𝑞\) ou bien \(𝑞<𝑝\).
\end{lemma}
%
\begin{proof}
On montre par récurrence que pour tout entier naturel \(𝑝\),
on a
\begin{equation*}
∀𝑞∈ℕ,\ 𝑝<𝑞\myor𝑝=𝑞\myor𝑞<𝑝.
\end{equation*}
\begin{itemize}
\item
Initialisation : sachant que tout entier est positif ou nul, avec le lemme précédent, nous avons
\(∀𝑞∈ℕ,\ 0<𝑞\myor0=𝑞\).
\item
Hérédité : on a d'après ce qui précède
\begin{gather*}
𝑝<𝑞⇒𝑝+1⩽𝑞⇒(𝑝+1<𝑞\myor𝑝+1=𝑞),
\\
𝑝=𝑞⇒𝑝+1>𝑞
\\
𝑞<𝑝⇒𝑞<𝑝<𝑝+1⇒𝑞<𝑝+1
\end{gather*}
On a le résultat par combinaison de \myor\ par rapport à \(⇒\).
\qedhere
\end{itemize}
\end{proof}\footnote{\(\bigl((A⇒B)\myor(C⇒D)\bigr)⟹\bigl((A\myor C)⇒(B\myor D)\bigr)\)}
%
\begin{theorem} 
\(⩽\) et \(⩾\) sont des relations d'ordre total.
\end{theorem}
%
\begin{proof}
\item 

Voir ce qui précède.
\end{proof}
%
\begin{theorem} 

Pour tous entiers naturels \(𝑝\) et \(𝑞\),
\begin{equation*}
(𝑝<𝑞)⟺(𝑝+1⩽𝑞)⟺(𝑝⩽𝑞₋)
\end{equation*}
\end{theorem}
%
\begin{proof}
\begin{align*}
𝑝<𝑞
&{}\mybydef{⟺}
∃𝛿∈ℕ^{\ast},\ 𝑝+𝛿=𝑞
\\&{}⟺\vphantom{\mybydef{⟺}}
∃𝛿'∈ℕ,\ 𝑝+(𝛿'+1)=𝑞
\\&{}⟺\vphantom{\mybydef{⟺}}
∃𝛿'∈ℕ,\ (𝑝+1)+𝛿'=𝑞
\\&{}\mybydef{⟺}
𝑝+1⩽𝑞
\end{align*}
\begin{align*}
∃𝛿'∈ℕ,\ 𝑝+(𝛿'+1)=𝑞
&{}⇔
∃𝛿'∈ℕ,\ (𝑝+𝛿')+1=𝑞
\\&{}⇔
∃𝛿'∈ℕ,\ 𝑝+𝛿'=𝑞₋
\\&{}⇔
𝑝⩽𝑞₋.
\end{align*}
\end{proof}
%
\begin{theorem}
\label{seq:refTheorem26} 
Pour tout entier naturel \(𝑛\),
\begin{equation*}
⟦0~;~𝑛⟧=\bigl\{𝑘∈ℕ\mathbin|0⩽𝑘⩽𝑛\bigr\}.
\end{equation*}
\end{theorem}
%
\begin{proof}
On a \(⟦0~;~0⟧=\bigl\{0\bigr\}=\bigl\{𝑘∈ℕ\mathbin|0⩽𝑘⩽0\bigr\}\) sachant que \(0⩽𝑘⩽0⇔𝑘=0\). Pour tout \(𝑘\)
\begin{gather*}
𝑘⩽𝑛+1⇔𝑘<n+1\myor𝑘=𝑛+1⇔𝑘⩽𝑛\myor𝑘=𝑛+1,
\end{gather*}
\begin{align*}
\bigl\{𝑘∈ℕ\mathbin|0⩽𝑘⩽𝑛+1\bigr\}
&{}=
\bigl\{𝑘∈ℕ\mathbin|0⩽𝑘\myand(𝑘⩽𝑛\myor𝑘=𝑛+1)\bigr\}
\\&{}=
\bigl\{𝑘∈ℕ\mathbin|0⩽𝑘⩽𝑛\bigr\}∪\bigl\{𝑛+1\bigr\}
\end{align*}
Cela fait deux suites définies par la même récurrence : elles sont égales.
\end{proof}
%
\begin{definition} 

Soit \(𝐴\) une partie de \(ℕ\).
%
\begin{enumerate}
\item \(𝑚\) \mykeyword{minore} \(𝐴\), \(𝑚\) est un \mykeyword{minorant} de \(𝐴\),
signifient que \(𝑚\) \mykeyword{minore} tout élément de \(𝐴\).
\item \(𝑚\) \mykeyword{majore} \(𝐴\), \(𝑚\) est un \mykeyword{majorant }de \(𝐴\),
signifient que \(𝑚\) \mykeyword{majore} tout élément de \(𝐴\).
\item \(𝑚\) est un \mykeyword{plus petit élément} de \(𝐴\), \(𝑚\) est un
\mykeyword{minimum} de \(𝐴\), signifient que \(𝑚\) est un minorant de \(𝐴\) appartenant à \(𝐴\).
\item \(𝑚\) est un \mykeyword{plus grand élément} de \(𝐴\), \(𝑚\) est un
\mykeyword{maximum} de \(𝐴\), signifient que \(𝑚\) est un majorant de \(𝐴\) appartenant à \(𝐴\).
\item \(𝐴\) est \mykeyword{majorée} si elle admet un majorant.
\end{enumerate}
\end{definition}
%
\begin{lemma}
Pour tout entier naturel \(𝑛\), \(⟦0;𝑛⟧\) a pour maximum \(𝑛\). \(ℕ\) n'est pas majoré.
\end{lemma}
%
\begin{proof}
Immédiat par la proposition \ref{seq:refTheorem26} ou les définitions.
\end{proof}
%
\begin{theorem} 
%
\begin{enumerate}
\item Un minorant d'un minorant est un minorant.
\item Un majorant d'un majorant est un majorant.
\item Un minorant de \(𝐴\) minore tout partie de \(𝐴\).
\item Un majorant de \(𝐴\) majore tout partie de \(𝐴\).
\item Quand il existe, un minimum est unique. \emph{Idem} pour un maximum.
\end{enumerate}
\end{theorem}
%
\begin{proof}
\par\noindent
\begin{enumerate}
\item Immédiat.
\item Immédiat.
\item Immédiat.
\item Immédiat.
\item Si \(𝑚\) et \(𝑚'\) sont deux minima, on a \(𝑚⩽𝑚'⩽𝑚\), donc \(𝑚=𝑚'\). \emph{Idem} dans l'autre
sens.
\end{enumerate}
\end{proof}
%
\begin{theorem} 
Toute partie non vide de \(ℕ\) admet un minimum unique.
\end{theorem}
%
\begin{proof}
\par\noindent
\begin{itemize}
\item
Unicité : immédiat.
\item
Existence.
Soit \(𝐴\) une partie de \(ℕ\) contenant \(𝑎\) et \(𝑀\) l'ensemble de ses minorants. \(𝑀\) contient 0. Si \(𝑀\) 
contenait le successeur de chacun de ses éléments, ce serait \(ℕ\) et il contiendrait \(𝑎+1\) qui ne minore pas \(𝑎\)
. Donc il existe un entier naturel \(𝑚\) dans \(𝑀\) tel que \(𝑚+1\) n'est pas dans \(𝑀\). Or 
\begin{align*}
𝑚∈𝑀 \myand𝑚+1∉𝑀
&{}⟹
𝑚∈𝑀 \myand∃𝑝∈𝐴,\ 𝑚+1>𝑝
\\&{}⟹
∃𝑝∈𝐴,\ 𝑚⩾𝑝 \myand𝑚⩽𝑝
\\&{}⟹
∃𝑝∈𝐴,\ 𝑚=𝑝\hfill
\\&{}⟹
𝑚∈𝐴
\end{align*}
Donc \(𝑚\) est un minimum de \(𝐴\).\qedhere
\end{itemize}
\end{proof}
%
\begin{theorem} 
Toute partie non vide majorée de \(ℕ\) admet un maximum unique.
\end{theorem}
%
\begin{proof}
Soit \(𝐴\) une partie de \(ℕ\) contenant \(𝑎\) et majorée par \(𝑚\), \(𝑀\) l'ensemble de ses majorants. \(𝑀\),
contenant \(𝑚\), n'est pas vide : soit \(𝑚'\) son minimum. Si \(𝑚'\) n'appartient pas à \(𝐴\), cela signifie que 
\(𝑎<𝑚'\), ou \(𝑎⩽𝑚'₋\), et de même pour tout autre élément de \(𝐴\). \(𝑚'₋\) est un maximum de \(𝐴\) qui
contredit la minimalité de \(𝑚'\). Ainsi, \(𝑚'\) est dans \(𝐴\) : c'est un maximum.
\end{proof}
%
\begin{definition}
[Intervalles entiers naturels]
Pour tous entiers naturels \(𝑝\) et \(𝑞\), on pose
\begin{itemize}
\item
\(⟦𝑝~;𝑞⟧\mybydef{=}\bigl\{𝑛∈ℕ\mathbin|𝑝⩽𝑛⩽𝑞\bigr\}\),
\item
\(⟦𝑝~;𝑞⟦\mybydef{=}\bigl\{𝑛∈ℕ\mathbin|𝑝⩽𝑛<𝑞\bigr\}\),
\item
\(⟧𝑝~;𝑞⟧\mybydef{=}\bigl\{𝑛∈ℕ\mathbin|𝑝<𝑛⩽𝑞\bigr\}\),
\item
\(⟧𝑝~;𝑞⟦\mybydef{=}\bigl\{𝑛∈ℕ\mathbin|𝑝<𝑛<𝑞\bigr\}\),
\item
\(⟦𝑝~;∞⟦\mybydef{=}\bigl\{𝑛∈ℕ\mathbin|𝑝⩽𝑛\bigr\}\),
\item
\(⟧𝑝~;∞⟦\mybydef{=}\bigl\{𝑛∈ℕ\mathbin|𝑝<𝑛\bigr\}\).
\end{itemize}
\end{definition}
%
\begin{remark}
\(⟦𝑝;𝑝⟧=\bigl\{𝑝\bigr\}\), \(⟦𝑝;𝑝⟦=∅\)...
\end{remark}
%
\begin{theorem}
[Récurrence décalée]
Si un sous-ensemble de \(ℕ\) contient un entier naturel \(𝑛\) et contient le successeur de chacun de ses éléments, alors elle
contient \(⟦𝑛~;∞⟦\).
\end{theorem}
%
\begin{proof}
Notons cette partie \(𝐴\). Par l'absurde, si \(⟦𝑛~;∞⟦∖𝐴\) n'est pas vide, soit \(𝑚\) son minimum. On a \(𝑛⩽𝑚\), 
\(𝑛≠𝑚\) qui garantit \(𝑚>0\). Par minimalité, \(𝑚-1∉⟦𝑛~;∞⟦∖𝐴\), donc \(𝑚-1∈𝐴\) et \(𝑚∈𝐴\) par succession.
Contradiction.
\end{proof}
%
\begin{theorem}
[Récurrence finie]
Si \(𝑢_0=𝑣_0\) et si pour tout \(0⩽𝑖<𝑛\), on a \(𝑢_{𝑖}=𝑣_i⇒𝑢_{𝑖+1}=𝑣_{i+1}\) alors pour tout \(0⩽𝑖⩽𝑛\), on a 
\(𝑢_{𝑖}=𝑣_i\).
\end{theorem}
%
\begin{proof}
Par récurrence sur \(𝑛\), en exercice...
\end{proof}
\subsection{Soustraction}
\subsubsection{Définition}
%
\begin{theorem} 
Pour tout entier naturel \(𝑛\), \(+_{𝑛}\) est une bijection de ℕ sur \(⟦𝑛~;∞⟦\) (voir \ref{seq:refDefinition6}).
\end{theorem}
%
\begin{proof}
L'injectivité vient de la régularité de l'addition. La surjectivité vient de la définition de l'ordre.
\end{proof}
%
\begin{definition}
[Soustraction des entiers naturels] \(∀𝑝,𝑞∈ℕ,\ 𝑞⩾𝑝\),
\begin{equation*}
𝑞-𝑝\mybydef{=}+_{𝑝}^{-1}(𝑞).
\end{equation*}
\end{definition}
\subsubsection{Propriétés}
%
\begin{theorem} 
\begin{equation*}
∀𝑛∈ℕ,\ 𝑛₋=𝑛-1
\end{equation*}
\end{theorem}
%
\begin{proof}
C'est une réécriture de la définition du prédécesseur.
\end{proof}
%
\begin{theorem}
\par\noindent
\begin{itemize}
\item
\(∀𝑝,𝑞∈ℕ,\ 𝑞⩾𝑝,\ (𝑞-𝑝)+𝑝=𝑞\)
\item
\(∀𝑝,𝑞∈ℕ,\ (𝑝+𝑞)-𝑞=𝑝\)
\end{itemize}
\end{theorem}
%
\begin{proof}
C'est une réécriture de \(+_{𝑝}○+_{𝑝}^{-1}\) et \(+_{𝑞}^{-1}○+_{𝑞}\).
\end{proof}
%
\subsection[Multiplication]{Multiplication}
\subsubsection{Définition}
%
\begin{lemma} 
Pour tout entier naturel \(𝑛\), il existe une unique application de \(ℕ\) dans \(ℕ\), notée \(𝑝↦×_𝑛(𝑝)\) telle que
%
\begin{enumerate}
\item \(×_{𝑛}(0)=0\),
\item \(∀𝑝∈ℕ,\ ×_{𝑛}(𝑝₊)=×_{𝑛}(𝑝)+𝑛\).
\end{enumerate}
\end{lemma}
%
\begin{proof}
C'est une définition par récurrence.
À compléter.
\end{proof}
%
\begin{definition}
[multiplication des entiers naturels]
%
\begin{equation*}
∀𝑛,𝑝∈ℕ,\ 𝑛×𝑝\mybydef{=}×_{𝑛}(𝑝)
\end{equation*}
\end{definition}
\begin{remark}
On pourra omettre le signe × s'il n'y a pas d'ambiguïté.
\end{remark}
\begin{attention}
La multiplication est prioritaire devant l'addition.
\end{attention}
%
\subsubsection{Propriétés}
%
\begin{theorem}
\par\noindent
%
\begin{enumerate}
\item Élément absorbant : \(∀𝑛∈ℕ,\ 𝑛×0=0×𝑛=0\) 
\item Élément neutre : \(∀𝑛∈ℕ,\ 𝑛×1=𝑛=1×𝑛\) 
\item Distributivités : \(
\begin{aligned}[t]
&∀𝑝,𝑞,𝑟∈ℕ,\ (𝑝+𝑞)×𝑟=𝑝×𝑟+𝑞×𝑟
\\
&∀𝑝,𝑞,𝑟∈ℕ,\ 𝑝×(𝑞+𝑟)=𝑝×𝑞+𝑝×𝑟
\end{aligned}
\)
\item Associativité : \(∀𝑝,𝑞,𝑟∈ℕ,\ (𝑝×𝑞)×𝑟=𝑝×(𝑞×𝑟)\) 
\item Commutativité : \(∀𝑝,𝑞∈ℕ,\ 𝑝×𝑞=𝑞×𝑝\) 
\item Intégrité : \(∀𝑝,𝑞∈ℕ,\ 𝑝×𝑞=0⇔𝑝=0\text{ ou }𝑞=0\) 
\item Régularité : \(∀𝑝,𝑞,𝑟∈ℕ,\ 𝑝×𝑟=𝑞×𝑟⟹𝑝=𝑞\) 
\end{enumerate}
\end{theorem}
%
\begin{proof}
\par\noindent
\begin{enumerate}
\item La première égalité est immédiate. La deuxième est prouvée par récurrence. Initialisation : immédiat. Hérédité :
%
\begin{equation*}
0×𝑛=0⟹0×(𝑛+1)=0×𝑛+0=0+0=0
\end{equation*}
\item On a \(𝑛×1=𝑛×0+𝑛=0+𝑛=𝑛\). Par récurrence pour la deuxième égalité. Initialisation : voir i). Hérédité :
%
\begin{equation*}
𝑛=1×𝑛⟹𝑛+1=1×𝑛+1=1×(𝑛+1)
\end{equation*}
\item Par récurrence sur \(𝑟\) pour prouver les deux égalités. Initialisation : \((𝑝+𝑞)×0=0=0+0=𝑝×0+𝑞×0\). Hérédité
: si \((𝑝+𝑞)𝑟=𝑝𝑟+𝑞𝑟\) 
%
\begin{align*}
(𝑝+𝑞)(𝑟+1)
&{}=
(𝑝+𝑞)𝑟+(𝑝+𝑞)
\\&{}=
(𝑝𝑟+𝑞𝑟)+(𝑝+𝑞)
\\&{}=
(𝑝𝑟+𝑝)+(𝑞𝑟+𝑞)
\\&{}=
𝑝(𝑟+1)+𝑞(𝑟+1)
\end{align*}
Initialisation : \(𝑝(𝑞+0)=𝑝𝑞=𝑝𝑞+0=𝑝𝑞+𝑝×0\). Hérédité : si \ \(𝑝(𝑞+𝑟)=𝑝𝑞+𝑝𝑟\),
%
\begin{enumerate}
\item[] %
\begin{align*}
𝑝(𝑞+(𝑟+1))
&{}=
𝑝((𝑞+𝑟)+1)
\\&{}=
𝑝(𝑞+𝑟)+𝑝
\\&{}=
(𝑝𝑞+𝑝𝑟)+𝑝
\\&{}=
𝑝𝑞+(𝑝𝑟+𝑝)
\\&{}=
𝑝𝑞+𝑝(𝑟+1)
\end{align*}
\end{enumerate}
\item Par récurrence sur \(𝑟\). Initialisation : on a \(𝑝×(𝑞×0)=𝑝×0=0=(𝑝×𝑞)×0\). Hérédité :
%
\begin{equation*}
(𝑝𝑞)×(𝑟+1)=(𝑝𝑞)×𝑟+(𝑝𝑞)=𝑝×(𝑞𝑟)+(𝑝𝑞)=𝑝×(𝑞𝑟+𝑞)=𝑝×(𝑞×(𝑟+1))
\end{equation*}
\item Par récurrence sur \(𝑞\). Initialisation : voir 1). Hérédité : si \(𝑝×𝑞=𝑞×𝑝\) alors
%
\begin{equation*}
𝑝×(𝑞+1)=𝑝×𝑞+𝑝=𝑞×𝑝+𝑝=𝑞×𝑝+1×𝑝=(𝑞+1)×𝑝
\end{equation*}
\item Déjà 1) donne un sens de l'équivalence, l'autre est montré par contraposition. Si \(𝑝\) et \(𝑞\) ne sont pas
nuls, ce sont des successeurs, donc on a
%
\begin{equation*}
𝑝×𝑞=(𝑝₋+1)×𝑞=𝑝₋×𝑞+𝑞=𝑝₋×𝑞+(𝑞₋+1)=(𝑝₋×𝑞+𝑞₋)+1
\end{equation*}
et leur produit est aussi un successeur : il n'est pas nul.
\item Si on a \(𝑝<𝑞\) alors \(𝑞=𝑝+𝛿₊\) et \(𝑞𝑟₊=(𝑝+𝛿₊)𝑟₊=𝑝𝑟₊+𝛿₊𝑟₊\) donc \(𝑞𝑟₊<𝑝𝑟₊\). Par
contraposition puis échange de \(𝑝\) et \(𝑞\) on obtient la régularité.
\end{enumerate}
\end{proof}
%
\begin{theorem}
[Puissances entières]
Pour tout entier naturel \(𝑝\), il existe une unique application de \(ℕ\) dans \(ℕ\), notée \(𝑛↦𝑝^{𝑛}\) telle que
%
\begin{enumerate}
\item \(𝑝^0=1\),
\item \(∀𝑛∈ℕ,\ 𝑝^{𝑛+1}=𝑝^{𝑛}×𝑝\).
\end{enumerate}
\end{theorem}
%
\begin{proof}
On applique le théorème de définition par récurrence à \(𝑎=1\), \(𝐸=ℕ\) et \(𝐻(𝑛,𝑦)\mybydef{=}𝑦×𝑝\).
\end{proof}
\subsection{Division}
Le problème de la division est traité dans le chapitre d'arithmétique.
