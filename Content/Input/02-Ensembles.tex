%!TEX program = xelatex
% !TEX root = ../Main/main.tex
% !TEX encoding = UTF-8

\section{Éléments de théorie des ensembles}
Vocabulaire de la théorie des ensembles.
\subsection{Égalité}
\begin{axiom}
[Principe de Leibnitz] \(𝖯(.)\) étant un prédicat d'une variable
\begin{equation*}
∀𝑥,𝑦\text{, }(𝑥=𝑦)⟹(𝖯(𝑥)⇔𝖯(𝑦))
\end{equation*}\end{axiom}
\begin{remark}
Cela fait un axiome par prédicat... pour pouvoir remplacer dans
n'importe quelle expression un objet par un autre qui lui est égal.
\end{remark}
%
\subsection{Éléments de théorie des ensembles}
\subsubsection{Axiomes ZFC (de Zermelo, Fraenkel et du choix)}
Pour la culture, voici une liste d'axiomes qui permettent de bâtir correctement la théorie des ensembles. À partir de
là, on peut construire l'ensemble des entiers naturels puis les autres ensembles de nombres, avec d'autres axiomes
éventuellement.

En général, la rédaction est descriptive puis symbolique. Au début, tous les objets sont des ensembles.
%
\begin{axiom}
[Extensionnalité]
\label{seq:refAxiom1}
Si deux ensembles ont les mêmes éléments, alors ils sont égaux.
\begin{equation*}
∀𝑥\;(∀𝑦\;((∀𝑧\;(𝑧∈𝑥⇔𝑧∈𝑦))⟹𝑥=𝑦))
\end{equation*}
\end{axiom}
\begin{remark}
C'est bien un axiome et pas une définition car les éléments étant eux-mêmes des
ensembles, parler des «~mêmes éléments~» utilise implicitement l'égalité des ensembles. La réciproque vient simplement
du principe de substitution.
\end{remark}
%
\begin{axiom}
[Ensemble vide]
Il existe un ensemble sans élément. Il est noté \(∅\).
\end{axiom}
%
\begin{lemma}
L'ensemble vide est unique.
\end{lemma}
\begin{proof}
Si \(𝑥\) et \(𝑦\) sont des ensembles vides, pour tout \(𝑧\) on a
\begin{equation*}
𝑧∈𝑥⟺\myfalse⟺𝑧∈𝑦
\end{equation*}
Donc tous les ensembles vides ont les mêmes éléments : ils sont égaux.
\end{proof}
%
\begin{axiom}
[Paire]
Si \(𝑥\) et \(𝑦\) sont des ensembles, alors il existe un ensemble contenant
\(𝑥\) et \(𝑦\), \ et eux seuls comme éléments. Il est noté \(\{𝑥,𝑦\}\).
\(∀𝑥\;\bigl(∀𝑦\;\bigl(∀𝑧\;(𝑧∈\{𝑥,𝑦\}⟹(𝑧=𝑥\myor𝑧=𝑦))\bigr)\bigr)\).
\end{axiom}
%
\begin{attention}
En théorie des ensembles, une paire peut avoir un seul élément : on n'a pas supposé que \(𝑥\) et \(𝑦\) étaient distincts.
\end{attention}
%
\begin{definition}
[Singleton]
Un \mykeyword{singleton} est une paire \(\{𝑥,𝑥\}\) où \(𝑥\) est un ensemble, il est noté \(\{𝑥\}\).
\end{definition}
%
\begin{lemma}
Un singleton n'est pas vide.
\end{lemma}
%
\begin{proof}
Par définition, \(𝑥∈\left\{𝑥\right\}\).
\end{proof}
\begin{theorem} 
Si \(𝑥\) est un ensemble, on a
\begin{equation*}
∀𝑦\;\bigl((𝑦∈\{𝑥\})⟹(𝑦=𝑥)\bigr),
\end{equation*}
et
\begin{equation*}
∀𝑦,𝑦'\;((𝑦∈\{𝑥\}\myand𝑦'∈\{𝑥\})⟹(𝑦=𝑦'))
\end{equation*}
\end{theorem}
%
\begin{proof}
Par l'axiome de la paire,
\begin{equation*}
∀𝑦\;(𝑦∈\{𝑥,𝑥\}⟹(𝑦=𝑥\myor𝑦=𝑥)⟹(𝑦=𝑥))
\end{equation*}
\(∀𝑦,𝑦'\;\bigl((𝑦∈\{𝑥\}\myand𝑦'∈\{𝑥\})⟹(𝑦=𝑥\myand𝑦'=𝑥)⟹(𝑦=𝑦')\bigr)\).
\end{proof}
%
\begin{theorem} 
Un ensemble non vide \(𝐸\) est un singleton si
\vskip\abovedisplayskip
\begin{enumerate}
\item
\(
∃𝑥\;\bigl(∀𝑦\;\bigl((𝑦∈𝐸)⟹(𝑦=𝑥)\bigr)\bigr)
\)
ou
\item
\(
∀𝑦,𝑦'\;\bigl((𝑦∈𝐸\myand𝑦'∈𝐸)⟹(𝑦=𝑦')\bigr)
\)
\end{enumerate}
\vskip\belowdisplayskip
Un ensemble \(𝐸\) est un singleton si
\(∃!𝑥(𝑥∈𝐸)\).
\end{theorem}
%
\begin{proof}
À compléter.
\par\noindent
\begin{enumerate}
\item La première proposition entraîne
\begin{equation*}
∀𝑦\ \bigl((𝑦∈𝐸)⟹(𝑦=𝑥)⟹(𝑦∈\{𝑥\})\bigr)
\end{equation*}
Avec \(𝑦_0\) dans \(𝐸\), elle entraîne aussi \(𝑦_0=𝑥\), ce qui donne \(𝑥∈𝐸\) puis
\begin{equation*}
∀𝑦\ \bigl((𝑦∈\{𝑥\})⟹(𝑦=𝑥)⟹(𝑦=𝑥\myand𝑥∈𝐸)⟹(𝑦∈𝐸)\bigr)
\end{equation*}
Par l'axiome d'extensionnalité, on obtient \(𝐸=\{𝑥\}\).
\item Soit \(𝑦_0\) dans \(𝐸\) , la deuxième proposition donne
\begin{equation*}
∀𝑦\ \bigl((𝑦∈𝐸)⟹(𝑦=𝑦_0)\bigr)
\end{equation*}
qui est la première proposition.
\(𝐸\) est non vide et l'unicité signifie exactement la dernière proposition.
\end{enumerate}
\end{proof}
%
\begin{axiom}
[Réunion]
Pour tout ensemble \(𝑥\), il existe un ensemble qui contient tous les éléments des éléments de
\(𝑥\) et eux seuls, il est noté \(\bigcup_{𝑦∈𝑥}𝑦\) :
\begin{equation*}
∀𝑥\;\left(
∀𝑧\;\left(
\vphantom{\displaystyle\bigcup}
\smash{𝑧∈\bigcup_{𝑦∈𝑥}}𝑦⟺∃𝑦\;(𝑧∈𝑦\myand𝑦∈𝑥)
\right)
\right)
\vphantom{\displaystyle\bigcup_{𝑦∈𝑥}}
\end{equation*}
\end{axiom}
%
\begin{axiom}
[Ensemble des parties]
Pour tout ensemble \(𝑥\), il existe un ensemble dont les éléments sont les sous-ensembles de
\(𝑥\) et eux seuls. Il est noté \(𝒫(𝑥)\).
\begin{equation*}
∀𝑥\;
\Bigl(∀𝑦\;
\bigl(
𝑦∈𝒫(𝑥)⟺∀𝑧\;(𝑧∈𝑦⇒𝑧∈𝑥)
\bigr)
\Bigr)
\end{equation*}
\end{axiom}
%
\begin{axiom}
[Infini]
Il existe un ensemble \(Ω\) qui contient \(\mathsf{ ∅}\) et tel que pour chacun de ses éléments
\(𝑥\), \(Ω\) contient \(𝑥∪\{𝑥\}\).
\begin{equation*}
∃Ω\;
\left(
(∅∈Ω)\myand
\left(∀𝑥\;(𝑥∈Ω⟹𝑥∪\{𝑥\}∈Ω)
\right)
\right)
\end{equation*}
\end{axiom}
%
\begin{axiom}
[Schéma de compréhension]
Pour tout ensemble \(𝐸\) et toute proposition dont l'écriture dépend formellement d'un paramètre \(𝑥\) notée
\(𝖯(𝑥)\), il existe un ensemble dont les éléments sont les éléments de \(𝐸\) vérifiant \(𝖯\).
\begin{equation*}
∀𝐸\;
\Bigl(∀𝖯(𝑥)\;
\bigl(∃𝐹\;
\bigl(∀𝑥\;(𝑥∈𝐹⟺(𝑥∈𝐸\myand𝖯(𝑥)))
\bigr)
\bigr)
\Bigr)
\end{equation*}
\end{axiom}
\begin{terminology}
On le note \(\bigl\{𝑥∈𝐸\mathbin|𝖯(𝑥)\bigr\}\), lu «ensemble des \(𝑥\) éléments de \(𝐸\) tels
que \(𝖯(𝑥)\)».
\end{terminology}
%
\begin{axiom}
[Schéma de remplacement]
Pour tout ensemble \(𝐸\) et toute relation fonctionnelle \(ℛ\), formellement définie comme une proposition \(𝑥ℛ𝑦\) et telle que \(𝑥ℛ𝑦\) et \(𝑥ℛ𝑦'\) entraîne 
\(𝑦=𝑦'\), il existe un ensemble contenant les images par \(ℛ\) des éléments de l'ensemble d'origine \(𝐸\) et elles seules.
\begin{equation*}
\left(
\vphantom{{A^A}^A}
∀𝐸\;\left(
\vphantom{A^{A^A}}
∀𝗑ℛ𝗒\;\left(
\vphantom{A^A}
∀𝑥\;\left(
\vphantom{A^A}
∀𝑦\;\left(∀𝑦'\;\left(𝑥ℛ𝑦\myand𝑥ℛ𝑦'⇒𝑦=𝑦'\right)\right)\right)\right)⟹...\right.\right.
\end{equation*}
\begin{equation*}
\left.\left....⟹
\left(∃Ω\;
\left(
∀𝑦\;((𝑦∈Ω)⇔∃𝑥\;((𝑥∈𝐸)\myand(𝑥ℛ𝑦)))
\vphantom{A^A}\right)
\vphantom{A^A}\right)
\vphantom{A^{A^A}}\right)
\vphantom{{A^A}^A}\right)
\end{equation*}
\end{axiom}
%
\begin{axiom}
[Fondation]
Tout ensemble \(𝐸\) non vide contient un élément \(𝑥\) qui ne contient aucun élément de
\(𝐸\).
\begin{equation*}
∀𝐸\;\left(
\vphantom{{A^A}^A}
\left(
\vphantom{A^A}
∃𝑥\;(𝑥∈𝐸)\right)
⟹
\left(
∃𝑥\;(∀𝑦\;(𝑦∈𝑥⇒𝑦∉𝐸))
\vphantom{A^A}
\right)
\vphantom{{A^A}^A}
\right)
\end{equation*}
\end{axiom}
%
\begin{axiom}
[Choix]
Étant donné un ensemble \(𝐸\) d'ensembles non vides mutuellement disjoints, il existe un ensemble qui
contient exactement un élément de chaque élément de \(𝐸\).
\end{axiom}
La formulation symbolique est laissée en exercice.
\begin{remark}
\(𝐸\) et \(𝐹\) sont des ensembles.
\end{remark}
\subsubsection{Comparaison}
\begin{definition}
[Inclusion des ensembles]
\begin{equation*}
𝐸⊂𝐹\mybydef{⟺}𝐹⊃𝐸\mybydef{⟺}∀𝑥\;(𝑥∈𝐸⇒𝑥∈𝐹)
\end{equation*}
On dit : \(𝐸\) est un \mykeyword{sous-ensemble} de \(𝐹\), \(𝐸\) est \mykeyword{inclus dans} \(𝐹\) ou \(𝐹\) \textbf{contient} \(𝐸\).
\mykeyword{Partie} est synonyme de sous-ensemble.
\end{definition}
%
\begin{definition}
[Égalité des ensembles]
\begin{equation*}
𝐸=𝐹\mybydef{⟺}∀𝑥(𝑥∈𝐸⇔𝑥∈𝐹)
\end{equation*}
\end{definition}
\begin{remark}
C'est en réalité un axiome de la théorie des ensembles, voir l'axiome \ref{seq:refAxiom1}
d'extensionnalité.
\end{remark}
%
\begin{theorem}
[Double inclusion]
\begin{equation*}
𝐸=𝐹⟺𝐸⊂𝐹\myand𝐹⊂𝐸
\end{equation*}\end{theorem}
\begin{proof}
On a \((𝑥∈𝐸⇔𝑥∈𝐹)⟺((𝑥∈𝐸⇒𝑥∈𝐹)\myand(𝑥∈𝐹⇒𝑥∈𝐸))\).

À compléter...
\end{proof}
%
\begin{definition}
[Partie stricte]
\begin{equation*}
𝐸⊊𝐹\mybydef{⟺}𝐹⊋𝐸\mybydef{⟺}∀𝑥\;(𝑥∈𝐸⇒𝑥∈𝐹)\myand∃𝑦\;(𝑦∉𝐸\text{ et
}𝑦∈𝐹)
\end{equation*}
On dit : \(𝐸\) est un \mykeyword{sous-ensemble strict} de \(𝐹\), \(𝐸\) est \textbf{inclus strictement dans} \(𝐹\) ou \(𝐹\)
\textbf{contient strictement} \(𝐸\).
\end{definition}
%
\begin{lemma} 
Il n'existe pas de partie stricte de \(∅\).
\end{lemma}
\begin{proof}
De la définition précédente, on déduit
\begin{equation*}
(∀𝑥\;(𝑥∈𝐸⇒𝑥∈𝐹)\myand∃𝑦\;(𝑦∉𝐸\myand𝑦∈𝐹))⇒∃𝑦\;(𝑦∈𝐹)⇒𝐹≠∅
\end{equation*}
d'où le résultat par contraposition.
\end{proof}
\subsubsection{Opérations binaires}
%
\begin{definition}
[Union et intersection]
Pour tout \(𝑥\)
\begin{equation*}
𝑥∈𝐸∪𝐹\mybydef{⟺}𝑥∈𝐸\myor𝑥∈𝐹
\end{equation*}
\begin{equation*}
𝑥∈𝐸∩𝐹\mybydef{⟺}𝑥∈𝐸\myand𝑥∈𝐹
\end{equation*}
\(∪\) se lit \mykeyword{union} ou \mykeyword{réunion}, \(∩\) se lit \mykeyword{inter} ou
\mykeyword{intersection}.
«L'union (ou la réunion) de \(𝐸\) et \(𝐹\)» désigne \(𝐸∪𝐹\), «l'intersection de \(𝐸\) et \(𝐹\)» désigne \(𝐸∩𝐹\).
\end{definition}
%
\begin{remark}%
Pour des définitions en compréhension, on a
\begin{equation*}
\left\{
\vphantom{A^A}
𝑥∈𝐸\middle|𝖯(𝑥)\right\}∪\left\{
\vphantom{A^A}
𝑥∈𝐸\middle|𝖯'(𝑥)\right\}
=\left\{
\vphantom{A^A}
𝑥∈𝐸\middle|𝖯(𝑥)\myor𝖯'(𝑥)\right\}
\end{equation*}
\begin{equation*}\left\{
\vphantom{A^A}
𝑥∈𝐸\middle|𝖯(𝑥)\right\}∩\left\{
\vphantom{A^A}
𝑥∈𝐸\middle|𝖯'(𝑥)\right\}
=\left\{
\vphantom{A^A}
𝑥∈𝐸\middle|𝖯(𝑥)\myand𝖯'(𝑥)\right\}
\end{equation*}
\end{remark}
%
\begin{theorem} 
Pour tout ensemble \(𝐸\) et toute proposition dont l'écriture dépend formellement d'un paramètre \(𝗑\) notée
\(𝖯(𝗑)\),
\begin{equation*}
\left\{
\vphantom{A^A}
𝑥∈𝐸\middle|𝖯(𝑥)\right\}∪\left\{
\vphantom{A^A}
𝑥∈𝐸\middle|\mynot𝖯(𝑥)\right\}=𝐸
\end{equation*}
\begin{equation*}
\left\{
\vphantom{A^A}
𝑥∈𝐸\middle|𝖯(𝑥)\right\}∩\left\{
\vphantom{A^A}
𝑥∈𝐸\middle|\mynot𝖯(𝑥)\right\}=∅
\end{equation*}
\end{theorem}
%
\begin{proof}
On a
\begin{align*}
𝑥∈\left\{
\vphantom{A^A}
𝑥∈𝐸\middle|𝙿(𝑥)\right\}∪
\left\{
\vphantom{A^A}
𝑥∈𝐸\middle|¬𝙿(𝑥)\right\}
&{}⟺\left(
\vphantom{A^A}
𝑥∈𝐸∧𝙿(𝑥)\right)∨
\left(
\vphantom{A^A}
𝑥∈𝐸∧¬𝙿(𝑥)\right)
\\
&{}⟺𝑥∈𝐸∧\left(
\vphantom{A^A}
𝙿(𝑥)∨¬𝙿(𝑥)
\right)
\\&{}⟺𝑥∈𝐸
\end{align*}
\begin{align*}
𝑥∈\left\{
\vphantom{A^A}
𝑥∈𝐸\middle|𝙿(𝑥)\right\}∩
\left\{
\vphantom{A^A}
𝑥∈𝐸\middle|¬𝙿(𝑥)\right\}
&{}⟺\left(\vphantom{A^A}
𝑥∈𝐸∧𝙿(𝑥)\right)
∧\left(\vphantom{A^A}
𝑥∈𝐸∧¬𝙿(𝑥)\right)
\\&{}⟺𝑥∈𝐸∧\left(
\vphantom{A^A}
𝙿(𝑥)∧¬𝙿(𝑥)\right)
\\&{}⟺𝑥∈∅
\end{align*}
\end{proof}
%
\begin{theorem}
[Propriétés]
\(∪\) et \(∩\) sont des opérations commutatives et associatives.
\(𝐸∩𝐹=𝐹∩𝐸\) et \(𝐸∪𝐹=𝐹∪𝐸\)
\(𝐸∩(𝐹∩𝐺)=(𝐸∩𝐹)∩𝐺\) et \(𝐸∪(𝐹∪𝐺)=(𝐸∪𝐹)∪𝐺\)
\begin{enumerate}
\item \(∅\) est élément neutre de \(∪\) et élément absorbant de \(∩,\)
\item \(∪\) est distributive par rapport à \(∩ :\)
\begin{equation*}
𝐸∪(𝐹∩𝐺)=(𝐸∪𝐹)∩(𝐸∪𝐺)
\end{equation*}
\item \(∩\) est distributive par rapport à \(∪ :\)
\begin{equation*}
𝐸∩(𝐹∪𝐺)=(𝐸∩𝐹)∪(𝐸∩𝐺)
\end{equation*}
\item \(∪\) et \(∩\) sont croissantes par rapport à chacun de leurs arguments :
\begin{equation*}
𝐹⊂𝐺⟹(𝐸∩𝐹⊂𝐸∩𝐺\myand𝐸∪𝐹⊂𝐸∪𝐺)
\end{equation*}
\end{enumerate}
\end{theorem}
%
\begin{proof}
\par\noindent
\begin{enumerate}
\item Pour la commutativité de la réunion : pour tout \(𝑥\)
\begin{equation*}
𝑥∈𝐸∪𝐹\mybydef{⟺}𝑥∈𝐸\myor𝑥∈𝐹⟺𝑥∈𝐹\myor𝑥∈𝐸\mybydef{⟺}𝑥∈𝐹∪𝐸
\end{equation*}
De même pour l'intersection par commutativité de \(\myand\) .
\item Pour l'associativité de la réunion : pour tout \(𝑥\)
\begin{align*}
𝑥∈𝐸∪(𝐹∪𝐺)
&{}⟺𝑥∈𝐸\;\myor\;𝑥∈𝐹∪𝐺
\\&{}⟺𝑥∈𝐸\;\myor\;(𝑥∈𝐹\;\myor\;𝑥∈𝐺)
\\&{}⟺(𝑥∈𝐸\;\myor\;𝑥∈𝐹)\;\myor\;𝑥∈𝐺
\\&{}⟺𝑥∈𝐸∪𝐹\;\myor\;𝑥∈𝐺
\\&{}⟺𝑥∈(𝐸∪𝐹)∪𝐺
\end{align*}
De même pour l'intersection par associativité de \(\myand\)
\item
Élément neutre : pour tout \(𝑥\)
\begin{equation*}
𝑥∈𝐸∪∅\mybydef{⟺}𝑥∈𝐸\myor𝑥∈∅⟺𝑥∈𝐸
\end{equation*}
On rappelle que \(𝑥∈∅\) est toujours \(\myfalse\).
\item
Élément absorbant : pour tout \(𝑥\)
\begin{equation*}
𝑥∈𝐸∩∅\mybydef{⟺}𝑥∈𝐸\myand𝑥∈∅⟺𝑥∈∅
\end{equation*}
\item Distributivités : deux versions en prenant d'abord l'opérateur au dessus, puis celui qui est dessous.
Pour tout \(𝑥\)
\begin{align*}
𝑥∈𝐸\myover{∪}{∩}\bigl(𝐹\myover{∩}{∪}𝐺\bigr)
&{}⟺𝑥∈𝐸\myorand𝑥∈𝐹\myover{∩}{∪}𝐺
\\&{}⟺𝑥∈𝐸\myorand\bigl(𝑥∈𝐹\myandor𝑥∈𝐺\bigr)
\\&{}⟺\bigl(𝑥∈𝐸\myorand𝑥∈𝐹\bigr)\myandor\bigl(𝑥∈𝐸\myorand𝑥∈𝐺\bigr)
\\&{}⟺𝑥∈𝐸\myover{∪}{∩}𝐹\myandor𝑥∈𝐸\myover{∪}{∩}𝐺
\\&{}⟺𝑥∈\bigl(𝐸\myover{∪}{∩}𝐹\bigr)\myover{∩}{∪}\bigl(𝐸\myover{∪}{∩}𝐺\bigr)
\end{align*}
\item Croissance : même convention que ci-dessus. On suppose \(𝐹⊂𝐺\). Pour tout \(𝑥\)
\begin{equation*}
𝑥∈𝐸\myover{∩}{∪}𝐹⟹𝑥∈𝐸\myandor𝑥∈𝐹⟹𝑥∈𝐸\myandor𝑥∈𝐺⟹𝑥∈𝐸\myover{∩}{∪}𝐺
\end{equation*}
\end{enumerate}
\end{proof}
%
\begin{definition} 
[Partition]
Une partition de \(𝐸\) est une partie \(𝑃\) de \(𝒫(𝐸)\) dont les éléments sont non vides,
mutuellement disjoints et ont pour réunion \(𝐸\), \textit{id est}
\begin{enumerate}
\vskip\abovedisplayskip
\item \hfill\(\displaystyle
∀𝐹∈𝑃\text{, }𝐹≠∅
\)\hfill~
\vskip\abovedisplayskip
\item\hfill\(\displaystyle
∀𝐹,𝐺∈𝑃\text{, }(𝐹≠𝐺)⟹(𝐹∩𝐺=∅)
\)\hfill~
\vskip\abovedisplayskip
\item \hfill\(𝐸=\bigcup_{𝐹∈𝑃}𝐹\).\hfill~
\vskip\belowdisplayskip
\end{enumerate}
Les éléments de \(𝑃\) sont ses \mykeyword{composantes}.
\end{definition}
%
\begin{definition}
[Différence]
Pour tout \(𝑥\)
\begin{equation*}
𝑥∈𝐸\mathbin{∖}𝐹\mybydef{⟺}𝑥∈𝐸\myand𝑥∉𝐹
\end{equation*}
Si \(𝐸\) contient \(𝐹\), \(𝐸\mathbin{∖}𝐹\) est le
\textbf{complémentaire} de \(𝐹\) dans \(𝐸\).
\end{definition}
\subsubsection[Produit cartésien]{Produit cartésien}
\begin{definition}
[Couple, composante]
Pour tous \(𝑥\) et \(𝑦\),\((𝒙,𝒚)\) est un \mykeyword{couple}.
\(𝑥\) et \(𝑦\) en sont les \mykeyword{composantes}, \(𝑥\) en est la
\mykeyword{première composante} et \(𝑦\) en est la \mykeyword{deuxième composante}. Pour tous
\(𝑥'\) et \(𝑦'\),
\begin{equation*}
(𝑥,𝑦)=(𝑥',𝑦')\mybydef{⟺}𝑥=𝑥'\myand𝑦=𝑦'
\end{equation*}
Dans le contexte de la géométrie, l'abscisse est synonyme de la première composante et l'ordonnée est synonyme de la deuxième composante.
\end{definition}
%
\begin{remark}
En théorie des ensembles avancée, on a \((𝑥,𝑦)\mybydef{⟺}\{\{𝑥\},\{𝑥,𝑦\}\}\).
L'existence du deuxième membre repose sur l'\emph{axiome de la paire}, son unicité repose sur l'\emph{axiome
d'extensionnalité}.
\end{remark}
%
\begin{definition}
[Produit cartésien]
\(𝐸\) et \(𝐹\) étant des ensembles, le \mykeyword{produit cartésien} de \(𝐸\)
par \(𝐹\), noté \(𝑬×𝑭\), est l'ensemble de tous les couples dont la première composante est dans \(𝐸\) et
la deuxième dans \(𝐹\).
\end{definition}
\begin{Exercise}
Montrer que le produit cartésien est associatif, mais pas commutatif.
\end{Exercise}
